%! Author = aybehrouz

The Argennon Smart Contract Execution Environment has two persistent memory areas: \emph{program area}, and \emph{heap}.
Program area stores the ASAR and the APM code of
applications\footnote{also it stores applications' constants.}, and heap stores heap chunks.
Both of these data elements can be considered as continuous arrays of bytes.
Throughout this chapter, we shall call these data elements \emph{chunks}.


\section{Storage Pages}\label{sec:storage-pages}

In the AscEE persistence layer, similar objects are clustered together and constitute a bigger data element which we
call a \emph{page}.\footnote{we avoid calling them clusters, because usually a cluster refers to a \emph{set}. AscEE
chunk clusters are not sets. They are ordered lists, like a page containing an ordered list of words or sentences.}
A page is an ordered list of an arbitrary number of chunks. Every page has a \emph{native} chunk that has the same
identifier as the page. In addition to the native chunk, a page can host any number of \emph{migrant} chunks.
A page of the AscEE storage should consists of chunks that have very similar access patterns. Ideally, when a page is
needed for validating a block, almost all of its chunks should be needed for either reading or writing. We prefer
that the chunks are needed for the same access type. In other words, the chunks of a page should be chosen in a way that
for validating a block, we need to either read all of them or modify\footnote{and probably read.} all of them.

When a page contains migrants, its native chunk can not be migrated. If the page does not
have any migrants, its native chunk can be migrated and after that the page will be converted into a special
\texttt{<moved>} page. When a non-native chunk is migrated to another page, it will be simply removed from the page.

\section{Publicly Verifiable Database Servers}\label{sec:zk-edb}

Pages of the AscEE storage are persisted using \emph{dynamic universal accumulators}. Argennon
has two dynamic accumulators: \emph{program} database, which stores the AscEE program area, and \emph{heap} database,
which stores the AscEE heap. The commitment of these accumulators are included in every block of the Argennon
blockchain. In the Argennon cloud, nodes that store these accumulators are called Publicly Verifiable Database
(PV-DB) servers.

We consider the following properties for a PV-DB:
\begin{itemize}
    \item The PV-DB contains a mapping from a set of keys to a set of values.
    \item Every state of the database has a commitment \(C\).
    \item The PV-DB has a method \((D, \pi) = \text{get}(x)\), where \(x\) is a key and \(D\) is the associated data
    with \(x\), and \(\pi\) is a proof.
    \item A user having \(C\) and \(\pi\) can verify that \(D\) is really associated with \(x\), and \(D\) is not
    altered. Consequently, a user who can obtain \(C\) from a trusted source does not need to trust the PV-DB\@.
    \item Having \(\pi\) and \(C\) a user can compute the commitment \(C'\) for the database in which \(D'\) is
    associated with \(x\) instead of \(D\).
\end{itemize}

The commitments of the AscEE cryptographic accumulators are affected by the way data chunks are clustered in pages.
Therefore, the Argennon clustering algorithm has to be a part of the consensus protocol.

Every block of the Argennon blockchain contains a set of \emph{clustering directives}. These directives
can only modify pages that were used for validating the block, and can
include directives for moving a chunk from one page to another or directives specifying which pages will contain
newly created chunks. These directives are applied at the end of block validation, after executing requests.

A block proposer is allowed to obtain clustering directives from any third party source\footnote{we can say the AscEE
clustering algorithm is essentially off-chain.}. This will not
affect Argennon security, since the integrity of a database can not be altered by clustering directives.
Those directives can only affect the performance of the Argennon network, and directives of a single block can
not affect the performance considerably.

\section{Object Clustering Algorithm}\label{sec:clustering}

\note{not yet written...}


