%! Author = aybehrouz


For implementing the persistence layer of the AVM, we assume that we have access to an updatable zero-knowledge
elementary database (ZK-EDB) with the following properties:

\begin{itemize}
    \item The ZK-EDB contains a mapping from a set of keys to a set of values.
    \item Every state of the database has a commitment \(C\).
    \item The ZK-EDB has a method \((D, p) = get(x)\), where \(x\) is a key and \(D\) is the associated data
    with \(x\), and \(p\) is a proof.
    \item A user can use \(C\) and \(p\) to verify that \(D\) is really associated with \(x\), and \(D\) is not
    altered. Consequently, a user who can obtain \(C\) from a trusted source does not need to trust the ZK-EDB\@.
    \item Having \(p\) and \(C\) a user can compute the commitment \(C'\) for the database in which \(D'\) is
    associated with \(x\) instead of \(D\).
\end{itemize}

We use a ZK-EDB for storing the AVM heap. We include the commitment of the current state of this DB in every
block of the Argennon blockchain, so ZK-EDB servers need not be trusted servers.

Every page of AVM heap will be stored with a key of the form: \texttt{applicationID|pageIndex} (the \texttt{|}
operator concatenates two numbers). Nodes do not keep a full copy of the AVM heap and for validating block
certificates or emulating the AVM ( i.e.~validating transactions) they need to connect to a ZK-EDB and retrieve
the required pages of AVM heap. For better performance, nodes keep a cache of heap pages to
reduce the amount of ZK-EDB access.

We also use a ZK-EDB for storing the code area of each segment, and we include the commitment of this DB in every
block. Every code area will be divided into blocks and every block will be stored in the DB with
\texttt{applicationID|blockID} as its key. Like heap pages, nodes keep a cache of code area blocks.

\note{Unlike heap pages, the AVM is not aware of different blocks of code area.}