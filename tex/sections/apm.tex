%! Author = aybehrouz
%! Date = 9/2/22

The Argennon Prover Machine (APM) is a virtual machine tailored for efficient verification of AscEE computations by
argument systems. The APM has a minimal RISC architecture with a very compact instruction set. This ensures that
its transition function has an efficient circuit complexity. Here by circuit complexity we mean the size
of the smallest circuit that, given two adjacent states in the trace, verifies that the transition between the two
states indeed respects the APM specification. The APM is a stack machine and has a random access key-value memory.
The word size of the APM is 64-bit.

The Argennon Prover Machine gets as its primary input a vector $(\mathbf{C}_H,\mathbf{C}_P,\mathbf{C}_R)$ where
$\mathbf{C}_H$ is a commitment to the AscEE Heap, $\mathbf{C}_P$ is a commitment to the AscEE program area and
$\mathbf{C}_R$ is a commitment to an ordered list of requests. The APM then outputs the updated
commitments to the AscEE heap and program area and a final $\mathbf{accept}$ flag which indicates if the execution has
ended successfully or not: $(\mathbf{C}_{H'},\mathbf{C}_{P'},\mathbf{accept})$.

Producing the required outputs from these inputs is not computationally feasible so the APM receives an auxiliary
input vector $(H,\pi_H,P,\pi_P,R,\pi_R,h)$, where $\pi_X$ is a proof that proves $X$ can be opened w.r.t
$\mathbf{C}_X$ and $h$ is a hint that helps the APM make nondeterministic choices correctly.

The APM consists of four main modules:
\begin{itemize}
    \item \textbf{Preprocessor:} This module prepares the input data for other modules. It verifies that $H,
    P,R$ are valid w.r.t the provided commitments. It also processes the input data and ensures it has the
    correct format and is valid. For example it verifies the signatures of authenticated messages or
    verifies that the proposed chunk bounds are valid.
    \item \textbf{Normal Execution Unit (NEU):} This module executes the requests whose execution completes
    normally. It outputs the updated heap chunks and an $accept$ flag. More formally it evaluates $(H',accept)=P(H,
    R_{good})$, where $H'$ is the updated heap chunks and the $accept$ flag will be set to false, If an
    application terminates abruptly.
    \item \textbf{Failure Repeater Unit (FRU):} This module evaluates $accept=P(H',R_{bad})$, where $R_{bad}$ is
    those requests whose execution terminates abnormally, $H'$ is the output of the NEU and the $accept$ flag is set
    to false if a request execution completes normally.
    \item \textbf{Postprocessor:} This module is responsible for calculating the updated commitments
    $(\mathbf{C}_{H'}, \mathbf{C}_{P'})$ and the final $\mathbf{accept}$ flag. Installing new applications or
    updating the code of existing application is performed by this module.
\end{itemize}


The configuration of the APM can be considered as a tuple
\[
(\mathcal{S},\mathcal{L}_{\text{NEU}},\mathcal{L}_{\text{FRU}},\mathcal{T}_{\text{NEU}},
\mathcal{T}_{\text{FRU}})\ ,
\]
where $\mathcal{S}$ is the size of the internal stack, $\mathcal{L}_{\text{NEU}},\mathcal{L}_{\text{FRU}}$ are the
amount of local memory of the normal execution and failure repeater units respectively, and
$\mathcal{T}_{\text{NEU}},\mathcal{T}_{\text{FRU}}$ are the number of cycles that the NEU and FRU runs for. It should
be noted that the APM does not use a different local memory for each application call.

%
%\begin{tikzpicture}
%    \draw (0,0) -- (4,0) -- (4,4) -- (0,4) -- (0,0);
%\end{tikzpicture}