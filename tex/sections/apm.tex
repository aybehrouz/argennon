%! Author = aybehrouz
%! Date = 9/2/22

\section{Introduction}\label{sec:introduction3}

The Argennon Prover Machine (APM) is a virtual machine tailored for efficient verification of AscEE computations by
argument systems. The APM has a minimal RISC architecture with a very compact instruction set. This ensures that
its transition function has an efficient circuit complexity. Here by circuit complexity we mean the size
of the smallest circuit that, given two adjacent states in the trace, verifies that the transition between the two
states indeed respects the APM specification. The APM is a stack machine and has a random access key-value memory.
The word size of the APM is 64-bit.

The Argennon Prover Machine gets as its primary input a vector $(\mathbf{C}_H,\mathbf{C}_P,\mathbf{C}_R)$ where
$\mathbf{C}_H$ is a commitment to the AscEE heap, $\mathbf{C}_P$ is a commitment to the AscEE program area and
$\mathbf{C}_R$ is a commitment to an ordered list of requests. The APM then outputs the updated
commitments to the AscEE heap and program area and a final $\mathbf{accept}$ flag which indicates if the execution has
ended successfully or not: $(\mathbf{C}_{H'},\mathbf{C}_{P'},\mathbf{accept})$.

Producing the required outputs from these inputs is not computationally feasible, so the APM receives an auxiliary
input vector $(H,\pi_H,P,\pi_P,R,\pi_R,h)$, where $\pi_X$ is a proof that proves $X$ can be opened w.r.t
$\mathbf{C}_X$ and $h$ is a hint that helps the APM make nondeterministic choices correctly.

The APM consists of four main modules:
\begin{itemize}
    \item \textbf{Preprocessor:} This module prepares the input data for other modules. It verifies that $H,
    P,R$ are valid w.r.t the provided commitments. It also processes the input data and ensures it has the
    correct format and is valid. For example, it verifies the signatures of authenticated messages or
    checks that the proposed chunk bounds are valid.
    \item \textbf{Normal Execution Unit (NEU):} This module executes the requests whose execution completes
    normally. It outputs the updated heap chunks and an $accept_N$ flag. More formally it evaluates
    $H',accept_N=P(H,R_{good})$, where $H'$ is the updated heap chunks and $accept_N$ flag will be set to false, If an
    application terminates abruptly.
    \item \textbf{Failure Repeater Unit (FRU):} This module evaluates $accept_F=P(H',R_{bad})$, where $R_{bad}$ is
    those requests whose execution terminates abnormally, $H'$ is the output of the NEU and $accept_F$ flag is set
    to false if a request execution completes normally.

    The FRU has a higher circuit complexity than the NEU\@. Unlike the NEU, the FRU is able to restore the initial state
    of the heap when an application fails. It also calculates the execution cost of every instruction and if the
    application's execution cost exceeds its predefined cap, the FRU will terminate the application.
    \item \textbf{Postprocessor:} This module is responsible for calculating the updated commitments
    $(\mathbf{C}_{H'}, \mathbf{C}_{P'})$ and the final $\mathbf{accept}$ flag. Installing new applications or
    updating the code of existing application is performed by this module.
\end{itemize}


The configuration of the APM can be considered as a tuple:
\[
(\mathcal{S},\mathcal{L}_{\text{NEU}},\mathcal{L}_{\text{FRU}},\mathcal{T}_{\text{NEU}},
\mathcal{T}_{\text{FRU}})\ ,
\]
where $\mathcal{S}$ is the size of the internal stack, $\mathcal{L}_{\text{NEU}},\mathcal{L}_{\text{FRU}}$ are the
amount of local memory of the normal execution and failure repeater units respectively, and
$\mathcal{T}_{\text{NEU}},\mathcal{T}_{\text{FRU}}$ are the number of cycles that the NEU and FRU runs for. It should
be noted that the APM does not use a different local memory for each application call.

\pagebreak


\section{Simplifying Complex Components}\label{sec:simplifying-complex-components}

Assume that we have a computing machine $M$, that accepts an input $in$ and produces an output $out$. We want to
calculate the output that results from applying $M$ on an initial input for $n$ times. Note that
$M$ may or may not be state-less:
\[
    M_n(in_1, out_1, out_2,\dots,out_{n-1}) \coloneqq out_n
\]

We assume that $M$ is composed of a number of subcomponents. Every subcomponent is like a computing machine and accepts
an input and produces an output. Like $M$, subcomponents are not necessarily state-less. When $M$ receives an input,
based on the input, a number of subcomponents gets activated, they generate their output from the input, and then
their outputs are combined to produce the output of $M$.

To build a verifying circuit for this computation, we need to construct a circuit for $M$ and then repeat this circuit
$n$ times to compute $out_n$ at the final step of the computation.\footnote{Since $M$ is not state-less, we may have
to feed inputs from steps, say, $i-1,i-2,\dots$ to step $i$.}

Assume that $M$ contains a subcomponent that has a considerable arithmetic circuit complexity.
Even if this subcomponent is
active in only a few computation steps, We will still have to repeat its circuit in every step of the computation. This
will increase the complexity of proof generation considerably. Fortunately, there is a workaround. Instead of
repeating the circuit of the subcomponent, we can use a simpler cryptographic hash calculator circuit during the
computation. At the end, we use a final verification circuit to verify the functionality of the component.

Every subcomponent accepts an input $in$ and produces an output $out$. If the component gets activated for $k$ steps, we
can denote it by a deterministic function that maps an input sequence with length $k$, to an output sequence with the
same length:\footnote{Note that subcomponents are not active in every step of the computation, so $in_i$ is \textbf{not}
indicating the input of the component at the $i$th step of the computation.}
\[
    f((in_1,in_2,\dots,in_k)) \coloneqq (out_1,out_2,\dots,out_k)
\]
In every step of the computation, we replace this component with a cryptographic hash calculator which receives $in$ and
$out$ as its inputs and gets activated in the same computation steps that the component must be active. When the
hash calculator is active, it hashes its inputs, so at the end of the computation it has computed a digest:
\[
    h(in_1,out_1,in_2,out_2,\dots,in_k,out_k) \coloneqq digest_f
\]
where $k$ is the number of steps that our component must have been active.

Now the prover needs to prove that he knows values $in_1,out_1,in_2,out_2,\dots,in_k,out_k$ such that they are correctly
produced by the component:
\[
    f((in_1,\dots,in_k)) = (out_1,\dots,out_k)
\]
and their digest is also correct:
\[
    h(in_1,out_1,\dots,in_k,out_k) = digest_f
\]
where functions $f$ and $h$, are both known to the prover and verifier.

The circuit for verifying this assessment usually is straight forward. When the computation involves a large
number of steps and the component is complex, this approach can reduce the cost of proof generation considerably. Memory
components are good candidates for being simplified by this method.

Interestingly, this approach can also reduce the number of computation steps. When we use a hash calculator circuit
instead of our component, $out$ will be available in the same computation step that $in$ is available. This will
eliminate one computation step that is needed for generating $out$ by the component's circuit.
%
%\begin{tikzpicture}
%    \draw (0,0) -- (4,0) -- (4,4) -- (0,4) -- (0,0);
%\end{tikzpicture}