%! Author = aybehrouz

\note{TODO: update this section!}

\subsection{Transaction Fee}\label{subsec:transaction-fee}

Every transaction in the Argennon blockchain starts with an \texttt{invoke\_external} instruction which calls a
special method from the root smart contract. This method will transfer the proposed fee of the transaction in ARGs
from a sender account to the fee sink accounts. Argennon has two fee sink accounts: \texttt{execFeeSink} collects
execution fees and \texttt{dbFeeSink} collects fees for ZK-EDB servers. The Protocol decides how to distribute the
transaction fee between these two fee sink accounts.

When a block is added to the blockchain, the proposer of that block will receive a share of the block fees.
Consequently, a block proposer is always incentivized to include more transactions in his block. However, if he
puts too many transactions in his block and the validation of the block becomes too difficult, some validators
may not be able to validate all transactions on time. If a validator can not validate a block in the required
time, he will consider the block invalid. So, when a proposed block contains too many transactions, the network
may reach consensus on another block, and the proposer of that block will not receive any fees. As a result, a
proposer is incentivized to use network transaction capacity optimally.

On the other hand, we believe that the proposer does not have enough incentives for optimizing the storage size
of the transaction set. Therefore, we require that \textbf{the size of the transaction set of every block in
bytes be lower than a certain threshold.}

Validators need to spend resources for validating transactions. When a validator starts the emulation of the AVM
to validate a transaction, solely from the code he can't predict the time the execution will finish. This will
give an adversary an opportunity to attack the network by broadcasting transactions that never ends. Since,
validators can not finish the execution of these transactions, the network will not be able to charge the
attacker any fees, and he would be able to waste validators resources for free.

To mitigate this problem, we require that every transaction specify a cap for all the resources it needs. This
will include memory, network and processor related resources. Also, the protocol defines an execution cost for
every AVM instruction reflecting the amount of resources its emulation needs. This will define a standard way for
measuring the execution cost of any \texttt{avmCall} transaction. Every \texttt{avmCall} transaction is required
to specify a maximum execution cost. If during emulation it reaches this maximum cost, the transaction will be
considered failed and the network can receive the proposed fee of that transaction.


\subsection{Incentives for ZK-EDB Servers}\label{subsec:zk-edb-servers}

The incentive mechanism for ZK-EDB servers should have the following properties:

\begin{itemize}
    \item It incentivizes storing all memory blocks, whether a heap page or a code area block, and not only those
    which are used more frequently.
    \item It incentivizes ZK-EDB servers to actively provide the required memory blocks for validators.
    \item Making more accounts will not provide any advantages for a ZK-EDB server.
\end{itemize}

For our incentive mechanism, we require that every time a validator receives a memory block from a ZK-EDB, after
validating the data, he give a receipt to the ZK-EDB. In this receipt the validator signs the following information:

\begin{itemize}
    \item \texttt{ownerAddr}: the ARG address of the ZK-EDB\@.
    \item \texttt{receivedBlockID}: the ID of the received memory block.
    \item \texttt{round}: the current round number.
\end{itemize}

\note{In a round, an honest validator never gives a receipt for an identical memory block to two different ZK-EDBs.}

To incentivize ZK-EDB servers, a lottery will be held every round and a predefined amount of ARGs from
\texttt{dbFeeSink} account will be distributed between winners as a prize. This prize will be divided equally
between all \emph{winning tickets} of the lottery.

\note{One ZK-EDB server could own multiple winning tickets in a round.}

To run this lottery, every round, based on the current block seed, a collection of \emph{valid} receipts will be
selected randomly as the \emph{winning receipts} of the round. A receipt is \emph{valid} in round $r$ if:

\begin{itemize}
    \item The signer was a validator in the round $r - 1$ and voted for the agreed-upon block.
    \item The data block in the receipt was needed for validating the \textbf{previous} block.
    \item The receipt round number is $r - 1$.
    \item The signer did not sign a receipt for the same data block for two different ZK-EDBs in the previous round.
\end{itemize}
For selecting the winning receipts we could use a random generator:
\begin{verbatim}
IF random(seed|validatorPK|receivedBlockID) < winProbability THEN
    the receipt issued by validatorPK for receivedBlockID is a winner
\end{verbatim}
\begin{itemize}
    \item \texttt{random()} produces uniform random numbers between 0 and 1, using its input argument as a seed.
    \item \texttt{validatorPK} is the public key of the signer of the receipt.
    \item \texttt{receivedBlockID} is the ID of the memory block that the receipt was issued for.
    \item \texttt{winProbability} is the probability of winning in every round.
    \item \texttt{seed} is the current block seed.
    \item \texttt{|} is a concatenation operator.
\end{itemize}

\note{The winners of the lottery were validators one round before the lottery round.}

Also, based on the current block seed, a random memory block, whether a heap page or a code area block, is
selected as the challenge of the round. A ZK-EDB that owns a winning receipt needs to broadcast a \emph{winning
ticket} to claim his prize. The winning ticket consists of a winning receipt and a \emph{solution} to the round
challenge. Solving a round challenge requires the content of the memory block which was selected as the round
challenge. This will encourage ZK-EDBs to store all memory blocks.

A possible choice for the challenge solution could be the cryptographic hash of the content of the challenge
memory block combined with the ZK-EDB ARG address: \texttt{hash(challenge.content|ownerAddr)}

The winning tickets of the lottery of round $r$ need to be included in the block of the round $r$,
otherwise they will be considered expired. Validation and prize distribution for the winning tickets of round
$r$ will be done in the round $r + 1$. This way, \textbf{the content of the challenge memory block could be
kept secret during the lottery round.} Every winning ticket will get an equal share of the lottery prize.

\subsection{Memory Allocation and De-allocation Fee}\label{subsec:memory-allocation-and-de-allocation}

Every $k$ round the protocol chooses a price per byte for AVM memory. When a smart contract executes a heap
allocation instruction, the protocol will automatically deduce the cost of the allocated memory from the ARG
address of the smart contract.

To determine the price of AVM memory, Every $k$ round, the protocol calculates \texttt{dbFee} and
\texttt{memTraffic} values. \texttt{dbFee} is the aggregate amount of collected database fees, and
\texttt{memTraffic} is the total memory traffic of the system. For calculating the memory traffic of the system
the protocol considers the total size of all the memory pages that were accessed for either reading or writing
during a time period. These two values will be calculated for the last $k$ rounds and the price per byte of
AVM memory will be a linear function of \texttt{dbFee/memTraffic}

When a smart contract executes a heap de-allocation instruction, the protocol will refund the cost of
de-allocated memory to the smart contract. Here, the current price of AVM memory does not matter and the protocol
calculates the refunded amount based on the average price the smart contract had paid for that allocated memory.
This will prevent smart contracts from profit taking by trading memory with the protocol.