%! Author = aybehrouz


\section{Introduction}\label{sec:introduction}

The Argennon Smart Contract Execution Environment (AscEE) is an abstract high level execution environment for executing
Argennon smarts contracts a.k.a\ Argennon applications in an efficient and isolated environment. An Argennon
application essentially is an HTTP server whose state is
kept in the Argennon blockchain and its logic is described using an Argennon Standard Application Representation (ASAR).

An Argennon Standard Application Representation is a programming language for describing Argennon
applications, optimized
for the architecture and properties of the Argennon platform. This text based representation is low level
enough to enable easy compilation from any high level programming language and is high level enough to preserve
the high level information of the application logic and facilitate platform specific compiler optimization at a host
machine. In this regard, the ASAR can be considered as an Intermediate Representation (IR).

The state of an Argennon application is stored in byte addressable finite arrays of memory called
\emph{heap chunks}. An application may have several heap chunks with different sizes, and can remove or
resize its heap chunks or allocate new chunks. Every chunk belongs to exactly one application and can only be modified
by its owner. Besides heap chunks, every application has a limited amount of non-persistent local memory for
storing temporary data.

The AscEE executes the requests contained in each block of the Argennon
blockchain in a three-step procedure. The first step is the \emph{preprocessing step}. In this step, the required
data for executing requests are retrieved and verified and the
helper data structures for next steps are constructed. This step is
designed in a way that can be done fully in parallel for each request without any risk of data races. The second
step is the \emph{Data Dependency Analysis (DDA) step}.
In this step by analyzing data dependency between requests, the AscEE determines requests that can be run in parallel
and requests that need to be run sequentially. This information is represented using an \emph{execution DAG} and in
the final step, the requests are executed using this data structure.


\section{Execution Sessions}\label{sec:sessions}

The Argennon Smart Contract Execution Environment can be seen as a machine for executing Argennon applications to
fulfill \emph{external} HTTP requests\footnote{External requests are requests that are not made by other Argennon
applications.}, produce their HTTP responses and update related heap chunks. The execution of requests can be
considered sequential\footnote{Actually requests may be executed in parallel but by performing data dependency analysis
the result is guaranteed to be identical with the sequential execution of requests.} and each request has a separate
\emph{execution session}. An execution session is a separate session of executing application's code in
order to fulfill an external HTTP request. This external request is the \emph{initiator} of the execution session.

The state of an execution session will be
destroyed at the end of the session and only the state of heap chunks is preserved. If a session fails and does not
complete normally, it will not have any effects on any heap chunks.

During an execution session an application can make \emph{internal} HTTP requests to other applications. These
requests will not start a new execution session and will be executed within the current session. In AscEE making an
internal HTTP request to an application is similar to a function invocation, and for that reason, we also refer to
them as \emph{application calls}.


\section{Memory}\label{mem}

Every Argennon application has two types of memory: local memory and heap. Local memory is not persistent and is
destroyed when the application call ends. Heap, on the other hand, is persistent and can be used for persisting data
between application calls. Local memory is used for storing local variables and is not directly
addressable. Heap is addressable and provides low level direct access. Both local memory and heap are limited, but
the limit is not specified by the AscEE. If an application tries to use too much
memory, that may cause the execution session to end abruptly. In that case, the execution session will not have any
effects on the state of heap.

\subsection{Heap Chunks}\label{subsec:heap}

The AscEE heap is split into chunks. Each heap chunk is a continuous finite array of bytes, has a unique identifier, and
is byte addressable. An application may have several heap chunks with different sizes and can remove or
resize its chunks or allocate new ones. Every chunk belongs to exactly one application. Only the owner application can
modify a chunk but there is no restrictions for reading a chunk\footnote{The reason behind this type of access
control design is the fact that smart contract
code is usually immutable. That means if a smart contract does not implement a
getter mechanism for some parts of its internal data, this functionality can never
be added later, and despite the internal data is publicly available, there will be no
way for other smart contracts to use this data on-chain.}.

When an application allocates a new heap chunk, the identifier of the new chunk is not generated by
the AscEE. Instead, the application can choose an identifier itself, provided the new identifier has a correct format.
This is an important feature of the AscEE heap which enables applications to use the AscEE heap as a map
data structure\footnote{also called a dictionary.}.
Since the \texttt{chunkID} is a prefix code, any application has its own identifier space, and an application
can easily find unique identifiers for its chunks. (See Section~\ref{sec:identifiers})

During an execution session every heap chunk has an access-type which may disallow certain operations during that
session. This access-type is declared by the initiator request of the execution session for every chunk:

\begin{itemize}
    \item \texttt{check\_only}: only allows check operations. These operations query the persistence
    status of a memory location.
    \item \texttt{read\_only}: only allows read and check operations.
    \item \texttt{writable}: allows reading and writing.
    \item \texttt{additive}: only allows additive operations. By additive, we mean an addition-like operator without
    overflow checking which is associative and commutative. Note that the content of the chunk cannot be read.
\end{itemize}

\subsubsection{Chunk Resizing}\label{subsubsec:ch-resize}

At the start of executing requests of a block, a validator can consider two values for every
heap chunk, the size: \texttt{chunkSize} and a size upper bound: \texttt{sizeUpperBound}. The value of
\texttt{chunkSize} can be determined uniquely at the start of
every execution session, and it may be updated during the session by the owner application. On the other hand,
the value of \texttt{sizeUpperBound} can be determined uniquely at the start of block validation and is proposed by the
block proposer for each block. This value is calculated based on resizing values
declared by external requests (i.e.\ transactions) that want to perform chunk resizing and needs to be
an upper bound of all the declared resizing values, indicating an upper bound of \texttt{chunkSize} during the
execution of the requests of a block.

The address space of a chunk starts from zero and only offsets lower than \texttt{sizeUpperBound} are valid. Trying to
access any offset higher than \texttt{sizeUpperBound} will always cause the execution session to end abruptly. The
value of \texttt{sizeUpperBound} is always greater than \texttt{chunkSize} and there is no way for an application to
query \texttt{sizeUpperBound}. As a result, in the view of an application,
accessing offsets higher than \texttt{chunkSize} results in undefined behaviour, while the behaviour is
well-defined in the view of a validator.
This enables validators to determine the validity of an offset at the start of the block validation in a
preprocessing phase without actually executing requests, while \texttt{sizeUpperBound} can be determined in a simple
parallelized algorithm.

The value of \texttt{chunkSize}, can be modified during an execution session. However, the new values of size can
only be increasing or decreasing. More precisely, if a request declares that it wants to expand (shrink) a chunk, it
can only increase (decrease) the value of \texttt{chunkSize} and any specified value during the execution
session, needs to be greater (smaller) than the previous value of \texttt{chunkSize}. Any request that wants to
expand (shrink) a chunk needs to specify a max size (min size). The value of \texttt{chunkSize} can not be set higher
(lower) than this value.

The value of \texttt{chunkSize} at the end of an execution session will determine if a memory location at an
offset is persistent or not: Offsets lower than the chunk size are persistent, and higher offsets are not.
Non-persistent locations will be re-initialized with zero at the start of every execution session.

Usually an application should not have any assumptions about the content of memory locations that are outside the chunk.
While these locations are zero initialized at the start of every execution session, multiple
invocations of an application may occur in a single execution session, and if one of them modifies a location outside
the chunk, the changes can be seen by next invocations.

While an application can use \texttt{chunkSize} to determine if an offset is persistent or not, that is not
considered a good practice. Reading \texttt{chunkSize} decreases transaction parallelization, and should be avoided.
Instead, applications should use a built-in AscEE function for checking the persistence status of memory addresses.

An application may load any chunk with a valid prefix identifier even if that chunk does not exist. For a non-existent
chunk the value of \texttt{chunkSize} is always zero.


\section{Identifiers}\label{sec:identifiers}

In Argennon a unique identifier is assigned to every application, heap chunk and account. Consequently, three distinct
identifier types exist: \texttt{appID}, \texttt{accountID}, and \texttt{chunkID}.
All these identifiers are \emph{prefix codes}, and hence can be represented by
\emph{prefix trees}\footnote{Also called tries.}.

Argennon has four primitive prefix trees: \emph{applications, accounts, local} and \emph{varUint}. All these trees
are in base 256, with the maximum height of 8.

An Argennon identifier may be simple or compound. A simple identifier is generated using a single tree, while a
compound identifier is generated by concatenating prefix codes generated by two or more trees:

\begin{itemize}
    \item \texttt{appID} is a prefix code built by the \emph{applications} prefix tree. An \texttt{appID} cannot
    be \texttt{0x0}.

    \item \texttt{accountID} is a prefix code built by the \emph{accounts} prefix tree. An \texttt{accountID} cannot
    be \texttt{0x0} or \texttt{0x1}.

    \item \texttt{chunkID} is a composite prefix code built by concatenating an \texttt{applicationID} to
    an \texttt{accountID} to a prefix code made by the \emph{local} prefix tree:
    \subitem \texttt{chunkID = (applicationID|accountID|<local-prefix-code>)} .
\end{itemize}

All Argennon prefix trees have an equal branching
factor \(\beta\)\footnote{A typical choice for \(\beta\) is \(2^8\).}, and we can represent an Argennon
prefix tree as a sequence of fractional numbers in base \(\beta\):
\[
    (A^{(1)},A^{(2)},A^{(3)},\dots)\ ,
\]
where \(A^{(i)}=(0.a_{1}a_{2}\dots a_{i})_\beta\), and we have \(A^{(i)} \leq A^{(i+1)}\). \footnote{It's possible to
have \(a_i=0\). For example, \(A^{(4)}=(0.2000)_{10}\) is correct.}

One important property of prefix identifiers is that while they have variable and unlimited length, they are
uniquely extractable from any sequence. Assume that we have a string of digits in base $\beta$, we
know that the sequence starts with an Argennon identifier, but we do not know the length of that identifier.
Algorithm~\ref{alg:prefix_id} can be used to extract the prefixed identifier uniquely. Also, we can apply this algorithm
multiple times to extract a composite identifier, for example \texttt{chunkID}, from a sequence.

%##\includegraphics[width=17cm]{../img/Alg1s.png}
\begin{algorithm}[t]
    \DontPrintSemicolon
    \SetKwInOut{Input}{input}\SetKwInOut{Output}{output}
    \Input{A sequence of $n$ digits in base $\beta$: $d_{1}d_{2}\dots d_{n}$ \newline
    A prefix tree: $<A^{(1)},A^{(2)},A^{(3)},\dots>$}
    \BlankLine
    \Output{Valid identifier prefix of the sequence.}
    \BlankLine
    \For{$i = 1$ \KwTo $n$}
    {
        \If{$(0.d_{1}d_{2}\dots d_{i})_\beta < A^{(i)}$}
        {
            \KwRet{$d_{1}d_{2}\dots d_{i}$}\;
        }
    }
    \KwRet{NIL}\;
    \caption{Finding a prefixed identifier}\label{alg:prefix_id}
\end{algorithm}

When we have a prefixed identifier, and we want to know if a sequence of digits is marked by that identifier,
we use Algorithm~\ref{alg:match_id} to match the prefixed identifier with the start of the sequence. The matching
can be done with only three comparisons, while invalid identifiers can be detected and will not match
any sequence.

In Argennon the shorter prefix codes are assigned to more active accounts and applications which tend to own more
data objects in the system. The prefix trees are designed by analyzing empirical data to make sure the number
of leaves in each level is chosen appropriately.

\begin{algorithm}[h]
    \DontPrintSemicolon
    \SetKwData{Id}{$id$}
    \SetKwInOut{Input}{input}\SetKwInOut{Output}{output}
    \Input{A prefixed identifier in base $\beta$ with $n$ digits: $\Id=a_{1}a_{2}\dots a_{n}$ \newline
    A sequence of digits in base $\beta$: $d_{1}d_{2}d_{3}\dots $ \newline
    A prefix tree: $<0,A^{(1)},A^{(2)},A^{(3)},\dots>$
    }
    \BlankLine
    \Output{$TURE$ if and only if the identifier is valid and the sequence starts with the identifier.}
    \BlankLine
    \If{$(0.a_{1}\dots a_{n})_\beta = (0.d_{1}\dots d_{n})_\beta$}
    {
        \If{$A^{(n-1)} \leq (0.a_{1}a_{2}\dots a_{n})_\beta < A^{(n)}$}
        {
            \KwRet{TRUE}\;
        }
    }
    \KwRet{FALSE}\;
    \caption{Matching a prefixed identifier}\label{alg:match_id}
\end{algorithm}

\section{Request Attachments}\label{sec:attachments}

The attachment of a request is a list of request identifiers of the current block that are \emph{attached} to the
request. That means, for validating this request a validator first needs to \emph{inject} the digest of attachments
into its HTTP request text. By doing so, the called application will have access to the digest of
attachments in a secure way, while it is ensured that the attached requests are included in the
current block.

The main usage of this feature is for fee payment. A request that wants to pay the fees for a number of requests,
declares those requests as its attachments. For paying fees the payer signs the digest of requests for which he
wants to pay fees. After injecting the digest of those request by validators, that signature can be validated by the
application that handles fee payment, and it is guaranteed that the attached requests are actually included in the
current block.


\section{Authorization}\label{sec:auth}

In blockchain applications, we usually need to authorize certain operations. For example, for sending an asset
from a user to another user, we need to make sure that the sender has authorized that operation.

The AscEE uses \emph{Authenticated Message Passing} for authorizing operations. In this method, every execution
session has a set of authenticated messages, and those messages are \textbf{explicitly} passed in requests to
applications for authorizing operations. These messages act exactly like digital signatures and applications can
ensure that they are issued by their claimed issuer accounts. The only difference is that the process of
message authentication is performed by the AscEE internally and an application does not explicitly verify cryptographic
signatures.

Each execution session has a list of authenticated messages. Each authenticated message has an index which will be
used for passing the message to an application as a request parameter. The AscEE uses cryptographic signatures to
authenticate messages for user accounts. The signatures are validated during the
preprocessing step in parallel, and any type of cryptographic signature scheme can be used.

Also, applications can use built-in functions of the AscEE to generate authenticated messages in run-time.
This enables an application to authorize operations for another application even if it is not calling that
application directly.

In addition to authenticated messages, the AscEE provides a set of
cryptographic functions for validating signatures and calculating cryptographic entities. By using these functions and
passing cryptographic signatures as parameters to methods, a programmer, having users' public keys, can implement
the required logic for authorizing operations.

Authorizing operations by Authenticated Message Passing and explicit signatures eliminates the need for approval
mechanisms or call back patterns in Argennon.\footnote{The AscEE has no instructions for issuing cryptographic
signatures.}


\section{Reentrancy Protection}\label{sec:reentrancy}

The AscEE provides optional low level reentrancy protection by providing low
level \emph{entrance locks}. When an application acquires an entrance lock it cannot acquire that lock again and trying
to do so will result in a revert. The entrance lock of an application will be released when the application explicitly
releases its lock or when the call that has acquired that lock completes.

The AscEE reentrancy protection mechanism is optional. An application can allow reentrancy, it can protect only certain
areas of its code, or can completely disallow reentrancy.


\section{Deferred Calls}\label{sec:deferred-calls}

\ldots



\section{The ArgC Language}\label{sec:the-argc-language}
\note{This section is outdated...}
\subsection{The ArgC Standard Library}\label{sec:asl}

In Argennon, some applications (smart contracts) are updatable. The ArgC Standard Library is an updatable smart
contract which can be updated by the Argennon governance
system. This means that bugs or security vulnerabilities in the ArgC Standard Library could be quickly patched and
applications could benefit from bugfixes and improvements of the ArgC Standard Library even if they are
non-updatable. Many important and useful functionalities,
such as fungible and non-fungible assets, access control mechanisms,
and general purpose DAOs are implemented in the ArgC Standard Library.

All Argennon standards, for instance ARC standard series, which defines standards regarding transferable assets,
are defined based on how a contract should use the ArgC standard library. As a result, Argennon standards are
different from conventional blockchain standards. Argennon standards define some type of standard logic and
behaviour for a smart contract, not only a set of method signatures. This enables users to expect certain type
of behaviour from a contract which complies with an Argennon standard.


\section{Data Dependency Analysis}\label{sec:concurrency}
%! Author = aybehrouz


\subsection{Memory Dependency Graph}\label{subsec:memory-dependency-graph}

Every block of the Argennon blockchain contains a list of transactions. This list is an ordered list and the
effect of its contained transactions must be applied to the AscEE state sequentially as they appear in the ordered
list.\footnote{This ordering is solely chosen by the block proposer, and users should not have any assumptions about
the ordering of transactions in a block.}

The fact that block transactions constitute a sequential list, does not imply they can not be executed and applied
to the AscEE state concurrently. Many transactions are actually independent and the order of their execution does not
matter. These transactions can be safely validated in parallel by validators.

A transaction can change the AscEE state by modifying either the code area or the AscEE heap. In Argennon, all
transactions declare the list of memory locations they want to read or write. This will enable us to determine the
independent sets of transactions which can be executed in parallel. To do so, we define the \emph{memory dependency
graph} \(G_d\) as follows:

\begin{itemize}
    \item \(G_d\) is an undirected graph.
    \item Every vertex in \(G_d\) corresponds to a transaction and vice versa.
    \item Vertices \(u\) and \(v\) are adjacent in \(G_d\) if and only if \(u\) has a memory location \(L\) in its
    writing list and \(v\) has \(L\) in either its writing list or its reading list.
\end{itemize}

If we consider a proper vertex coloring of \(G_d\), every color class will give us an independent set of
transactions which can be executed concurrently. To achieve the highest parallelization, we need to color \(G_d\)
with minimum number of colors. Thus, the \emph{chromatic number} of the memory dependency graph shows how good a
transaction set could be run concurrently.

Graph coloring is computationally NP-hard. However, in our use case we don't need to necessarily find an optimal
solution. An approximate greedy algorithm will perform well enough in most circumstances.

After constructing the memory dependency graph, we can use it to construct the
\emph{execution DAG} of transactions. The execution DAG of transaction set \(T\) is a directed acyclic
graph \(G_e = (V_e,E_e)\) which has the \emph{execution invariance} property:
\begin{itemize}
    \item Every vertex in \(V_e\) corresponds to a transaction in \(T\) and vice versa.
    \item Executing the transactions of \(T\) in any order that \emph{respects} \(G_e\) will result in
    the same AscEE state.
    \begin{itemize}
        \item An ordering of transactions of \(T\) respects \(G_e\) if for every directed edge \((u,v) \in E_e\)
        the transaction \(u\) comes before the transaction \(v\) in the ordering.
    \end{itemize}
\end{itemize}

Having the execution DAG of a set of transactions, using Algorithm~\ref{alg:exec_dag}, we can apply the transaction
set to the AscEE state concurrently, using multiple processor, while we can be sure that the resulted AscEE state will
always be the same no matter how many processor we have used.

%##\includegraphics[width=17cm]{../img/Alg1s.png}
\begin{algorithm}
    \DontPrintSemicolon
    \SetKwData{Ready}{$R_e$}\SetKwData{V}{$v_f$}\SetKwData{Graph}{$G_e$}\SetKwData{Vertices}{$V$}\SetKwData
    {Txns}{$T$}
    \KwData{The execution dag $\Graph = (\Vertices,E)$ of transaction set \Txns}
    \KwResult{The state after applying \Txns with any ordering respecting \Graph}
    \BlankLine
    \Ready $\gets$ the set of all vertices of \Vertices with in degree 0\;
    \While{$\Vertices \neq \varnothing$}
    {
        wait until a new free processor is available\;
        \If{the execution of a transaction was finished}
        {
            remove the vertex of the finished transaction \V from \Graph\;
            \For{each vertex $u \in Adj[\V]$}
            {
                \If{$u$ has zero in degree}
                {
                    $\Ready \gets \Ready \cup u$\;
                }
            }
        }
        \If{$\Ready \neq \varnothing$}
        {
            remove a vertex from \Ready and assign it to a processor\;
        }
    }
    \caption{Executing DAG transactions}\label{alg:exec_dag}
\end{algorithm}

By replacing every undirected edge of a memory dependency graph with a directed edge in such a way that the
resulted graph has no cycles, we will obtain a valid execution DAG. Thus, from a memory dependency graph different
execution DAGs can be constructed with different levels of parallelization ability.

If we assume that we have unlimited number of processors and all transactions take equal time for executing, it
can be shown that by providing a minimal graph coloring to Algorithm~\ref{alg:gen_dag} as input, the resulted
DAG will be optimal, in the sense that it results in the minimum overall execution time.

%##\includegraphics[width=17cm]{../img/Alg2s.png}
\begin{algorithm}
    \DontPrintSemicolon
    \SetKwData{Txns}{$T$}\SetKwData{Gd}{$G_d=(V_d,E_d)$}
    \SetKwInOut{Input}{input}\SetKwInOut{Output}{output}
    \Input{The memory dependency graph \Gd of transaction set \Txns\\A proper coloring of $G_d$}
    \Output{An execution dag $G_e=(V_e,E_e)$ for the transaction set \Txns}
    \BlankLine
    $V_e \gets V_d$\;
    $E_e \gets \varnothing$\;
    define a total order on colors of $G_d$\;
    \For{each edge $\{u,v\} \in E_d$}
    {
        \eIf{$color[u] < color[v]$}
        {
            $E_e \gets E_e \cup (u,v)$\;
        }{
            $E_e \gets E_e \cup (v,u)$\;
        }
    }
    \caption{Constructing an execution DAG}\label{alg:gen_dag}
\end{algorithm}

The block proposer is responsible for proposing an efficient execution DAG alongside his proposed block. This
execution DAG will determine the ordering of block transactions and help validators to validate transactions in
parallel. Since with better parallelization a block can contain more transactions, a proposer is incentivized enough
to find a good execution DAG for transactions.

\subsection{Memory Spooling}\label{subsec:spooling}

When two transactions are dependant and they are connected with an edge \((u,v)\) in the execution DAG,
the transaction \(u\) needs to be run before the transaction \(v\). However, if \(v\) does not read any
memory locations that \(u\) modifies, we can run \(u\) and \(v\) in parallel. We just need to make sure
\(u\) does not see any changes \(v\) is making in AscEE memory. This can be done by appropriate versioning
of the memory locations which is shared between \(u\) and \(v\). We call this method \emph{memory spooling}.
After enabling memory spooling between two transactions the edge connecting them can be safely removed from the
execution DAG\@.

\subsection{Concurrent Counters}\label{subsec:concurrent-counters}

We know that in Argennon every transaction needs to transfer its proposed fee to the \texttt{feeSink} accounts
first. This essentially makes every transaction a reader and a writer of the memory locations which store the
balance record of the \texttt{feeSink} accounts. As a result, all transactions in Argennon will be dependant and
parallelism will be completely impossible. Actually, any account that is highly active, for example the account
of an exchange or a payment processor, could become a concurrency bottleneck in our system which makes all
transactions interacting with them dependant.

This problem can be easily solved by using a concurrent counter for storing the balance record of this type of
accounts. A concurrent counter is a data structure which improves concurrency by using multiple memory locations for
storing a single counter. The value of the concurrent counter is equal to the sum of its sub counters and it can
be incremented or decremented by incrementing/decrementing any of the sub counters. This way, a concurrent
counter trades concurrency with memory usage.

Algorithm~\ref{alg:CC} implements a concurrent counter which returns an error when the value of the counter
becomes negative.

%##\includegraphics[width=17cm]{../img/Alg3s.png}
\begin{algorithm}
    \DontPrintSemicolon
    \SetKwData{CC}{Counter}
    \SetKwFunction{Inc}{Increment}\SetKwFunction{Dec}{Decrement}\SetKwFunction{AtomInc}{AtomicIncrement}
    \SetKwFunction{AtomDec}{AtomicDecrement}\SetKwFunction{AtomSet}{AtomicSet}\SetKwFunction{Get}{GetValue}
    \SetKwFunction{Acquire}{Lock.Acquire}\SetKwFunction{Release}{Lock.Release}
    \SetKwProg{Fn}{Function}{}{}
    \Fn{\Get{\CC}}
    {
        $s \gets 0$\;
    \Acquire{}\;
    \For{$i \gets 0$ \KwTo $\CC.size - 1$}
    {
        $s \gets s + \CC.cell[i]$\;
    }
    \Release{}\;
    \KwRet{s}\;
    }
    \BlankLine
    \Fn{\Inc{\CC, value, seed}}
    {
        $i \gets seed \bmod \CC.size$\;
    \AtomInc{$\CC.cell[i]$, value}\;
    }
    \BlankLine
    \Fn{\Dec{\CC, value, seed, attempt}}
    {
        \If {attempt = \CC.size}
        {
            restore \CC by adding back the subtracted value\;
            \KwRet{Error}\;
        }
        $i \gets seed \bmod \CC.size$\;
        $i \gets (i + attempt) \bmod \CC.size$\;
    \eIf {$\CC.cell[i] \geq value$}
    {
        \AtomDec{$\CC.cell[i]$, value}\;
    }{
        $r \gets value - \CC.cell[i]$\;
        \AtomSet{$\CC.cell[i]$, $0$}\;
        \Dec{\CC, r, seed, $attempt + 1$}\;
    }
    }
    \caption{Concurrent counter}\label{alg:CC}
\end{algorithm}

It should be noted that in a blockchain application we don't have concurrent threads and therefore we don't need
atomic functions. For usage in a smart contract, the atomic functions of this pseudocode can be implemented like
normal functions.

Concurrent counter data structure is a part of the ArgC standard library, and any smart contract can use this data
structure for storing the balance record of highly active accounts.
