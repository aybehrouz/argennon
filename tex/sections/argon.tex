%! Author = aybehrouz

\lstset{
    language=Java,
    morekeywords={main, initialize, constructor, external, virtual, Map, Account, signature, overrides, is, string},
    basicstyle=\small\sffamily,
    columns=fullflexible,
%   keywordstyle=\bfseries,
    breaklines=true,
    showstringspaces=false,
    mathescape
}

\section{Introduction}\label{sec:introduction2}

The Argon programming language is a class-based, object-oriented language designed for writing Argennon's smart
contracts. The Argon programming language is inspired by Solidity and is similar to Java, with a number of aspects
of them omitted and a few ideas from other languages included. Argon is designed to be fully compatible with
the Argennon Virtual Machine and be able to use all advanced features of the Argennon blockchain.

Argon applications (i.e.\ smart contracts) are organized as sets of packages. Each package has its own set of names
for types, which helps to prevent name conflicts. Every package can contain an arbitrary number of classes.
Every Argon application is required
to have exactly one \texttt{main} method and one \texttt{initialize} method. The \texttt{main} method is the
only method of an Argon application which can be called by other smart contracts.

The \texttt{main} method is required to have a single parameter named \texttt{request}. The type of this parameter
should be \texttt{RestRequest} or \texttt{HttpRequest}. The return value of the \texttt{main} function needs to be a
\texttt{RestResponse} or \texttt{HttpResponse}.


\section{Features Overview}\label{sec:features-overview}

\subsection{Access Level Modifiers}\label{subsec:access-level-modifiers}

Access level modifiers determine whether other classes can use a particular field or invoke a particular method or
if a method can be invoked externally by other smart contracts.

\begin{center}
    \begin{tabular}{lllll}
        \hline
        & Class & Package & Subclass & Program \\
        \hline
        private   & yes   & no      & no       & no  \\
        protected & yes   & no      & yes      & no  \\
        package   & yes   & yes     & yes      & no  \\
        public    & yes   & yes     & yes      & yes \\
        \hline
    \end{tabular}\label{tab:table}
\end{center}


\begin{lstlisting}[frame=TB, float, title=A simple Argon application,label={lst:code1}]
public class MirrorToken {
    private static SimpleToken token;
    private static SimpleToken reflection;

    // `initialize` is a special static method that is called by the AVM after the code of a contract
    // is stored in the AVM code area.
    public static void initialize(double supply1, double supply2) {
        // `new` does not create a new smart contract. It just makes an ordinary object.
        token = new SimpleToken(supply1);
        reflection = new SimpleToken(supply2);
    }
    // `main` is the only method of the application (i.e. smart contract) that can be called
    // by other applications. Every application should have exactly one main method defined
    // in some class. Alternatively, the keyword `dispatcher` could be used instead of `main`.
    public static RestResponse main(RestRequest request) {
        RestResponse response = new RestResponse();
        if (request.pathMatches("/balances/{user}")) {
            Account sender = request.getParameter<Account>("user");
            if (request.operationIsPUT()) {
                sender.authorize(request.toMessage(), request.getParameter<byte[]>("sig"));
                Account recipient = request.getParameter<Account>("to");
                double amount = request.getParameter<double>("amount");
                token.transfer(sender, recipient, amount);
                reflection.transfer(recipient, sender, Math.sqrt(amount));
                return response.setStatus(Http.Status.OK);
            } else if (request.operationIsGET()) {
                response.append<double>("balance", token.balanceOf(sender));
                response.append<double>("reflection", reflection.balanceOf(user));
                return response.setStatus(Http.Status.OK);
            } else {
                return response.setStatus(Http.Status.MethodNotAllowed);
            }
        }
    }
}

package class SimpleToken {
    private Map(Account -> double) balances;

    // The visibility of a member without an access modifier will be the package level.
    constructor(double initialSupply) {
        // initializes the object
    }

    void transfer(Account sender, Account recipient, double amount) {
        if (balances[sender] < amount) throw("Not enough balance.");
        // implements the required logic...
    }
    // implements other methods...
}
\end{lstlisting}

\subsection{Shadowing}\label{subsec:shadowing}

If a declaration of a type (such as a member variable or a parameter name) in a particular scope (such as an
inner block or a method definition) has the same name as another declaration in the enclosing scope, it will
result in a compiler error. In other words, the Argon programming language does not allow shadowing.
