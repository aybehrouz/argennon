%! Author = aybehrouz

The Argennon blockchain is a sequence of blocks. Every block represents an ordered list of external requests, intended
to be executed by the Argennon Smart Contract Execution Environment (AscEE). The first block of the blockchain, the
\emph{genesis} block, is a spacial block that fully describes the initial state of the AscEE and every block of the
Argennon blockchain thus corresponds to a unique AscEE state which can be calculated deterministically from the genesis
block.

A block of the Argennon blockchain contains the following information:

\begin{center}
    \begin{tabular}{||c||}
        \hline
        \textbf{Block} \\ [0.6ex]
        \hline\hline
        height: $h$                          \\ [1.2ex]
        commitment to the program database: $\mathbf{C}_{P}$             \\ [1.2ex]
        commitment to the heap database: $\mathbf{C}_{H}$               \\ [1.2ex]
        commitment to the ordered list of requests: $\mathbf{C}_{R}$        \\ [1.2ex]
        clustering directives: $dir$                        \\ [1.2ex]
        certificate of the validator assembly for \\
        the block with height $h - k$: $v \mhyphen cert_{h-k}$         \\ [1.2ex]
        previous block hash                           \\ [1.2ex]
        \hline
    \end{tabular}
\end{center}

\subsection{Block Certificate}\label{subsec:block-certificate}

An Argennon block certificate is an aggregate signature of some predefined subset of accounts. This predefined subset
is called the certificate assembly or committee and their signature ensures that the certified block is conditionally
valid given the validity of some previous block.

Because it is not usually possible to collect the signatures of all members of a certificate committee, an Argennon
block certificate essentially is an Accountable-Subgroup Multi-signature (ASM).

The Argennon network uses a parallel algorithm to produce block certificates and therefore the signature scheme needs
to satisfy certain properties:
\begin{itemize}
    \item \textbf{Associative aggregation:} the signature aggregation operator is associative.
    \item \textbf{Efficient cancellation:} if $S$ is a predefined and fixed set of users and $U$ is an arbitrary subset
    of~$S$, verifying an aggregate signature of $S-U$ can be done in time $O(T+|U|)$, if the aggregate
    signature of $S$ can be verified in $O(T)$.
\end{itemize}

An example for a signature scheme that supports all these properties is the BLS signature scheme.

\subsubsection{BLS Signatures}

The BLS signature scheme operates in a prime order group and supports simple threshold signature generation,
threshold key generation, and signature aggregation. To review, the scheme uses the following ingredients:

\newcommand{\G}{\mathbb{G}}
\newcommand{\Z}{\mathbb{Z}}
\newcommand{\adv}{{\cal A}}
\newcommand{\bdv}{{\cal B}}
\newcommand{\deq}{\mathrel{\mathop:}=}
\newcommand{\SK}{\mathit{sk}}
\newcommand{\PK}{\mathit{pk}}
\newcommand{\C}{\mathit{cert}}
\newcommand{\APK}{\mathit{apk}}
\newcommand{\DPK}{\mathit{\Delta pk}}
\newcommand{\MM}{\mathcal{M}}
\newcommand{\xwedge}{\, \operatorname{\text{$\wedge$}}\, }
\newcommand{\abs}[1]{\lvert #1 \rvert}
\newcommand{\Hm}{H_0}
\newcommand{\Hpk}{H_1}
\newcommand{\qHpk}{Q_{\Hpk}}
\newcommand{\qHm}{Q_{\Hm}}
\newcommand{\qsig}{Q_{\text{sig}}}

\begin{itemize}
    \item An efficiently computable \emph{non-degenerate} pairing $e:\G_0 \times \G_1 \to \G_T$
    in groups $\G_0$, $\G_1$ and $\G_T$ of prime order $q$. We let $g_0$ and $g_1$ be generators
    of $\G_0$ and $\G_1$ respectively.
    \item A hash function $H_0: \mathcal{M} \rightarrow \mathbb{G}_0$, where $\mathcal{M}$ is the message space.
    The hash function will be treated as a random oracle.
\end{itemize}

The BLS signature scheme is defined as follows:

\begin{itemize}
    \item $\textbf{KeyGen}()$: choose a random $\alpha$ from $\Z_q$ and set $h \gets g_1^\alpha \in \G_1$.
    output $\PK \deq (h)$ and $\SK \deq (\alpha)$.
    \item $\textbf{Sign}(\SK, m)$: output $\sigma \gets \Hm(m)^\alpha \in \G_0$.
    The signature $\sigma$ is a \emph{single} group element.
    \item $\textbf{Verify}(\PK,m,\sigma)$: if $e(g_1, \sigma) = e\big(\PK,\ \Hm(m)\big)$  then output ``accept'',
    otherwise output ``reject''.
\end{itemize}

Given triples $(\PK_i,\ m_i,\ \sigma_i)$ for $i=1,\ldots,n$,
anyone can aggregate the signatures $\sigma_1,\ldots,\sigma_n \in \G_0$
into a short convincing aggregate signature $\sigma$ by computing
\begin{equation}
    \label{eq:agg}
    \sigma \gets \sigma_1 \cdots \sigma_n \in \G_0\ \cdot
\end{equation}
Verifying an aggregate signature $\sigma \in \G_0$ is done by checking that
\begin{equation}
    \label{eq:aggdiff}
    e(g_1, \sigma) = e\big(\PK_1,\ \Hm(m_1)\big) \cdots e\big(\PK_n,\ \Hm(m_n)\big)\ \cdot
\end{equation}
When all the messages are the same ($m = m_1 = \ldots = m_n$), the verification relation~\eqref{eq:aggdiff} reduces to
a simpler test that requires only two pairings:
\begin{equation}
    \label{eq:aggsame}
    e(g_1, \sigma) = e\Big(\PK_1 \cdots \PK_n,\ \Hm(m)\Big)\ \cdot
\end{equation}
We call $\APK=\PK_1 \cdots \PK_n$ the aggregate public key.

To defend against \emph{rogue public key} attacks, Argennon uses Prove Knowledge of the Secret Key (KOSK) scheme. As we
explained in Section~\ref{sec:accounts}, when an account is created its public keys need to be registered in
the ARG smart contract. Therefore, the KOSK scheme can be easily implemented in Argennon.

Argennon uses a simple ASM scheme based on BLS aggregate signatures.
Argennon block certificates constitute an ordered sequence based on the order of blocks they certify. If we show
the $i$-th certificate\footnote{note that the $i$-th certificate is not
necessarily the certificate of the $i$-th block.} of committee $C$ with $\C_i$, and the set of signers
with $S_i$, then the block certificate $\C_i$ can be considered as a tuple:
\begin{equation}
    \C_i=(\sigma_i,\ C-S_{i})\label{eq:cert}\ ,
\end{equation}
where $\sigma_i$ is the aggregate signature issued by $S_i$.

The aggregate public key of the certificate can
be calculated from:
\begin{equation}
    \APK_i=\APK_C\APK_{C-S_i}^{-1}\label{eq:aggCertPK}\ ,
\end{equation}
where $\APK_{A}$ shows the aggregate public key of all accounts in $A$.

Alternately, we can use $\APK_{i-1}$ to calculate the aggregate public key:
\begin{equation}
    \APK_i=\APK_{i-1}\APK_{S_i-S_{i-1}}\APK_{S_{i-1}-S_i}^{-1}\ .\label{eq:aggPK-2}
\end{equation}

When an Argennon account is created, both its $\PK$ and $\PK^{-1}$ is registered in the ARG smart contract, so the
inverse of any aggregate public key can be easily computed.\footnote{since the group operator of a cyclic
group is commutative, we have $(ab)^{-1}=a^{-1}b^{-1}$.}

\subsection{Block Validation}\label{subsec:block-validation}

To validate a block three main conditions need to be validated: (i) commitments to the program and heap
database are resulted from applying the request list and clustering directives of the block to the previous AscEE
state, (ii) previous block is valid and has height $h - 1$, (iii) $v \mhyphen cert_{h-k}$ is valid.

We can denote condition (i) by a Computational Integrity (CI) statement:
\begin{equation}
    \label{eq:ci}
    \mathbf{C}_{P_h},\mathbf{C}_{H_h} \coloneqq \tau(\mathbf{C}_{P_{h-1}},\mathbf{C}_{H_{h-1}},\mathbf{C}_{R_{h}},
    dir_h)\ ,
\end{equation}
where $\tau$ is a transition function that encodes the AscEE computation logic and the necessary preprocessing and
postprocessing steps\footnote{including opening and updating the state commitments}.

Verifying statement~\ref{eq:ci} can be done by either replaying the AscEE computation and performing the required
preprocessing and postprocessing steps, or alternatively by
verifying a computational integrity proof\footnote{More accurately we should call it an argument instead of a proof.
However, using the word argument can confuse a reader who is not familiar with the subject, so we avoid it here.}.
To use computational integrity proofs the verifier and prover need to
share an arithmetized representation of $\tau$ and use it for both proof generation and verification.

The Argennon Prover Machine provides a convenient and universal way for arithmetization of any Argennon
application. Moreover, since a compiled version of every Argennon application to the APM code is stored in the AscEE
program area, validators that use the APM do not need to locally store the APM code of applications.

The Argennon protocol does not enforce the usage of the APM. Validators and PVC servers can use any argument system
with any arithmetization, and if required, the ASAR of an application can be used
for generating the appropriate arithmetization instead of the APM arithmetization.

For validating the previous block, instead of directly validating the contents of the block, a validator only
verifies the block certificates. Each block of the Argennon blockchain has two certificates: the certificate of
validators, $v \mhyphen cert$, and the certificate of delegates, $d \mhyphen cert$. Verifying $d \mhyphen
cert_{h-1}$ is straightforward but verifying $v \mhyphen cert_{h-1}$ is more challenging. $v \mhyphen cert_{h-1}$ is an
aggregate signature and validating it requires accessing the staking database at block~$h-1-m$, where $m$ is the
total number of validator assemblies, to obtain public keys and stake values. Again, in addition to direct
verification, a validator can use computational integrity proofs, received from the Argennon cloud, to perform
cheaper verification of this certificate.

The block at height~$h$ includes a certificate of validators for block~$h-k$ which is used to record the participation
of validators and facilitate reward calculation. This certificate needs to be validated based on stake and public key
database at block~$h-k-m$, and can be cheaply done by using computational proofs, obtained from the Argennon cloud.
Here $k$ is the maximum allowed length of the unvalidated part of the Argennon blockchain. (See
Section~\ref{subsec:validators-committee})

This type of block validation only validates the transition from block~$h-1$ to block~$h$, and the block is valid
only if its previous block is valid. We
call this type of block validation \emph{conditional block validation}, since the validity of the current block is
conditioned on the validity of the previous block.

Interestingly, conditional block validation of multiple blocks can be done in parallel. Moreover, the required proofs
can be generated independently and by different PVC servers. As we will see in
Section~\ref{subsec:validators-committee}, this property plays an important role in the Argennon consensus protocol.

In summary, a validator validates a block by verifying three computational integrity proofs: $\pi_{\tau}$ which proves
the transition is correct, $\pi_{h-1}$ which proves the previous block has a correct certificate of validators and
$\pi_{h-k}$ which proves the validity of the included certificate of validators of block~$h-k$. It should be noted that
these proofs are not part of the block contents and therefore the exact argument system used for generating them is
not specified by the Argennon protocol.