%! Author = aybehrouz

The credibility of a block of the Argennon blockchain is determined by the certificates it has received
from different sets of users, known as committees. There are two primary type of certificate committees in
Argennon: the committee of \emph{delegates} and the assembly of \emph{validators}. Argennon has \emph{one} committee
of delegates and $m$ assemblies of validators.

The committee of delegates issues a certificate for every block of the Argennon blockchain, and each
assembly of validators issues a certificate every $m$ blocks. A validators assembly will
certify a block only if it has already been certified by the committee of delegates. Every assembly of validators has
an index between $0$ and $m - 1$, and it issues a certificate for block number $n$, if $n$ modulo $m$ equals
the assembly index.

Every block of the Argennon blockchain needs a certificate from both the committee of delegates and
the assembly of validators. A block is considered final after its \textbf{next} block receives \textbf{both} of
its certificates. In Argennon as long as more than half of the total stake of validators is controlled by honest users,
the probability of discarding a final block is near zero even if all the delegates are malicious.

In addition to primary committees, Argennon has several community driven committees. Certificates of these
committees are not required for block finality, but they could be used by members of the
validators' committee to better decide about the validity of a block.

When an anomaly is detected in the consensus mechanism, the \emph{recovery} protocol is initiated by validators. The
recovery protocol is designed to be resilient to many types of attacks in order to be able to restore the normal
functionality of the system.

\subsection{The Committee of Delegates}\label{subsec:the-committee-of-delegates}

The committee of delegates is a small committee of trusted delegates, elected by Argennon users through the
Argennon Decentralized Autonomous Governance system (ADAGs\footnote{pronounced \textipa{/eI-dagz/}.}).
At the start of the Argennon mainnet, this committee will have
five members, and later its size could be changed by the ADAGs in a procedure described
in Section~\ref{sec:adags}.

The committee of delegates is responsible for creating new blocks of the Argennon blockchain, and it issues a
certificate for every block of the Argennon blockchain. A certificate needs to be signed
by \textbf{all} of the committee members in order to be considered valid.

Besides the main delegates' committee, a reserve committee of delegates consisting of three members is elected by users
either through the ADAGs or by \emph{emergency agreement} during the recovery protocol. In case the main committee
fails to generate new blocks or behaves maliciously, the task of
block generation will be assigned to the reserve committee until a new main delegates' committee is elected
through the ADAGs.

Usually, the delegates are large organizations, and they have enough computational resources to generate blocks
very fast. However, a block is not completely final if it does not have the certificate of its validators.
A certified block by the delegates will not be accepted by the network, if the last block certified by
the validators is behind it more than a certain number of blocks.

The committee of delegates may use any type of agreement protocol to reach consensus on the
next block. Usually a very simple and fast protocol can do the job: one of the members
is randomly chosen as the proposer, and other members vote "yes" or "no" on the proposed block.

If one of the delegates lose its network connectivity, no new blocks can be generated. For this reason,
the delegates should invest on different types of communication infrastructure, to make sure they never lose
connectivity to each other and to the Argennon network.

\subsection{Validators}\label{subsec:validators-committee}

The Argennon protocol calculates a stake value for every account, which is an estimate of a user's stake in the
system, and is measured in ARGs. Any account whose stake value is higher than
\texttt{minValidatorsStake} threshold is considered a \emph{validator}.
The \texttt{minValidatorsStake}
threshold is determined by the ADAGs, but it can never be higher than $500$ ARGs.

Every \texttt{committeeLifeTime} number of blocks, randomly $m$ committees are selected from
validators, in a way that the total stakes of committees are almost equal, and every
account is a member of \textbf{at least} one committee.

Every validator has a status which can be either \texttt{online} or \texttt{offline}.
This status is stored in the ARG smart contract and is a part of the staking database. A validator can change
his status through a method invocation
from the ARG smart contract. When an account sets its status to \texttt{offline}, it receives a small reward, and
it can not change it back to \texttt{online} for \texttt{statusCoolDown} number of blocks.

\note{When a validator changes his status, the change has no effect until the block containing the status
change transaction gets certified by his committee.}

A block certificate issued by some members of a validators assembly is considered valid, if according to
the staking database of the previous block \textbf{certified by the same committee}, we have:\footnote{If we calculate
the stake values based on the previous block a malicious committee can select the validators of the next block.}
\begin{itemize}
    \item The total stake of \texttt{online} members of the committee is higher than \texttt{minOnlineStake} fraction
    of the total stake of the committee. This threshold can be changed by the ADAGs, but it can never be lower
    than $2/3$.
    \item All signers of the certificate have \texttt{online} status.
    \item The sum of stake values of the certificate signers is higher than $3/4$ of the total stake
    of the committee members that have \texttt{online} status.
\end{itemize}

The delegates can generate blocks very fast. Therefore, the Argennon blockchain always has an
unvalidated part, which contains blocks that have a certificate from the committee of delegates but have not
yet received a certificate from the validators.

As we mentioned before, the block with height $n$ needs a certificate from the assembly of
validators with index $n$ modulo $m$. To decide about signing the certificate of a block which already has
a certificate from the delegates, a validator checks the conditional
validity\footnote{See Section~\ref{subsec:block-validation}} of the block, and
if the block is valid he issues
an "accept" signature. If the block is invalid, he initiates the recovery protocol. The validator will broadcast the
certificate \textbf{only after} he sees the certificate of the validators of the previous block.
Some validators may also require seeing a certificate from
some community driven committee. An honest validator never signs a certificate for two different blocks with the
same height.

So in Argennon, the block validation by committees is performed in parallel, and validators
do not wait for seeing the certificate of the previous block validators to start transaction validation. On the
other hand, the block certificates are published and broadcast sequentially. A validator does not publish its
certificate if the certificate of the previous block has not been published yet. This ensures that an invalid
fork made by malicious delegates will not receive any certificates from validators.

The value of $m$ is determined by the ADAGs. but it can never be higher than $25$. This way, it is guaranteed
that on average, any block of the Argennon blockchain is validated by at least $0.02$ of the total ARG supply.

block certificates issued by validators assemblies are included in the blocks of the Argennon blockchain.
A block can contain multiple certificates, provided that
those certificates belong to consecutive blocks.

\subsection{Status Blocks}\label{subsec:status-blocks}

If according to the staking database of block $n$, the total online stake of the committee with index $n$ modulo $m$ is
lower than \texttt{minOnlineStake} threshold, the block $n + m$ can never be certified by validators.

To prevent the blockchain from halting in such situations, the protocol performs a predefined partial
reshuffling of committee members.
In this reshuffling which is based on the block random seed, some online members from other committees will
be moved to the committee without enough online stake to make it active again.

If the reshuffling can not solve the problem due to low total online stake, the protocol requires the next block
of the blockchain to be a special \emph{status block}. A status block is a special block which can only contain
status change transactions. The status block needs to be certified by the delegates and by 2/3 of the total stake of
validators. The \texttt{online}/\texttt{offline} status of validators will not be considered in the validity of
the status block certificate.\footnote{Theoretically at the status block, the total online stake of the system could
be very low. Therefore, the status block should not be certified only by online stake.} After applying
the transactions of the status block, the total online stake of all committees of validators
must go higher than \texttt{minOnlineStake} threshold.

\subsection{Signature Aggregation}\label{subsec:sig-agg}

In Argennon, signature aggregation is mostly performed by ZK-EDB servers. To distribute the aggregation workload
between different servers, Every committee of validators is divided into pre-determined groups, and each ZK-EDB
server is responsible for signature aggregation of one group. To make sure that there is enough redundancy, the
total number of groups should be less than the number of ZK-EDB servers and each group should be assigned to
multiple ZK-EDB servers.

Any member of a group knows all the servers that are responsible for signature aggregation of his group. When a member
signs a block certificate, he sends his signature to \textbf{all} the severs that aggregate the signatures of his group.
These servers aggregate the signatures they receive and then send the aggregated signature to the delegates.
Furthermore, the delegates aggregate these signatures to produce the final block certificate
and then include it in the next block.

The role of the delegates in the signature aggregation is limited. The important part of the work is done by ZK-EDB
servers. As long as there are enough honest ZK-EDB servers, the network will be able to perform signature aggregation
even if the delegates are malicious.

\subsection{The Recovery Protocol}\label{subsec:recovery}

The recovery protocol is a resilient protocol designed for recovering the Argennon blockchain from critical situations.
In the terminology of the CAP theorem, the recovery protocol is designed to choose consistency over availability,
and is not a protocol supposed to be executed occasionally. Ideally this protocol should never be used
during the lifetime of the Argennon blockchain.

The recovery protocol can recover the functionality
of the Argennon blockchain as long as more than 2/3 of the total stake of the system is controlled by honest users
and any network partition resolves after a finite amount of time. The recovery protocol uses two main emergency
procedures to recover the functionality of the Argennon blockchain: \emph{emergency forking} and \emph{emergency
agreement} protocol.

\subsubsection{Emergency Forking}

The reserve committee of delegates is able to fork the Argennon blockchain, if it receives a valid fork request
from the validators.
This fork needs to be confirmed by validators and can not discard any blocks that has been already certified
by validators.
A valid fork request is an unexpired request signed by more than half of the
total \texttt{online} stake of the validators.

For forking at block $b$, the reserve committee of delegates
makes a special \emph{fork block} which only contains a valid fork request, and its parent is the block $b$.
The height of the fork block is
$b + 1$ so it needs a valid certificate from the committee of validator with the index $b+1$ modulo $m$.

For signing a fork block, a validator ensures that the block is signed by the reserve committee and contains
a valid fork request. The parent of the fork block does not necessarily need a validators' certificate. If the
parent does not have a certificate, the validator checks the certificate of the block before the parent and
the conditional validity of the parent instead. This enables the reserve committee
to recover the liveliness of
the blockchain in a situation where a malicious committee has generated multiple blocks at the same height.

A validator always chooses a valid fork block over a block of the main chain. However, as we mentioned before,
a validator never signs a certificate for two different blocks with the same height. Consequently, if a validator
has already signed the block $b$ of the main chain, he will not sign the fork block and vice versa.

When the fork block is broadcast, it is possible that the validators of the committee with index $b+1$ modulo $m$
get divided between the fork block and the block $b+1$ of the main chain, in a way that no block gets enough validators.
This will cause the blockchain to halt. To prevent this, the protocol allows the reserve delegates to revoke a
fork block. After a fork block
is revoked, the validators who voted for it are allowed to vote for another block with the same
height.

To revoke a fork block with height $b+1$, the delegates need to seal the fork by adding a special
\emph{seal block} after the fork block.\footnote{The parent of the seal block must be the fork block.} The seal block
has the height $b+2$ and needs to be certified by the validators' committee with the
index $b+2$ modulo $m$. The fork block is considered revoked only after the seal block is certified by the validators.

The seal block is not a normal block and validators who signed a certificate for a seal block are allowed
to sign a certificate for a block with the same height and vice versa. However, it should be noted that
generating a block with the same parent as the seal block is considered a malicious behaviour of the reserve
committee and validators will not sign such a block.

As long as more than half of the total online stake of every committee of validators is controlled
by honest users, a malicious committee of delegates can not use the emergency forking procedure to discard blocks
that have a certificate from validators.

Moreover, an honest committee of delegates will always try to perform the emergency forking in such a way that
valid blocks do not get discarded, including blocks that are not certified by the validators yet.

\subsubsection{Emergency Agreement}

The emergency agreement protocol is a resilient protocol for deciding between a set of proposals when no committee of
delegates can be trusted. For initiating the protocol, a validator signs a message containing the subject
of the agreement and a start time.

A validator enters the agreement protocol if he
receives a request that is signed by more than half of the total stake
of the validators and its start time has not passed. The validator calculates
the stake values based on the staking database of the last final block in his blockchain without
considering the \texttt{online}/\texttt{offline} status of validators.

The agreement protocol consists of two phases: the \emph{voting phase}, which selects a single proposal
and the \emph{confirmation phase} which confirms the selected proposal. The voting phase is done in rounds.
Each round lasts for approximately $\lambda$ units of time, and after $k$ rounds, the current agreement session ends and a
new session starts. All votes and messages are tagged in a way that the messages of one session can
not be used in another session.

Users cast three type of votes: \emph{$i$-votes}, which are votes that are valid only in
round $i$, \emph{final-votes}, which are votes that are valid in any round, and \emph{c-votes}, which are votes
used only in the confirmation phase.

A user executes the following procedure
in round $r$ of the voting phase:
\paragraph{Voting Phase:}
\begin{itemize}
    \item if the user has not yet final-voted any value, he $r$-votes a single desired proposal.
    \item if he sees more than 2/3 $r$-votes for a proposal $p$, he final-votes $p$.
    \item if he sees more than 2/3 final-votes for a proposal, he goes to the confirmation phase for $p$.
    \item if $clock > r \cdot \lambda$ and $r < k$ user goes to the round $r + 1$ and if $r = k$ user starts a
    new agreement session.
\end{itemize}
\paragraph{Confirmation Phase for Proposal $p$:}
\begin{itemize}
    \item user c-votes $p$.
    \item if he sees more than 2/3 c-votes for $p$ he selects $p$ and ends the agreement protocol.
    \item if $clock > k \cdot \lambda$ user starts a new agreement session.
\end{itemize}

We assume that users have clocks with the same speed, and $\lambda \gg \epsilon$, where $\epsilon$ is the maximum
clock difference between users. We also assume that more than 2/3 of the total stake of the system is controlled
by honest users, and network partitions are resolved after a finite amount of time. With these assumptions it can be
shown that the emergency recovery protocol has the following important properties:
\begin{itemize}
    \item no two users will end the agreement protocol with two different proposals as the result of the agreement.
    \item if honest users can agree upon some proposal value, the agreement protocol will converge to that value
    after a finite number of sessions.
\end{itemize}

When the emergency agreement protocol is used for electing a new reserve committee to fork the blockchain, the
confirmation phase could be skipped. In that case, the confirmation of the fork block by the appropriate committee
of validators acts like the confirmation phase.

\subsubsection{Initiating the Recovery Protocol}

When a validator does not receive any blocks for \texttt{blockTimeOut} amount of time, or when he sees an
evidence which proves the delegates are malicious, he initiates the recovery protocol.

To do so, first the validator activates the censorship resilient mode of his networking module, then he checks the
validity of the blocks that do not have a validators' certificate and determines the
last valid block of his blockchain.

In the next step, he will sign and broadcast an \textbf{emergency fork request} message, alongside some useful metadata
such as the last valid block of his blockchain and the evidence of delegates' misbehaviour.\footnote{this metadata is
not a part of the fork request.}

If the reserve committee of delegates is already active, or if a validator sees a valid fork request signed
by more than half of the total online stake of the validators, but does not receive the fork
block after a certain amount of time, he will sign and broadcast a request for
\textbf{emergency agreement} on a new reserve committee. The agreement on new delegates usually needs user
interaction and is not a fully automatic process.

The evidence which proves a committee of delegates is malicious is an invalid block that is signed by at least
one delegate:
\begin{itemize}
    \item a block that is not conditionally valid
    \item two different blocks with the same parent
    \item a block that has an invalid format
\end{itemize}

\subsection{Estimating Stake Values}\label{subsec:user's-stake}

In a proof of stake system the influence of a user in the consensus protocol should be proportional to the amount
of stake the user has in the system. Conventionally in these systems, a user's stake is considered to be equal with the
amount of native system tokens, he has "staked" in the system. A user stakes his tokens by locking them in
his account or a separate staking account for some period of time. During this time, he will not be able to transfer
his tokens.

Unfortunately, there is a subtle problem with this approach. It is not clear in a real world economic system
how much of the main currency of the system can be locked and kept out of the circulation indefinitely. It seems that
this amount for currencies like US dollar, is quite low comparing to the total market cap of the currency.
This means that for a real world currency this type of staking mechanism will result in putting the
fate of the system in the hands of the owners of a small fraction of the total supply.

To mitigate this problem, Argennon uses a hybrid approach for estimating the stake of a user.
Every \texttt{stakingDuration} blocks, which is called a \emph{staking period}, Argennon calculates
a \emph{trust value} for each user.

The user's stake
at time step \(t\), is estimated based on the user's trust value and his ARG balance:
\begin{equation}
    S_{u,t} = \min (B_{u,t}, Trust_{u,k})\ ,\label{eq:stake}
\end{equation}
where:
\begin{itemize}
    \item \(S_{u,t}\) is the stake of user \(u\) at time step \(t\).
    \item \(B_{u,t}\) is the ARG balance of user \(u\) at time step \(t\).
    \item \(Trust_{u,k}\) is an estimated trust value for user \(u\) at staking period \(k\).
\end{itemize}

Argennon users can lock their ARG tokens in their account for any period of time. During this time a user
will not be able to transfer his tokens and there is no way for cancelling a lock.
The trust value of a user is calculated based on the amount of his locked tokens and the
Exponential Moving Average (EMA) of his ARG balance:
\begin{equation}
    Trust_{u,k} = L_{u,k} + M_{u,t_k}\ ,\label{eq:trust}
\end{equation}
where
\begin{itemize}
    \item $L_{u,k}$ is the amount of locked tokens of user $u$, whose release time is \textbf{after the end} of
    the staking period $k+1$.
    \item $M_{u,t_k}$ is the Exponential Moving Average (EMA) of the ARG balance of user \(u\) at time step \(t_k\).
    $t_k$ is the start time of the staking period $k$.
\end{itemize}

In Argennon a user who held ARGs and participated in the consensus for a long time is more trusted
than a user with a higher balance whose balance has increased recently. An attacker who has obtained a large
amount of ARGs, also needs to hold them for a long period of time before being able to attack the system.

For calculating the EMA of a user's balance at time step \(t\), we can use the following
recursive formula:
\[
    M_{u,t} = (1 - \alpha) M_{u,t-1} + \alpha B_{u,t} = M_{u,t-1} + \alpha (B_{u,t} - M_{u,t-1})\ ,
\]
where the coefficient \(\alpha\) is a constant smoothing factor between \(0\) and \(1\), which represents the
degree of weighting decrease. A higher \(\alpha\) discounts older observations faster.

Usually an account balance will not change in every time step, and we can use older values of EMA for calculating
\(M_{u,t}\): (In the following equations the \(u\) subscript is dropped for simplicity)
\[
    M_{t} = (1 - \alpha)^{t-k}M_{k} + [1 - (1 - \alpha)^{t - k}]B\ ,
\]
where:
\[
    B = B_{k+1} = B_{k+2} = \dots = B_{t}\ .
\]
We know that when \(|nx| \ll 1\) we can use the binomial approximation \({(1 + x)^n \approx 1 + nx}\). So, we can
further simplify this formula:
\[
    M_{t} = M_{k} + (t - k) \alpha (B - M_{k})\ .
\]

For choosing the value of \(\alpha\) we can consider the number of time steps that the trust value of a user needs
for reaching a specified fraction of his account balance. We know that for large \(n\) and \(|x| < 1\) we have
\((1 + x)^n \approx e^{nx}\), so by letting \(M_{u,k} = 0\) and \(n = t - k\) we can write:
\begin{equation}
    \alpha =- \frac{\ln\left(1 - \frac{M_{n+k}}{B}\right)}{n}\ .\label{eq:alpha}
\end{equation}
The value of \(\alpha\) for a desired configuration can be calculated by this equation. For instance, we could
calculate the \(\alpha\) for a relatively good configuration in which \(M_{n+k} = 0.8B\) and \(n\) equals to the
number of time steps of 10 years.

\subsection{Analysis}\label{subsec:consensus-math}
\note{not yet written...}
