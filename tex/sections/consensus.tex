%! Author = aybehrouz

The credibility of a block of the Argennon blockchain is determined by the certificates it receives
from different sets of users, known as committees. There are two primary types of certificate committee in
Argennon: the committee of \emph{delegates} and the assembly of \emph{validators}. Argennon has \emph{one} committee
of delegates and $m$ assemblies of validators.

The committee of delegates issues a certificate for every block of the Argennon blockchain, and each
assembly of validators issues a certificate every $m$ blocks. A validator assembly will
certify a block only if it has already been certified by the committee of delegates. Every assembly of validators has
an index between $0$ and $m - 1$, and it issues a certificate for block with height $h$, if $h$ modulo $m$ equals
the assembly index.

Every block of the Argennon blockchain needs a certificate from both the committee of delegates and
the assembly of validators. A block is considered final after its \textbf{next} block receives \textbf{both} of
its certificates. In Argennon as long as more than $2/3$ of the total stake of validators is controlled by honest users,
the probability of discarding a final block is zero even if all the delegates are malicious.

Having multiple assemblies gives validators some ``resting'' time and allows a validator to be out of sync with the
network for some time, without losing the opportunity to vote for any blocks.

In addition to primary committees, Argennon could have several community driven committees. Certificates of these
committees are not required for block finality, but they could be used by members of validator assemblies to better
decide about the validity of a block.

When an anomaly is detected in the consensus mechanism, the \emph{recovery} protocol is initiated by validators. The
recovery protocol is designed to be resilient to many types of attacks in order to be able to restore the normal
functionality of the system.

\subsection{The Committee of Delegates}\label{subsec:the-committee-of-delegates}

The committee of delegates is a small committee of trusted delegates, elected by Argennon users through the
Argennon Decentralized Autonomous Governance system (ADAGs\footnote{pronounced \textipa{/eI-dagz/}.}).
At the start of the Argennon main-net, this committee will be elected for one-year terms and will have five members.
Later, this can be changed by the ADAGs in a procedure described in Section~\ref{sec:adags}.

The committee of delegates is responsible for creating new blocks of the Argennon blockchain, and issues a
certificate for every block of the Argennon blockchain. The certificate needs to be signed
by \textbf{all} the committee members in order to be considered valid.

Besides the main committee, a reserve committee of delegates consisting of three members is elected by validators
either through the ADAGs or by the \emph{emergency agreement} during the recovery protocol. In case the main committee
fails to generate new blocks or behaves maliciously, the task of
block generation will be assigned to the reserve committee until the main committee comes back online or a new
committee is elected through the ADAGs.

A block certified only by the committee of delegates is relatively credible, but it is not considered final
until its next block receives the certificate of its validator assembly. Since a block at height $h$ contains the
validators certificate of the block at height $h-k$, the unvalidated part of the Argennon blockchain can not be longer
than $k$ blocks.

The committee of delegates may use any type of agreement protocol to reach consensus on the
next block. Usually the delegates are large organizations, and they can communicate with each
other efficiently using their reliable networking infrastructure. This mostly eliminates the complications of their
consensus protocol and any protocol could have a good performance in practice. Usually a very simple and fast protocol
can do the job: one of the members is randomly chosen as the proposer, and other members vote ``yes'' or ``no'' on the
proposed block. For better performance, the delegates should run their agreement protocol for reaching consensus about
small batches of transactions in their mem-pools, instead of the whole block.

If one of the delegates loses its network connectivity, no new blocks can be generated until the reserve committee
gets activated. For this reason, the delegates should invest on different types of communication infrastructure, to
make sure they will never lose connectivity to each other and to the Argennon network.

\subsection{The Assemblies of Validators}\label{subsec:validators-committee}

The Argennon protocol calculates a stake value for every account, which is an estimate of a user's stake in the
system, and is measured in ARGs. Any account whose stake value is higher than
\texttt{minValidatorStake} threshold is considered a \emph{validator}.
The \texttt{minValidatorStake}
threshold is determined by the ADAGs, but it can never be higher than $1000$ ARG\@.

Every \texttt{AssemblyLifeTime} number of blocks, randomly $m$ assemblies are selected from
validators, in a way that the total stakes of different assemblies are almost equal, and every
account is a member of \textbf{at least} one assembly.\footnote{An account can be a member of multiple assemblies.}

The value of $m$ is determined by the ADAGs, but it can never be higher than $32$. This way, it is guaranteed
that on average, any block of the Argennon blockchain is validated by at least $2\%$ of the total ARG supply.

\subsubsection{Activity Status}

Every validator has a status which can be either \texttt{online} or \texttt{offline}.
This status is stored in the ARG application and is part of the staking database. A validator can change
his status to \texttt{offline} through an external request (transaction) to the ARG application. In this request he
exactly specifies for how long he wants to be offline and after this period his status will be automatically considered
\texttt{online} again. When a validator sets his status to \texttt{offline} for some period of time, he
will receive a small portion of the maximum possible reward that a validator can receive in that period of time by
actively participating in the consensus protocol. This ensures that a validator has an incentive for changing his
status to offline rather than simply becoming inactive.

The staking database of a validator assembly can be updated only by the assembly itself. That means, an external
request which changes the status of a validator can be included only in a block that is validated by the assembly of
that validator.

There is no transaction type for changing the status of a validator to \texttt{online}. A malicious committee
of delegates would be able to censor this type of transactions and prevent honest validators from coming back online.
For this reason the status of a validator is considered online automatically when the specified period of time for
being offline ends.

A block certificate issued by some members of a validators assembly is considered valid, if according to
the staking database of the previous block \textbf{certified by the same assembly}, we have:\footnote{If we calculate
the stake values based on the previous block, a malicious assembly can select the validators of the next block by
manipulating the staking database.}
\begin{itemize}
    \item The total stake of \texttt{online} members of the assembly is higher than \texttt{minOnlineStake} fraction
    of the total stake of the assembly. This threshold can be changed by the ADAGs, but it can never be lower
    than $2/3$.
    \item All signers of the certificate have \texttt{online} status.
    \item The sum of stake values of the certificate signers is higher than $3/4$ of the total stake
    of the assembly members that have \texttt{online} status.
\end{itemize}

If according to the staking database of block $h$, the total online stake of the assembly with index $h$ modulo $m$ is
lower than \texttt{minOnlineStake} threshold, the block $h + m$ can never be certified by validators. To prevent the
blockchain from halting in such situations, the validator assembly with index $h$ modulo $m$ will get merged into the
assembly that has the most online stake at block $h$. This will decrease the number of assemblies to $m-1$, and the
indices of assemblies will be updated appropriately.

The merging will continue recursively until the online stake of all remaining assemblies is higher than
\texttt{minOnlineStake} fraction.
If eventually all assemblies get merged together and only one assembly remains, the condition
for validity of block certificates changes: A certificate of validators will be considered valid if the sum of stakes
of the certificate signers is higher than $2/3$ of the total stake of validators and \texttt{online/offline}
status of validators becomes ineffective. This prevents the system from going into temporary deadlocks and the
community will always be able to preserve the liveliness without waiting for the expiration of offline status of
some accounts.

\subsubsection{Signing the Block Certificate}

The delegates can generate blocks very fast. Consequently, the Argennon blockchain always has an
unvalidated part which contains the blocks that have a certificate from the committee of delegates but have not
yet received a certificate from the validators.

As we mentioned before, the block with height $h$ needs a certificate from the assembly of
validators with index $h$ modulo $m$. To decide about signing the certificate of a block which already has
a certificate from the delegates, a validator checks the conditional
validity of the block (See Section~\ref{subsec:block-validation}), and if the block is valid, he issues
an ``accept'' signature. If the block is invalid, he initiates the recovery protocol. The validator will broadcast
the signature \textbf{only after} he sees the certificate of the validator assembly of the previous block.
Some validators may also require seeing a certificate from
some community based committees. An honest validator never signs a certificate for two different blocks with the
same height.

Consequently, in Argennon the block validation by assemblies is performed in parallel, and validators
do not wait for seeing the validators certificate of the previous block to start block validation. On the
other hand, the block certificates are published and broadcast sequentially. A validator does not publish his vote,
if the certificate of the validator assembly of the previous block has not been published yet. This ensures
that an invalid fork made by malicious delegates can not receive any valid certificates from any validator assemblies.

\subsubsection{Analysis}

We analyze the minimum amount of stake that is required for conducting different types of attacks against the Argennon
blockchain. In these attack scenarios, we assume that a single validator assembly is corrupted, all the delegates
are malicious and the adversary is able to fully control message transmission between nodes and partition the network
arbitrarily.

We denote the total stake of the corrupted validator assembly with $s$ and the total stake of malicious users of the
assembly with $m$. We use $d$ to denote the stake of users of the assembly who have \texttt{offline} status and $h$
to denote the stake of users of the assembly who have \texttt{online} status and do not participate in the
protocol.\footnote{$d$ stands for \emph{deactivated} and $h$ stands for \emph{hidden}.} We assume
that a certificate is accepted if it is signed by more than $r$ fraction of the total online stake of the assembly.
We obtain the minimum required malicious stake for three types of attacks:
\begin{itemize}
    \item Certifying an invalid block:
    \[ m > r(s-d) \]
    \item Forking the blockchain by double voting and network partitioning:
    \[m > (2r-1)(s-d)+h \]
    \item Halting the blockchain by refusing to vote:
    \[m > (1-r)(s-d)-h \]
\end{itemize}

In Argennon we have $r=\frac{3}{4}$ and $d < \frac{1}{3}s$.
Consequently, in Argennon to confirm an invalid block, the adversary needs at least $\frac{1}{2}$ of the total stake
of an assembly. For forking the blockchain, interestingly $m$ is
minimized when $h=0$, thus the minimum required stake is $\frac{1}{3}s$. For halting the blockchain,
an adversary requires a stake higher than $\frac{1}{6}s-h$.

In particular, we are interested in comparing the Argennon protocol with a simple protocol that accepts a certificate
if it is signed by more than $\frac{2}{3}$ of the total stake of the assembly and there is no \texttt{online/offline}
status for users.

We observe that the minimum required stake for halting the blockchain in the Argennon protocol will be higher, as
long as the
following inequality holds:
\[
    \frac{1}{4}(s-d)-h > \frac{1}{3}s-(d+h)\ \cdot
\]
So if $d>\frac{1}{9}s$, the minimum required stake for halting the blockchain is higher in the
Argennon protocol and the value of $h$ does not matter.

\subsubsection{Signature Aggregation}\label{subsubsec:sig-agg}

The validators certificate of a block is an aggregate signature of members of the corresponding assembly of
validators. Validator assemblies could include millions of users and calculating their aggregate
signature requires an efficient distributed algorithm.

In Argennon, signature aggregation is mostly performed by PVC servers. To distribute the aggregation workload
between different servers, every validator assembly is divided into pre-determined groups, and each PVC
server is responsible for signature aggregation of one group. To make sure that there is enough redundancy, the
total number of groups should be less than the number of PVC servers and each group should be assigned to
multiple PVC servers.

Any member of a group knows all the servers that are responsible for signature aggregation of his group. When a member
signs a block certificate, he sends his signature to all the severs that aggregate the signatures of his group.
These PVC servers aggregate the signatures they receive and then send the aggregated signature to the delegates.
Furthermore, the delegates aggregate these signatures to produce the final block certificate
and then broadcast it to the PVC servers network.

The role of the delegates in the signature aggregation algorithm is limited. The important part of the work is done by
PVC servers and slightly modified versions of this algorithm can perform signature aggregation
even if all the delegates are malicious, as long as there are enough honest PVC servers.

\subsection{The Recovery Protocol}\label{subsec:recovery}

The recovery protocol is a resilient protocol designed for recovering the Argennon blockchain from critical situations.
In the terminology of the CAP theorem, the recovery protocol is designed to choose consistency over availability,
and is not a protocol supposed to be executed occasionally. Ideally this protocol should never be used
during the lifetime of the Argennon blockchain.

We assume that an adversary is able to fully control message transmission between users and is able to partition the
network arbitrarily for finite periods of time. Under these circumstances, the recovery protocol can recover the
functionality of the Argennon blockchain as long as more than $2/3$ of the total stake of every validator assembly is
controlled by honest users. The recovery protocol uses two main emergency procedures to
recover the functionality of the Argennon blockchain: the \emph{emergency forking} and \emph{emergency
agreement} protocol.

\subsubsection{Emergency Forking}

The reserve committee of delegates is able to fork the Argennon blockchain, if it receives a valid fork request
from the validators.
This fork needs to be confirmed by validators and can not discard more than one block that has been certified by
validators.
A valid fork request is an unexpired request signed by more than half of the total stake of validators.

For forking at block $h$, the reserve committee of delegates
makes a special \emph{fork block} which only contains a valid fork request, and its parent is the block $h$.
The height of the fork block therefore is $h + 1$, and the fork block needs a valid certificate from the assembly of
validators with index $h+1$ modulo $m$. When a
fork block gets certified by validators, its parent is also confirmed and will become a part of the blockchain, even if
it does not have a certificate of validators.

For signing a fork block at height $h+1$, a validator ensures that the following conditions hold:
\begin{itemize}
    \item the fork block is signed by the reserve committee.
    \item the fork block contains a valid fork request.
    \item the parent block of the fork block is issued by the previous committee.
    \item the parent block of the fork block is certified by validators, or the parent block is conditionally
    valid and there is a fork block with height $h$ which is certified by validators, or the parent block is
    conditionally valid and the parent block of the parent has a validators certificate.
    \item the validator has not already signed a certificate for a fork or normal block at height $h+1$.
\end{itemize}

The parent of the fork block does not necessarily need a validators certificate. This enables the
reserve committee to recover the liveliness of the blockchain in a situation where a malicious committee has
generated multiple blocks at the same height. Notice that the block before the parent always needs a validators
certificate.

A validator always chooses a valid fork block over a block of the main chain and may sign different fork
blocks with different heights. However, as we mentioned before, an honest validator
\textbf{never signs a certificate for two different blocks with the same height}. Consequently, a validator never
signs two fork blocks at the same height, and if he has already signed the fork block at height $h+1$, he will not
sign the block $h+1$ of the main chain and vice versa.

The reserve committee of delegates is allowed to generate multiple fork blocks with different heights, as long as the
parent block is generated by the previous committee. When the reserve committee generates multiple fork blocks
at different heights, the next normal block must be always added after the fork block with the highest height.

The reserve committee of delegates should try to perform the emergency forking in such a way that
valid blocks do not get discarded, including blocks that have not been certified by the validators yet.

For forking the blockchain, the reserve committee uses a straightforward algorithm: let $h_v$ be the height of the last
block with a validator certificate and $h_v+k$ be the height of the last valid block that the reserve committee has
seen. For forking the main chain, the reserve committee generates all fork blocks with heights~$h_v+1,h_v+2,\dots,
h_v+k+1$. The parent of the fork block with height $h_v+i$ will be the block~$h_v+i-1$ of the main chain. The
reserve committee will wait until the fork block with height~$h_v+k+1$ receives a certificate from validators and
then will continue the normal chain after that fork block. Hence, the fork block with height~$h_v+k+1$
will be the parent of the first normal block generated by the reserve committee.

\paragraph{Analysis}

When the reserve committee gets activated, the main committee might have been malicious, so any number of blocks could
exist at each height. However, at each height at max one block can have a validators certificate. Moreover, if at some
height there are not any blocks with a validators certificate, then no blocks at higher heights can have a
validators certificate either, because validators do not sign the certificate of a normal block if its parent does
not have a certificate.

If $h_{\max}$ denotes the height of the highest block with a validators certificate, as long as more than $2/3$ stake
of every assembly of validators is honest, for a fork block with height $h_f$ we have:\footnote{This fork block
forks the blockchain at height $h_f-1$.}
\begin{itemize}
    \item if $h_f \leq h_{\max}$, the fork block can not receive a certificate from validators.
    \item if $h_f = h_{\max} + 1$, when there is no main block with height~$h_f$, the fork block can always receive a
    certificate from validators, otherwise the fork block may receive a certificate or not. It is possible that
    the validators of the assembly with index $h_f$ modulo $m$ get divided between the fork block and a block at
    height $h_f$ of the main chain.
    \item if $h_f = h_{\max} + 2$, the fork block can always receive a certificate from validators, if network
    partitions do not last forever.
    \item if $h_f \geq h_{\max} + 3$, the fork block can always receive a certificate from validators \textbf{only if} a
    fork block at height $h_f-1$ gets certified by validators.
\end{itemize}

The reserve committee does not know the value of $h_{\max}$. However, if the reserve committee creates all fork blocks
with heights~$h_0,h_0+1,h_0+2,\dots$ every fork block with height $h \geq h_{\max} + 2$ will surely get a certificate.
If there is no block in the main chain with height~$h_{\max} + 1$, then the fork block with height~$h_{\max} +
1$ will also get a certificate, otherwise the reserve committee should be able to eventually find the main block
at $h_{\max}+1$ and generate a fork block with height~$h_{\max} + 2$ as its successor. This way, the reserve
committee will always be able to create at least one fork block which gets certified and continue the chain after the
fork block with the highest height.

If two fork blocks at heights $h_0$ and $h_0+k$ are generated by the reserve committee and both blocks receive a
certified from validators, we must have $h_0 > h_{\max}$, and there must exist fork blocks with heights
$h_0+1,\dots,h_0+k-1$ which are certified by validators. That means if a malicious reserve committee generates a normal
block after any fork block with height less than $h_0+k$, that normal block can not receive a certificate from
validators.

If a malicious reserve committee creates two fork blocks with the same height, either only one of
them can get a certificate, which makes the other one ineffective or validators get split between blocks
and the chain will halt. In this case the emergency agreement protocol will start.

During the emergency forking, at max one block with a validators certificate may be discarded. This happens when
the malicious main committee of delegates has forked the main chain by producing blocks $b_1$ and $b_2$ at height
$h$; the block $b_1$ has received a validators certificate and validators have not certified any blocks at height $h+1$.
The reserve committee adds a fork block whose parent is block $b_2$, that will essentially discard $b_1$. Notice
that if a block at height $h+1$ had been certified by validators the fork block after $b_2$ could not have received a
certificate from validators.

\subsubsection{Emergency Agreement}

The emergency agreement protocol is a resilient protocol for deciding between a set of proposals when no committee of
delegates can be trusted. For initiating the protocol, a validator signs a message containing the subject
of the agreement and a start time.

A validator enters the agreement protocol if he
receives a request that is signed by more than half of the total stake
of the validators and its start time has not passed. The validator calculates
the stake values based on the staking database of the last \textbf{final} block in his blockchain without
considering the \texttt{online}/\texttt{offline} status of validators.

The emergency agreement protocol is essentially an election procedure and involves human interaction. Users need to
determine who they want to vote for by interacting with the software. As long as users can not agree upon electing a
candidate, the voting process has to continue.

The voting process is done in rounds and each round usually lasts for approximately $\lambda$ units of time.
$\lambda$ is selected by the ADAGs and could be several hours. All votes and messages \textbf{are tagged} in a way
that a vote cast in a round can not be used in another round. Votes are weighted based on users' stakes and
\texttt{online/offline} status of users is not considered. When we say 2/3 votes, we mean the sum of the stake of
voters is 2/3 of the total stake.

A user executes the following procedure in each agreement session:
\paragraph{Voting Phase in Round $r$:}
\begin{itemize}
    \item if the user has locked his vote on a proposal $p$, he votes $p$, otherwise he votes a single desired
    proposal.
    \item when $clock = \lambda$, if the user has seen more than $2/3$ votes for a proposal $p$, he votes $p$-win,
    otherwise he votes $draw$. A user votes either $p$-win or $draw$, not both.
    \item if the user sees more than 2/3 $draw$ votes, he goes to the round $r+1$ and sets $clock=0$.
    \item if the user sees more than 2/3 $p$-lock votes, he goes to the round $r+1$, sets $clock=0$ and locks his
    vote on $p$.
    \item when $clock = k \lambda$ for $k=2,3,\dots$, if the user has seen more than $2/3$ votes for a proposal
    $p$ he votes $p$-lock.
\end{itemize}

\paragraph{Termination:}
\begin{itemize}
    \item as soon as the user sees more than 2/3 $p$-win votes for $p$, he selects $p$ and ends
    the agreement protocol. The $p$-win votes can be for any round, but all must belong to the same round.
\end{itemize}


We assume that users have clocks with the same speed, and $\lambda \gg \delta$, where $\delta$ is the maximum
clock difference between users. We also assume that more than 2/3 of the total stake of the system is controlled
by honest users, and network partitions are resolved after a finite amount of time. With these assumptions it can be
shown that the emergency recovery protocol has the following important properties:
\begin{itemize}
    \item no two users will end the agreement protocol with two different proposals as the result of the agreement.
    \item if honest users can agree upon some proposal value, the agreement protocol will converge to that value
    after a finite number of rounds.
\end{itemize}

A honest user during a round only votes a single proposal. This ensures that as long as more than $2/3$ of the total
stake of the system is controlled by honest users, no two different proposals can get more than $2/3$ votes.
As a result, we can not have 2/3 votes for both $p$-lock and $p'$-lock if $p \neq p'$. Also, a honest user
either votes $p$-win or $draw$, so only one of $p$-win or $draw$ can get more than $2/3$ votes and when a
user sees more than $2/3$ $p$-win votes, he can be sure that $draw$ has less than $2/3$ votes. Therefore, for going
to the next round we will need 2/3 $p$-lock
votes and all honest users will lock their vote on $p$ when they start the next round. As a result only $p$ can be
confirmed by the agreement protocol.

Note that when a single honest user terminates the protocol, he can convince all other honest users to terminate
their protocol by sending those 2/3 $p$-win votes that he has seen.

If honest users can agree upon some proposal value, the agreement protocol will converge to that value. When a round
ends and we go to the next round, all honest users will lock their vote on
the same proposal or no one will lock his vote, so an agreement could be reached in next rounds. We will never get
stuck in a round. If at some
round $draw$ gets less than 2/3 votes, that means at least $\epsilon$ honest stake has voted
$p$-win.\footnote{It is possible that the $\epsilon$ stake has not voted yet. However based on the finite time
partitioning assumption, at some point that honest stake should get connected to the network and vote.}
That means there must be more than 2/3 votes for some proposal $p$ which convinced the $\epsilon$ honest stake to
vote $p$-win. Therefore, after waiting long enough, all the honest stake will see those votes and will eventually
vote for $p$-lock, and $p$-lock will get more than 2/3 votes.

\subsubsection{Initiating the Recovery Protocol}

When the validator software does not receive any blocks for \texttt{blockTimeOut} amount of time, or when it observes an
evidence which proves the delegates are malicious, after prompting the user and after his confirmation, it will
initiate the recovery protocol.

To do so, first the validator software activates the censorship resilient mode of the networking module, then it checks
the validity of the blocks that do not have a validators certificate and determines the last valid block of its
version of the blockchain.

In the next step, it will sign and broadcast an \textbf{emergency fork request} message, alongside some useful metadata
such as the last valid block of its blockchain and the evidence of delegates' misbehaviour.\footnote{this metadata is
not a part of the fork request.} Before starting the recovery protocol, validators try to synchronize their blockchains
as much as possible.

If the reserve committee of delegates is already active, or if the validator software sees a valid fork request signed
by more than half of the total online stake of the validators, but does not receive the fork
block after a certain amount of time, after user confirmation, it will sign and broadcast a request for
\textbf{emergency agreement} on a new reserve committee. The agreement on new delegates usually needs user
interaction and is not a fully automatic process.

The evidence which proves a committee of delegates is malicious is an invalid block that is signed by at least
one delegate:
\begin{itemize}
    \item a block that is not conditionally valid
    \item two different blocks with the same parent
\end{itemize}

\subsection{Estimating Stake Values}\label{subsec:user's-stake}

In a proof of stake system the influence of a user in the consensus protocol should be proportional to the amount
of stake the user has in the system. Conventionally in these systems, a user's stake is considered to be equal with the
amount of native system tokens, he has ``staked'' in the system. A user stakes his tokens by locking them in
his account or a separate staking account for some period of time. During this time, he will not be able to transfer
his tokens.

Unfortunately, there is a subtle problem with this approach. It is not clear in a real world economic system
how much of the main currency of the system can be locked and kept out of the circulation indefinitely. It seems that
this amount for currencies like US dollar, is quite low comparing to the total market cap of the currency.
This means that for a real world currency this type of staking mechanism will result in putting the
fate of the system in the hands of the owners of a small fraction of the total supply.

To mitigate this problem, Argennon uses a hybrid approach for estimating the stake of a user.
Every \texttt{stakingDuration} blocks, which is called a \emph{staking period}, Argennon calculates
a \emph{trust value} for each user.

The user's stake
at time step \(t\), is estimated based on the user's trust value and his ARG balance:
\begin{equation}
    S_{u,t} = \min (B_{u,t}, Trust_{u,k})\ ,\label{eq:stake}
\end{equation}
where:
\begin{itemize}
    \item \(S_{u,t}\) is the stake of user \(u\) at time step \(t\).
    \item \(B_{u,t}\) is the ARG balance of user \(u\) at time step \(t\).
    \item \(Trust_{u,k}\) is an estimated trust value for user \(u\) at staking period \(k\).
\end{itemize}

Argennon users can lock their ARG tokens in their account for any period of time. During this time a user
will not be able to transfer his tokens and there is no way for cancelling a lock.
The trust value of a user is calculated based on the amount of his locked tokens and the
Exponential Moving Average (EMA) of his ARG balance:
\begin{equation}
    Trust_{u,k} = L_{u,k} + M_{u,t_k}\ ,\label{eq:trust}
\end{equation}
where
\begin{itemize}
    \item $L_{u,k}$ is the amount of locked tokens of user $u$, whose release time is \textbf{after the end} of
    the staking period $k+1$.
    \item $M_{u,t_k}$ is the Exponential Moving Average (EMA) of the ARG balance of user \(u\) at time step \(t_k\).
    $t_k$ is the start time of the staking period $k$.
\end{itemize}

In Argennon a user who held ARGs and participated in the consensus for a long time is more trusted
than a user with a higher balance whose balance has increased recently. An attacker who has obtained a large
amount of ARGs, also needs to hold them for a long period of time before being able to attack the system.

For calculating the EMA of a user's balance at time step \(t\), we can use the following
recursive formula:
\[
    M_{u,t} = (1 - \alpha) M_{u,t-1} + \alpha B_{u,t} = M_{u,t-1} + \alpha (B_{u,t} - M_{u,t-1})\ ,
\]
where the coefficient \(\alpha\) is a constant smoothing factor between \(0\) and \(1\), which represents the
degree of weighting decrease. A higher \(\alpha\) discounts older observations faster.

Usually an account balance will not change in every time step, and we can use older values of EMA for calculating
\(M_{u,t}\): (In the following equations the \(u\) subscript is dropped for simplicity)
\[
    M_{t} = (1 - \alpha)^{t-k}M_{k} + [1 - (1 - \alpha)^{t - k}]B\ ,
\]
where:
\[
    B = B_{k+1} = B_{k+2} = \dots = B_{t}\ \cdot
\]
We know that when \(|nx| \ll 1\) we can use the binomial approximation \({(1 + x)^n \approx 1 + nx}\). So, we can
further simplify this formula:
\[
    M_{t} = M_{k} + (t - k) \alpha (B - M_{k})\ \cdot
\]

For choosing the value of \(\alpha\) we can consider the number of time steps that the trust value of a user needs
for reaching a specified fraction of his account balance. We know that for large \(n\) and \(|x| < 1\) we have
\((1 + x)^n \approx e^{nx}\), so by letting \(M_{u,k} = 0\) and \(n = t - k\) we can write:
\begin{equation}
    \alpha =- \frac{\ln\left(1 - \frac{M_{n+k}}{B}\right)}{n}\ .\label{eq:alpha}
\end{equation}
The value of \(\alpha\) for a desired configuration can be calculated by this equation. For instance, we could
calculate the \(\alpha\) for a relatively good configuration in which \(M_{n+k} = 0.8B\) and \(n\) equals to the
number of time steps of 10 years.

\subsection{Analysis}\label{subsec:consensus-math}
\note{not yet written...}
