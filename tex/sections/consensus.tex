%! Author = aybehrouz


\note{not yet written....}

\subsection{Estimating A User's Stake}\label{subsec:estimating-a-user's-stake}

In a proof of stake system the influence of a user in the consensus protocol should be proportional to the amount
of stake the user has in the system. Conventionally in these systems, for estimating a user's stake, we use the
amount of native system tokens the user is holding. Unfortunately, one problem with this approach is that a
strong attacker may be able to obtain a considerable amount of system tokens, for example by borrowing from a
DEFI application, and use this stake to attack the system.

To mitigate this problem, for calculating a user's stake at time step \(t\), instead of using the raw ARG
balance, we use the minimum of a \emph{trust value} the system has calculated for the user and the user's
ARG balance:
\[
    S_{u,t} = \min (B_{u,t}, Trust_{u,t})
\]
Where:
\begin{itemize}
    \item \(S_{u,t}\) is the stake of user \(u\) at time step \(t\).
    \item \(B_{u,t}\) is the ARG balance of user \(u\) at time step \(t\).
    \item \(Trust_{u,t}\) is an estimated trust value for user \(u\) at time step \(t\).
\end{itemize}

The agreement protocol, at time step \(t\), will use \(\sum_{u}S_{u,t}\) to determine the required
number of votes for the confirmation of a block, and we let \(Trust_{u,t} = M_{u,t}\), where \(M_{u,t}\) is the
exponential moving average of the ARG balance of user \(u\) at time step \(t\).

In our system a user who held ARGs and participated in the consensus for a long time is more trusted
than a user with a higher balance whose balance has increased recently. An attacker who has obtained a large
amount of ARGs, also needs to hold them for a long period of time before being able to attack the system.

For calculating the exponential moving average of a user's balance at time step \(t\), we can use the following
recursive formula:
\[
    M_{u,t} = (1 - \alpha) M_{u,t-1} + \alpha B_{u,t} = M_{u,t-1} + \alpha (B_{u,t} - M_{u,t-1})
\]
Where the coefficient \(\alpha\) is a constant smoothing factor between \(0\) and \(1\) which represents the
degree of weighting decrease, A higher \(\alpha\) discounts older observations faster.

Usually an account balance will not change in every time step, and we can use older values of EMA for calculating
\(M_{u,t}\): (In the following equations the \(u\) subscript is dropped for simplicity)
\[
    M_{t} = (1 - \alpha)^{t-k}M_{k} + [1 - (1 - \alpha)^{t - k}]B
\]
Where:
\[
    B = B_{k+1} = B_{k+2} = \dots = B_{t}
\]
We know that when \(|nx| \ll 1\) we can use the binomial approximation \({(1 + x)^n \approx 1 + nx}\). So, we can
further simplify this formula:
\[
    M_{t} = M_{k} + (t - k) \alpha (B - M_{k})
\]

For choosing the value of \(\alpha\) we can consider the number of time steps that the trust value of a user needs
for reaching a specified fraction of his account balance. We know that for large \(n\) and \(|x| < 1\) we have
\((1 + x)^n \approx e^{nx}\), so by letting \(M_{u,k} = 0\) and \(n = t - k\) we can write:
\[
    \alpha =- \frac{\ln\left(1 - \frac{M_{n+k}}{B}\right)}{n}
\]
The value of \(\alpha\) for a desired configuration can be calculated by this equation. For instance, we could
calculate the \(\alpha\) for a relatively good configuration in which \(M_{n+k} = 0.8B\) and \(n\) equals to the
number of time steps of 10 years.

In our system a newly created account will not have voting power for some time, no matter how high its
balance is. While this is a desirable property, in case a large proportion of total system tokens are
transferred to newly created accounts, it can result in too much voting power for older accounts. This may decrease
the degree of decentralization in our system.

However, this situation is easily detectable by comparing the total stake of the system with the total balance of
users. If after confirming a block the total stake of the system goes too low and we have:
\[
    \sum_{u}S_{u,t} < \gamma \sum_{u}B_{u,t}
\]
The protocol will perform a \emph{time shift} in the system: the time step of the system
will be incremented for \(m\) steps while no blocks will be confirmed. This will increase the value of \(M_{u,t}\)
for new accounts with a non-zero balance, giving them more influence in the agreement protocol.

For calculating the value of \(m\) which determines the amount of time shift in the system, we should note that when
\(B_{u,t} = B_{u, t-1} = B_u\), we can derive a simple recursive rule for the stake of a user:
\[
    S_{u,t} = (1 - \alpha) S_{u,t-1} + \alpha B_u
\]
Therefore, we have:
\[
    \sum_{u}S_{u,t} = (1 - \alpha) \sum_{u}S_{u,t - 1} + \alpha \sum_{u}B_u
\]
This equation shows that when the balance of users is not changing over time the total stake of the system is the
exponential average of the total ARGs of the system. Consequently, when we shift the time for \(m\) steps, we can
calculate the new total stake of the system from the following equation:

\[
    \sum_{u}S_{u,t+m} = (1 - \alpha)^{m}\sum_{u}S_{u,t} + [1 - (1 - \alpha)^{m}]\sum_{u}B_u
\]
Hence, if we want to increase the total stake of the system from \(\gamma \sum_{u}B_u\) to \(\lambda \sum_{u}B_u\),
we can obtain \(m\) from the following formula, assuming \(\alpha\) is small enough:
\[
    m = \frac{1}{\alpha} \ln \left(\frac{1 - \gamma}{1 - \lambda}\right)
\]