%! suppress = TooLargeSection
%! Author = aybehrouz
%! Date = 2/28/21


\documentclass[11pt, a4paper]{report}
\usepackage[a4paper]{geometry}

% Packages
\usepackage{tipa}
\usepackage{graphicx}
\usepackage{hyperref}
\usepackage{amsfonts}
\usepackage[ruled,boxed]{algorithm2e}
\usepackage{algpseudocode}
\usepackage{listings}
\usepackage{amssymb}
\usepackage{tcolorbox}
\usepackage{amsmath}


%\binoppenalty=99999
%\relpenalty=99999
\newcommand{\note}[1] {
    \begin{tcolorbox}[colframe=white,colback=white]
        \emph{#1}
        %\emph{#1}\medskip
    \end{tcolorbox}
}

\lstset{
    language=Java,
    morekeywords={main, initialize, constructor, external, virtual, Map, Account, signature, overrides, is, string},
    basicstyle=\small\sffamily,
    columns=fullflexible,
%   keywordstyle=\bfseries,
    breaklines=true,
    showstringspaces=false,
    mathescape
}

\title{Argennon: A Scalable Cloud Based Smart Contract Platform}
\author{aybehrouz}
\date{January 2021}


\begin{document}
    \maketitle
    \begin{abstract}
        Argennon is a next generation cloud based blockchain and smart
        contract platform. The Argennon blockchain uses
        a hybrid proof of stake (HPoS) consensus protocol, which is capable of combining the benefits of
        a centralized and a decentralized system. In Argennon, ledger storage and transaction validation are
        outsourced to the cloud and normal personal computers or smartphones, with limited hardware
        capabilities, can validate transactions and actively
        participate in the Argennon consensus protocol. This property makes Argennon a truly decentralized and
        democratic blockchain and one of the most secure existing platforms.

        In Argennon, Computational Integrity (CI) is achieved by using succinct cryptographic proofs (STARK/SNARK)
        and data integrity is guaranteed by cryptographic accumulators. There is no need to trust cloud servers.
        At the same time, a smart clustering algorithm keeps
        the network usage and the overhead of commitment schemes manageable.

        The Argennon protocol strongly incentivizes the formation of a permission-less network of Publicly Verifiable
        Cloud (PVC) servers. A PVC server in Argennon, is a conventional data server which uses its computational and
        storage resources to help the Argennon network process transactions. This encourages the development
        of conventional networking, storage and compute hardware, which can benefit all areas of information technology.
        This contrasts with the approach of some older blockchains that incentivizes the development of a totally
        useless technology of hash calculation.

    \end{abstract}
    \tableofcontents


    \chapter{The Argennon Smart Contract Execution Environment}\label{ch:AVM}
    %! Author = aybehrouz


\section{Introduction}\label{sec:introduction}

The Argennon Smart Contract Execution Environment (AscEE) is an abstract high level execution environment for executing
Argennon smarts contracts (a.k.a\ Argennon applications) in a secure and isolated environment. An Argennon application
essentially is an HTTP server whose state is
kept in the Argennon blockchain and its logic is described using an Argennon Standard Application Representation (ASAR).

An Argennon Standard Application Representation is a programming language for describing Argennon
applications, optimized
for the architecture and properties of the Argennon platform. This text based representation should be low level
enough to enable easy compilation from any high level programming language and should be high level enough to preserve
the high level information of the application logic and facilitate platform specific compiler optimization at a host
machine. In this regard, the ASAR can be considered as an Intermediate Representation (IR).

The state of an Argennon application is stored in byte addressable finite arrays of memory called
\emph{heap chunks}. An application may have several heap chunks with different sizes, and can remove or
resize its heap chunks or allocate new chunks. Every chunk belongs to exactly one application and can only be modified
by its owner. In addition to heap chunks, every application has a limited amount of non-persistent local memory for
storing temporary data.

The AscEE executes the requests contained in each block of the Argennon
blockchain in a three-step procedure. The first step is the \emph{preprocessing step}. In this step, the required
data for executing requests are retrieved and verified and the
helper data structures for next steps are constructed. This step is
designed in a way that can be done fully in parallel for each request without any risk of data races. The second
step is the \emph{Data Dependency Analysis (DDA) step}.
In this step by analyzing data dependency between requests, the AscEE determines requests that can be run in parallel
and requests that need to be run sequentially. This information is represented using an \emph{execution DAG} data
structure and in the final step, the requests are executed using this data structure.


\section{Execution Sessions}\label{sec:sessions}

The Argennon Smart Contract Execution Environment can be seen as a machine for executing Argennon applications to
fulfill \emph{external} HTTP requests\footnote{External requests are requests that are not made by other Argennon
applications.}, produce their HTTP responses and update related heap chunks. The execution of requests can be
considered sequential\footnote{Actually requests may be executed in parallel but by performing data dependency analysis
the result is guaranteed to be identical with the sequential execution of requests.} and each request has a separate
\emph{execution session}. Therefore, an execution session is a separate session of executing application's code in
order to fulfill an external HTTP request. We call this external request the \emph{initiator} of the execution session.


The state of an execution session will be
destroyed at the end of the session and only the state of heap chunks is preserved. If a session fails and does not
complete normally, it will not have any effects on any heap chunks.

During an execution session an application can make \emph{internal} HTTP requests to other applications. Those
requests will not start a new execution session and will be executed within the current session. In AscEE making an
internal HTTP request to an application is similar to a function invocation, and for that reason, we also refer to
them as \emph{application calls}.


\section{Memory}\label{mem}

Every Argennon application has two types of memory: local memory and heap. Local memory is not persistent and is
destroyed when the application call ends. Local memory is used for storing local variables and is not directly
addressable. Heap, on the other hand, is persistent and can be used for persisting data between application calls.
Heap is addressable and provides low level direct access. Both local memory and heap are limited, but the
AscEE does not specify any particular limits for them. When an application tries to use too much memory, that may
cause the execution session to end abruptly. In that case, the execution session will not have any effects on the
state of the heap.

\subsection{Heap Chunks}\label{subsec:heap}

The AscEE heap is split into chunks. Each heap chunk is a continuous finite array of bytes, has a unique identifier, and
is byte addressable. An application may have several heap chunks with different sizes and can remove or
resize its chunks or allocate new ones. Every chunk belongs to exactly one application. Only the owner application can
modify a chunk but there is no restrictions for reading a chunk\footnote{The reason behind this type of access
control design is the fact that smart contract
code is usually immutable. That means if a smart contract does not implement a
getter mechanism for some parts of its internal data, this functionality can never
be added later, and despite the internal data is publicly available, there will be no
way for other smart contracts to use this data on-chain.}.

When an application allocates a new heap chunk, the identifier of the new chunk is not generated by
the AscEE. Instead, the application can choose an identifier itself, provided the new id has a correct format. This
is an important feature of the AscEE heap which allows applications to use the AscEE heap as a map
data structure\footnote{also called a dictionary.}.
Since the \texttt{chunkID} is a prefix code, any application has its own identifier space, and an application
can easily find unique identifiers for its chunks.

During an execution session every heap chunk has an access type which may disallow certain operations. This access
type is declared by the initiator request of the execution session:

\begin{itemize}
    \item \texttt{check\_only}: only allows check operations. These operations query the persistence
    status of a memory location.
    \item \texttt{read\_only}: only allows read and check operations.
    \item \texttt{writable}: allows reading and writing.
    \item \texttt{additive}: only allows additive operations. By additive we mean an addition-like operator without
    overflow checking which is associative and commutative. Note that the content of these chunks cannot be read.
\end{itemize}

\subsubsection{Chunk Resizing}\label{subsubsec:ch-resize}

At the start of executing requests of a block, a validator can consider two values for every
heap chunk, its size: \texttt{chunkSize} and a size upper bound: \texttt{sizeUpperBound}. The value of
\texttt{chunkSize} can be determined uniquely at the start of
every execution session, and it may be updated during the session by the owner application. On the other hand,
the value of \texttt{sizeUpperBound} is proposed by the block proposer for each block and is calculated based on values
declared by external requests (i.e.\ transactions) which want to perform chunk resizing. This value needs to be
an upper bound of all the declared resizing values and indicates the upper bound of \texttt{chunkSize} during the
execution of a block.

The value of \texttt{chunkSize}, can be modified during an execution session. However, the new values of size can
only be increasing or decreasing. More precisely, if a request declares that it wants to expand (shrink) a chunk, it
can only increase (decrease) the value of \texttt{chunkSize} and any specified value during the execution
session, needs to be greater (smaller) than the previous value of the chunk's size. Any request that wants to expand
(shrink) a chunk needs to specify a max size (min size). The value of \texttt{chunkSize} can not be set higher
(lower) than this value.

The value of \texttt{chunkSize} at the end of an execution session will determine if a memory location at an
offset is persistent or not: Offsets lower than the chunk size are persistent, and higher offsets are not.
Non-persistent locations will be re-initialized with zero at the start of every execution session.

Usually an application should not have any assumption about the content of memory locations that are outside the chunk.
While these locations are zero initialized at the start of every execution session, multiple
invocations of an application may occur in a single execution session, and if one of them modifies a location outside
the chunk, the changes can be seen by next invocations.

There is no way for an application to query \texttt{sizeUpperBound}. As a result, for an application
accessing offsets higher than \texttt{chunkSize} results in undefined behaviour, while the behaviour is well-defined
in the view of validators.
This enables validators to determine the validity of an offset at the start of the block validation in a parallelized
preprocessing phase without actually executing requests.

While an application can use \texttt{chunkSize} to determine if an offset is persistent or not, that is not
considered a good practice. Reading \texttt{chunkSize} decreases transaction parallelization, and should be avoided.
Instead, applications should use a built-in AscEE function for checking the persistence status of memory addresses.

An application can load any chunk with a valid prefix identifier even if that chunk does not exist. For a non-existent
chunk the value of \texttt{chunkSize} is always zero.

The address space of a chunk starts from zero and only offsets lower than \texttt{sizeUpperBound} are valid. Trying to
access any offset higher than this value will always result in a revert for the application.

\section{Identifiers}\label{sec:identifiers}

In Argennon a unique identifier is assigned to every application, heap chunk and account. Consequently, three distinct
identifier types exist: \texttt{appID}, \texttt{accountID}, and \texttt{chunkID}.
All these identifiers are \emph{prefix codes}, and hence can be represented by
\emph{prefix trees}\footnote{Also called tries.}.

Argennon has four primitive prefix trees:
\emph{applications, accounts, local} and \emph{varUint}.
All these trees are in base 256, with the maximum height
of 8.

An Argennon identifier may be simple or compound. A simple identifier is generated using a single tree, while a
compound identifier is generated by concatenating prefix codes generated by two or more trees:

\begin{itemize}
    \item \texttt{appID} is a prefix code built by \emph{applications} prefix tree. An \texttt{appID} cannot
    be \texttt{0x0}.

    \item \texttt{accountID} is a prefix code built by \emph{accounts} prefix tree. An \texttt{accountID} cannot
    be \texttt{0x0} or \texttt{0x1}.

    \item \texttt{chunkID} is a composite prefix code built by concatenating an \texttt{applicationID} to
    an \texttt{accountID} to a prefix code made by \emph{local} prefix tree:
    \subitem \texttt{chunkID = (applicationID|accountID|<local-prefix-code>)} .
\end{itemize}

All Argennon prefix trees have an equal branching factor \(\beta\)\footnote{A typical choice for \(\beta\) is \(2^8\)
    .}, and
we can represent an Argennon
prefix tree as a sequence of fractional numbers in base \(\beta\):
\[
    (A^{(1)},A^{(2)},A^{(3)},\dots)\ ,
\]
where \(A^{(i)}=(0.a_{1}a_{2}\dots a_{i})_\beta\), and we have \(A^{(i)} \leq A^{(i+1)}\). \footnote{It's possible to
have \(a_i=0\). For example, \(A^{(4)}=(0.2000)_{10}\) is correct.}

One important property of prefix identifiers is that while they have variable and unlimited length, they are
uniquely extractable from any sequence. Assume that we have a string of digits in base $\beta$, we
know that the sequence starts with an Argennon identifier, but we do not know the length of that identifier.
Algorithm~\ref{alg:prefix_id} can be used to extract the prefixed identifier uniquely. Also, we can apply this algorithm
multiple times to extract a composite identifier, for example \texttt{chunkID}, from a sequence.

%##\includegraphics[width=17cm]{../img/Alg1s.png}
\begin{algorithm}[t]
    \DontPrintSemicolon
    \SetKwInOut{Input}{input}\SetKwInOut{Output}{output}
    \Input{A sequence of $n$ digits in base $\beta$: $d_{1}d_{2}\dots d_{n}$ \newline
    A prefix tree: $<A^{(1)},A^{(2)},A^{(3)},\dots>$}
    \BlankLine
    \Output{Valid identifier prefix of the sequence.}
    \BlankLine
    \For{$i = 1$ \KwTo $n$}
    {
        \If{$(0.d_{1}d_{2}\dots d_{i})_\beta < A^{(i)}$}
        {
            \KwRet{$d_{1}d_{2}\dots d_{i}$}\;
        }
    }
    \KwRet{NIL}\;
    \caption{Finding a prefixed identifier}\label{alg:prefix_id}
\end{algorithm}

When we have a prefixed identifier, and we want to know if a sequence of digits is marked by that identifier,
we use Algorithm~\ref{alg:match_id} to match the prefixed identifier with the start of the sequence. The matching
can be done with only three comparisons, and an invalid prefixed identifier can be detected and will not match
any sequence.

In Argennon the shorter prefix codes are assigned to more active accounts and applications which tend to own more
data objects in the system. The prefix trees are designed by analyzing empirical data to make sure the number
of leaves in each level is chosen appropriately.

\begin{algorithm}[h]
    \DontPrintSemicolon
    \SetKwData{Id}{$id$}
    \SetKwInOut{Input}{input}\SetKwInOut{Output}{output}
    \Input{A prefixed identifier in base $\beta$ with $n$ digits: $\Id=a_{1}a_{2}\dots a_{n}$ \newline
    A sequence of digits in base $\beta$: $d_{1}d_{2}d_{3}\dots $ \newline
    A prefix tree: $<0,A^{(1)},A^{(2)},A^{(3)},\dots>$
    }
    \BlankLine
    \Output{$TURE$ if and only if the identifier is valid and the sequence starts with the identifier.}
    \BlankLine
    \If{$(0.a_{1}\dots a_{n})_\beta = (0.d_{1}\dots d_{n})_\beta$}
    {
        \If{$A^{(n-1)} \leq (0.a_{1}a_{2}\dots a_{n})_\beta < A^{(n)}$}
        {
            \KwRet{TRUE}\;
        }
    }
    \KwRet{FALSE}\;
    \caption{Matching a prefixed identifier}\label{alg:match_id}
\end{algorithm}

\section{Request Attachments}\label{sec:attachments}

The attachment of a request is a list of request identifiers of the current block that are ``attached'' to the request.
That means, for validating the request a validator first needs to ``inject'' the digest of attached requests into the
HTTP request text. By doing so, the called application will have access to the digest of attachments in a secure way.

The main usage of this feature is for fee payment. A request that wants to pay the fees for a number of requests,
declares those requests as its attachments. For paying fees the payer signs the digest of requests for which he
wants to pay fees. After injecting the digest of those request by validators, that signature can be validated
correctly and securely by the application that handles fee payment.


\section{Authorization}\label{sec:auth}

In blockchain applications, we usually need to authorize certain operations. For example, for sending an asset
from a user to another user, we need to make sure that the sender has authorized that operation.

The AscEE uses \emph{Authenticated Message Passing} for authorizing operations. In this method, every execution
session has a set of authenticated messages, and those messages are \textbf{explicitly} passed in requests to
applications for authorizing operations. These messages act exactly like digital signatures and applications can
ensure that they are issued by their claimed issuer accounts. The only difference is that the process of
message authentication is performed by the AscEE internally and an application does not explicitly verify cryptographic
signatures.

Each execution session has a list of authenticated messages. Each authenticated message has an index which will be
used for passing the message to an application as a request parameter. The AscEE uses cryptographic signatures to
authenticate messages for user accounts. The signatures are validated in parallel during the
preprocessing step, and any type of cryptographic signature scheme can be used.

Also, applications can use built-in functions of the AscEE to generate authenticated messages in run-time.
This enables an application to authorize operations for another application even if it is not calling that
application directly.

In addition to authenticated messages, the AscEE provides a set of
cryptographic functions for validating signatures and calculating cryptographic entities. By using these functions and
passing cryptographic signatures as parameters to methods, a programmer, having users' public keys, can implement
the required logic for authorizing operations.

Authorizing operations by Authenticated Message Passing and explicit signatures eliminates the need for approval
mechanisms or call back patterns in Argennon.\footnote{The AscEE has no instructions for issuing cryptographic
signatures.}


\section{Reentrancy Protection}\label{sec:reentrancy}

The AscEE provides optional low level reentrancy protection by providing low
level \emph{entrance locks}. When an application acquires an entrance lock it cannot acquire that lock again and trying
to do so will result in a revert. The entrance lock of an application will be released when the application explicitly
releases its lock or when the call that has acquired that lock completes.

The AscEE reentrancy protection mechanism is optional. An application can allow reentrancy, it can protect only certain
areas of its code, or can completely disallow reentrancy.


\section{Deferred Calls}\label{sec:deferred-calls}

\ldots



\section{The ArgC Language}\label{sec:the-argc-language}

\subsection{The ArgC Standard Library}\label{sec:asl}

In Argennon, some applications (smart contracts) are updatable. The ArgC Standard Library is an updatable smart
contract which can be updated by the Argennon governance
system. This means that bugs or security vulnerabilities in the ArgC Standard Library could be quickly patched and
applications could benefit from bugfixes and improvements of the ArgC Standard Library even if they are
non-updatable. Many important and useful functionalities,
such as fungible and non-fungible assets, access control mechanisms,
and general purpose DAOs are implemented in the ArgC Standard Library.

All Argennon standards, for instance ARC standard series, which defines standards regarding transferable assets,
are defined based on how a contract should use the AVM standard library. As a result, Argennon standards are
different from conventional blockchain standards. Argennon standards define some type of standard logic and
behaviour for a smart contract, not only a set of method signatures. This enables users to expect certain type
of behaviour from a contract which complies with an Argennon standard.


\section{Data Dependency Analysis}\label{sec:concurrency}
%! Author = aybehrouz


\subsection{Memory Dependency Graph}\label{subsec:memory-dependency-graph}

Every block of the Argennon blockchain contains a list of transactions. This list is an ordered list and the
effect of its contained transactions must be applied to the AscEE state sequentially as they appear in the ordered
list. This ordering is solely chosen by the block proposer, and users should not have any assumptions about
the ordering of transactions in a block.

The fact that block transactions constitute a sequential list, does not mean they can not be executed and applied
to the AscEE state concurrently. Many transactions are actually independent and the order of their execution does not
matter. These transactions can be safely validated in parallel by validators.

A transaction can change the AscEE state by modifying either the code area or the AscEE heap. In Argennon, all
transactions declare the list of memory locations they want to read or write. This will enable us to determine the
independent sets of transactions which can be executed in parallel. To do so, we define the \emph{memory dependency
graph} \(G_d\) as follows:

\begin{itemize}
    \item \(G_d\) is an undirected graph.
    \item Every vertex in \(G_d\) corresponds to a transaction and vice versa.
    \item Vertices \(u\) and \(v\) are adjacent in \(G_d\) if and only if \(u\) has a memory location \(L\) in its
    writing list and \(v\) has \(L\) in either its writing list or its reading list.
\end{itemize}

If we consider a proper vertex coloring of \(G_d\), every color class will give us an independent set of
transactions which can be executed concurrently. To achieve the highest parallelization, we need to color \(G_d\)
with minimum number of colors. Thus, the \emph{chromatic number} of the memory dependency graph shows how good a
transaction set could be run concurrently.

Graph coloring is computationally NP-hard. However, in our use case we don't need to necessarily find an optimal
solution. An approximate greedy algorithm will perform well enough in most circumstances.

After constructing the memory dependency graph, we can use it to construct the
\emph{execution DAG} of transactions. The execution DAG of transaction set \(T\) is a directed acyclic
graph \(G_e = (V_e,E_e)\) which has the \emph{execution invariance} property:
\begin{itemize}
    \item Every vertex in \(V_e\) corresponds to a transaction in \(T\) and vice versa.
    \item Executing the transactions of \(T\) in any order that \emph{respects} \(G_e\) will result in
    the same AscEE state.
    \begin{itemize}
        \item An ordering of transactions of \(T\) respects \(G_e\) if for every directed edge \((u,v) \in E_e\)
        the transaction \(u\) comes before the transaction \(v\) in the ordering.
    \end{itemize}
\end{itemize}

Having the execution DAG of a set of transactions, using Algorithm~\ref{alg:exec_dag}, we can apply the transaction
set to the AscEE state concurrently, using multiple processor, while we can be sure that the resulted AscEE state will
always be the same no matter how many processor we have used.

%##\includegraphics[width=17cm]{../img/Alg1s.png}
\begin{algorithm}
    \DontPrintSemicolon
    \SetKwData{Ready}{$R_e$}\SetKwData{V}{$v_f$}\SetKwData{Graph}{$G_e$}\SetKwData{Vertices}{$V$}\SetKwData
    {Txns}{$T$}
    \KwData{The execution dag $\Graph = (\Vertices,E)$ of transaction set \Txns}
    \KwResult{The state after applying \Txns with any ordering respecting \Graph}
    \BlankLine
    \Ready $\gets$ the set of all vertices of \Vertices with in degree 0\;
    \While{$\Vertices \neq \varnothing$}
    {
        wait until a new free processor is available\;
        \If{the execution of a transaction was finished}
        {
            remove the vertex of the finished transaction \V from \Graph\;
            \For{each vertex $u \in Adj[\V]$}
            {
                \If{$u$ has zero in degree}
                {
                    $\Ready \gets \Ready \cup u$\;
                }
            }
        }
        \If{$\Ready \neq \varnothing$}
        {
            remove a vertex from \Ready and assign it to a processor\;
        }
    }
    \caption{Executing DAG transactions}\label{alg:exec_dag}
\end{algorithm}

By replacing every undirected edge of a memory dependency graph with a directed edge in such a way that the
resulted graph has no cycles, we will obtain a valid execution DAG. Thus, from a memory dependency graph different
execution DAGs can be constructed with different levels of parallelization ability.

If we assume that we have unlimited number of processors and all transactions take equal time for executing, it
can be shown that by providing a minimal graph coloring to Algorithm~\ref{alg:gen_dag} as input, the resulted
DAG will be optimal, in the sense that it results in the minimum overall execution time.

%##\includegraphics[width=17cm]{../img/Alg2s.png}
\begin{algorithm}
    \DontPrintSemicolon
    \SetKwData{Txns}{$T$}\SetKwData{Gd}{$G_d=(V_d,E_d)$}
    \SetKwInOut{Input}{input}\SetKwInOut{Output}{output}
    \Input{The memory dependency graph \Gd of transaction set \Txns\\A proper coloring of $G_d$}
    \Output{An execution dag $G_e=(V_e,E_e)$ for the transaction set \Txns}
    \BlankLine
    $V_e \gets V_d$\;
    $E_e \gets \varnothing$\;
    define a total order on colors of $G_d$\;
    \For{each edge $\{u,v\} \in E_d$}
    {
        \eIf{$color[u] < color[v]$}
        {
            $E_e \gets E_e \cup (u,v)$\;
        }{
            $E_e \gets E_e \cup (v,u)$\;
        }
    }
    \caption{Constructing an execution DAG}\label{alg:gen_dag}
\end{algorithm}

The block proposer is responsible for proposing an efficient execution DAG alongside his proposed block. This
execution DAG will determine the ordering of block transactions and help validators to validate transactions in
parallel. Since with better parallelization a block can contain more transactions, a proposer is incentivized enough
to find a good execution DAG for transactions.

\subsection{Memory Spooling}\label{subsec:spooling}

When two transactions are dependant and they are connected with an edge \((u,v)\) in the execution DAG,
the transaction \(u\) needs to be run before the transaction \(v\). However, if \(v\) does not read any
memory locations that \(u\) modifies, we can run \(u\) and \(v\) in parallel. We just need to make sure
\(u\) does not see any changes \(v\) is making in AscEE memory. This can be done by appropriate versioning
of the memory locations which is shared between \(u\) and \(v\). We call this method \emph{memory spooling}.
After enabling memory spooling between two transactions the edge connecting them can be safely removed from the
execution DAG\@.

\subsection{Concurrent Counters}\label{subsec:concurrent-counters}

We know that in Argennon every transaction needs to transfer its proposed fee to the \texttt{feeSink} accounts
first. This essentially makes every transaction a reader and a writer of the memory locations which store the
balance record of the \texttt{feeSink} accounts. As a result, all transactions in Argennon will be dependant and
parallelism will be completely impossible. Actually, any account that is highly active, for example the account
of an exchange or a payment processor, could become a concurrency bottleneck in our system which makes all
transactions interacting with them dependant.

This problem can be easily solved by using a concurrent counter for storing the balance record of this type of
accounts. A concurrent counter is a data structure which improves concurrency by using multiple memory locations for
storing a single counter. The value of the concurrent counter is equal to the sum of its sub counters and it can
be incremented or decremented by incrementing/decrementing any of the sub counters. This way, a concurrent
counter trades concurrency with memory usage.

Algorithm~\ref{alg:CC} implements a concurrent counter which returns an error when the value of the counter
becomes negative.

%##\includegraphics[width=17cm]{../img/Alg3s.png}
\begin{algorithm}
    \DontPrintSemicolon
    \SetKwData{CC}{Counter}
    \SetKwFunction{Inc}{Increment}\SetKwFunction{Dec}{Decrement}\SetKwFunction{AtomInc}{AtomicIncrement}
    \SetKwFunction{AtomDec}{AtomicDecrement}\SetKwFunction{AtomSet}{AtomicSet}\SetKwFunction{Get}{GetValue}
    \SetKwFunction{Acquire}{Lock.Acquire}\SetKwFunction{Release}{Lock.Release}
    \SetKwProg{Fn}{Function}{}{}
    \Fn{\Get{\CC}}
    {
        $s \gets 0$\;
    \Acquire{}\;
    \For{$i \gets 0$ \KwTo $\CC.size - 1$}
    {
        $s \gets s + \CC.cell[i]$\;
    }
    \Release{}\;
    \KwRet{s}\;
    }
    \BlankLine
    \Fn{\Inc{\CC, value, seed}}
    {
        $i \gets seed \bmod \CC.size$\;
    \AtomInc{$\CC.cell[i]$, value}\;
    }
    \BlankLine
    \Fn{\Dec{\CC, value, seed, attempt}}
    {
        \If {attempt = \CC.size}
        {
            restore \CC by adding back the subtracted value\;
            \KwRet{Error}\;
        }
        $i \gets seed \bmod \CC.size$\;
        $i \gets (i + attempt) \bmod \CC.size$\;
    \eIf {$\CC.cell[i] \geq value$}
    {
        \AtomDec{$\CC.cell[i]$, value}\;
    }{
        $r \gets value - \CC.cell[i]$\;
        \AtomSet{$\CC.cell[i]$, $0$}\;
        \Dec{\CC, r, seed, $attempt + 1$}\;
    }
    }
    \caption{Concurrent counter}\label{alg:CC}
\end{algorithm}

It should be noted that in a blockchain application we don't have concurrent threads and therefore we don't need
atomic functions. For usage in a smart contract, the atomic functions of this pseudocode can be implemented like
normal functions.

Concurrent counter data structure is a part of the ArgC standard library, and any smart contract can use this data
structure for storing the balance record of highly active accounts.



    \chapter{Persistence Layer}\label{ch:persistance}
    %! Author = aybehrouz

The Argennon Virtual Machine has two persistent memory areas: \emph{method area}, and \emph{heap}. Method area stores
bytecodes of methods\footnote{also it stores constant area blocks.}, and heap stores memory chunks. Both of these
data elements, bytecodes and chunks, can be considered as continuous pieces of byte addressable memory. Throughout this
chapter, we shall call these data elements \emph{objects}.


\section{Storage Pages}\label{sec:storage-pages}

In the AVM persistence layer, similar objects are clustered together and constitute a bigger data element which we call
a \emph{page}.\footnote{we avoid calling them clusters, because usually a cluster refers to a \emph{set}. AVM object
clusters are not sets. They are ordered lists, like a page containing an ordered list of words or sentences.}
A page is an ordered list of an arbitrary number of objects, which their order reflects the order they were added to
the page:
\[
    P = (O_1,O_2,\dots,O_n),\quad i < j \; \Leftrightarrow \; \textrm{$O_i$ was added before $O_j$}\ .
\]

A page of the AVM storage should contain objects that have very similar access pattern. We expect that when a page
is needed for validating a block, almost all of its objects are needed for either reading or writing. We also prefer
that the objects are needed for the same access type. In other words, the objects of a page are chosen in a way that
for validating a block, we usually need to either read all of them or modify\footnote{and probably read.} all of them.

\section{Zero-knowledge Databases}\label{sec:zk-edb}

Pages of the AVM storage are persisted using updatable zero-knowledge elementary databases (ZK-EDB). Argennon
has three zero-knowledge databases: \emph{staking} database, which stores all the data that is associated with
the Argennon consensus protocol. \emph{method} database, which stores the AVM method area, and \emph{heap} database,
which stores the AVM heap. The commitment of these three ZK-EDBs are included in every block of the Argennon blockchain.

We consider the following properties for a ZK-EDB:
\begin{itemize}
    \item The ZK-EDB contains a mapping from a set of keys to a set of values.
    \item Every state of the database has a commitment \(C\).
    \item The ZK-EDB has a method \((D, \pi) = \text{get}(x)\), where \(x\) is a key and \(D\) is the associated data
    with \(x\), and \(\pi\) is a proof.
    \item A user having \(C\) and \(\pi\) can verify that \(D\) is really associated with \(x\), and \(D\) is not
    altered. Consequently, a user who can obtain \(C\) from a trusted source does not need to trust the ZK-EDB\@.
    \item Having \(\pi\) and \(C\) a user can compute the commitment \(C'\) for the database in which \(D'\) is
    associated with \(x\) instead of \(D\).
\end{itemize}

Pages of the AVM storage are stored in the ZK-EDBs, with an index: \texttt{pageIndex} as their key.
The \texttt{pageIndex} is required to be smaller than a certain value, determined by the
protocol, to facilitate the usage of ZK-EDBs that are based on vector commitments.
For this reason, the AVM clustering algorithm always tries to reuse indices and keep the number of used indices
as low as possible.

The commitments of the AVM ZK-EDBs are affected by the way data objects are clustered. Therefore,
the Argennon clustering algorithm has to be a part of the consensus protocol.

Every block of the Argennon blockchain contains a set of \emph{clustering directives}. These directives
can only modify pages that were used for validating the block, and can
include directives for moving an object from one page to another or directives specifying which pages will contain
the newly created objects. These directives are always executed by nodes at the end of block validation.

A block proposer could obtain clustering directives from any third party source\footnote{we can say the AVM clustering
algorithm is essentially off-chain.}. This will not
affect Argennon security, since the integrity of a database can not be altered by clustering directives.
Those directives can only affect the performance of the Argennon network, and directives of a single block can
not affect the performance considerably.

\subsection{Vector Commitments}\label{subsec:impl-zk-edbs}

Informally, vector commitments allow committing to an ordered sequence of $q$ values
(i.e.\ a vector), rather than to single messages. This is done in a way such that it is later possible
to open the commitment with respect to specific positions (e.g., to prove that $m_i$ is the $i$-th committed
message). More precisely, vector commitments are required to satisfy what is called position binding.
Position binding states that an adversary should not be able to open a commitment to two different
values at the same position. While this property, by itself, would be trivial to realize using standard
commitment schemes, what makes vector commitments interesting is that they are concise, i.e.,
the size of the commitment string as well as the size of each opening \textbf{is independent of the
vector length}.

\note{not yet written...}

\section{Object Clustering Algorithm}\label{sec:clustering}

\note{not yet written...}





    \chapter{Networking Layer}\label{ch:networking}


    \section{Normal Mode}\label{sec:normal-mode}
    Unlike conventional blockchains, Argennon does not use a P2P network architecture. Instead, it uses a
    client-server topology, based on a permission-less list of ZK-EDB servers. ZK-EDB servers are a
    crucial part of the Argennon ecosystem, and they form the backbone of the Argennon networking layer.
    \note{not yet written...}


    \section{Censorship Resilient Mode}\label{sec:cens-res-mode}
    \note{not yet written...}


    \chapter{The Argennon Blockchain}\label{ch:argennon-blockchain}


    \section{Blocks}\label{sec:blocks}
    %! Author = aybehrouz

The Argennon blockchain is a sequence of blocks. Every block represents an ordered list of transactions, intended to be
executed by the Argennon Virtual Machine. The first block of the blockchain, the \emph{genesis} block, is a spacial
block that fully describes the initial state of the AVM. Every block of the Argennon blockchain thus corresponds to a
unique AVM state which can be calculated deterministically from the genesis block.

A block of the Argennon blockchain contains the following information:

\begin{center}
    \begin{tabular}{||c||}
        \hline
        \textbf{Block} \\ [0.5ex]
        \hline\hline
        commitment to the staking database            \\ [1.2ex]
        commitment to the method database             \\ [1.2ex]
        commitment to the heap database               \\ [1.2ex]
        commitment to the set of transactions         \\ [1.2ex]
        last block certificate issued by         \\
        the validators committee                      \\ [1.2ex]
        clustering directives                         \\ [1.2ex]
        random seed                                   \\ [1.2ex]
        previous block hash                           \\ [1.2ex]
        \hline
    \end{tabular}
\end{center}

\subsection{Block Validation}\label{subsec:block-validation}

Having the previous AVM state, the transaction list and the clustering directives of a block, a node can calculate
commitments to the staking, method and heap databases of the current block by emulating the AVM execution. If the
node can obtain the previous block information from a trusted source, it does not need to have a trusted local
copy of the AVM state,
and it can reliably retrieve the required storage pages from a ZK-EDB server. We call this type of block verification
\emph{conditional} block validation. This validation is conditional because the validity of the current block is
conditioned on the validity of the previous block.

Interestingly, conditional block validation of multiple blocks can be done in parallel. If a node has enough bandwidth
and computational resources, it can conditionally verify any number of blocks from a previously created blockchain
simultaneously and in parallel. As we will see in Section~\ref{subsec:validators-committee}, this property plays an
important role in the Argennon consensus protocol.

To some extent, conditional validation of a single block could be parallelized as well. Many transactions
in a block are actually independent and the order of their execution does not
matter. These transactions can be safely validated in parallel. Section~\ref{sec:concurrency} further
develops this concept.


\subsection{Block Certificate}\label{subsec:block-certificate}

An Argennon block certificate is an aggregate signature of some predefined subset of accounts. This predefined subset
is called the certificate committee and their signature ensures that the certified block is conditionally
valid given the validity of some previous block.

Argennon uses BLS aggregate signatures to represent block certificates. To better understand block certificates and
the Argennon consensus protocol, we need to briefly review the BLS signature scheme and its aggregation mechanism.

The BLS signature scheme operates in a prime order group and supports simple threshold signature generation,
threshold key generation, and signature aggregation. To review, the scheme uses the following ingredients:

\newcommand{\G}{\mathbb{G}}
\newcommand{\Z}{\mathbb{Z}}
\newcommand{\adv}{{\cal A}}
\newcommand{\bdv}{{\cal B}}
\newcommand{\deq}{\mathrel{\mathop:}=}
\newcommand{\SK}{\mathit{sk}}
\newcommand{\PK}{\mathit{pk}}
\newcommand{\C}{\mathit{cert}}
\newcommand{\APK}{\mathit{apk}}
\newcommand{\DPK}{\mathit{\Delta pk}}
\newcommand{\MM}{\mathcal{M}}
\newcommand{\xwedge}{\, \operatorname{\text{$\wedge$}}\, }
\newcommand{\abs}[1]{\lvert #1 \rvert}
\newcommand{\Hm}{H_0}
\newcommand{\Hpk}{H_1}
\newcommand{\qHpk}{Q_{\Hpk}}
\newcommand{\qHm}{Q_{\Hm}}
\newcommand{\qsig}{Q_{\text{sig}}}

\begin{itemize}
    \item An efficiently computable \emph{non-degenerate} pairing $e:\G_0 \times \G_1 \to \G_T$
    in groups $\G_0$, $\G_1$ and $\G_T$ of prime order $q$. We let $g_0$ and $g_1$ be generators
    of $\G_0$ and $\G_1$ respectively.
    \item A hash function $H_0: \mathcal{M} \rightarrow \mathbb{G}_0$, where $\mathcal{M}$ is the message space.
    The hash function will be treated as a random oracle.
\end{itemize}

The BLS signature scheme is defined as follows:

\begin{itemize}
    \item $\textbf{KeyGen}()$: choose a random $\alpha$ from $\Z_q$ and set $h \gets g_1^\alpha \in \G_1$.
    output $\PK \deq (h)$ and $\SK \deq (\alpha)$.
    \item $\textbf{Sign}(\SK, m)$: output $\sigma \gets \Hm(m)^\alpha \in \G_0$.
    The signature $\sigma$ is a \emph{single} group element.
    \item $\textbf{Verify}(\PK,m,\sigma)$: if $e(g_1, \sigma) = e\big(\PK,\ \Hm(m)\big)$  then output "accept",
    otherwise output "reject".
\end{itemize}

Given triples $(\PK_i,\ m_i,\ \sigma_i)$ for $i=1,\ldots,n$,
anyone can aggregate the signatures $\sigma_1,\ldots,\sigma_n \in \G_0$
into a short convincing aggregate signature $\sigma$ by computing
\begin{equation}
    \label{eq:agg}
    \sigma \gets \sigma_1 \cdots \sigma_n \in \G_0\ .
\end{equation}
Verifying an aggregate signature $\sigma \in \G_0$ is done by checking that
\begin{equation}
    \label{eq:aggdiff}
    e(g_1, \sigma) = e\big(\PK_1,\ \Hm(m_1)\big) \cdots e\big(\PK_n,\ \Hm(m_n)\big)\ .
\end{equation}
When all the messages are the same ($m = m_1 = \ldots = m_n$), the verification relation~\eqref{eq:aggdiff} reduces to
a simpler test that requires only two pairings:
\begin{equation}
    \label{eq:aggsame}
    e(g_1, \sigma) = e\Big(\PK_1 \cdots \PK_n,\ \Hm(m)\Big)\ .
\end{equation}
We call $\APK=\PK_1 \cdots \PK_n$ the aggregate public key.

To defend against \emph{rogue public key} attacks, Argennon uses Prove Knowledge of the Secret Key (KOSK) scheme. As we
explained in Section~\ref{sec:accounts}, when an account is created its public keys need to be registered in
the ARG smart contract. Therefore, the KOSK scheme can be easily implemented in Argennon.

Because it is not usually possible to collect the signatures of all members of a certificate committee, an Argennon
block certificate essentially is an Accountable-Subgroup Multi-signature (ASM). Argennon uses a simple ASM scheme
based on BLS aggregate signatures.

Argennon block certificates constitute an ordered sequence based on the order of blocks they certify. If we show
the certificate of committee $C$ for the $i$-th block\footnote{note that the $i$-th block certificate is not
necessarily the certificate of the $i$-th block.} with $\C_i$, and the set of signers
with $S_i$, then the block certificate $\C_i$ can be considered as a tuple:
\begin{equation}
    \C_i=(\sigma_i,\ C-S_{i})\label{eq:cert}\ ,
\end{equation}
where $\sigma_i$ is the aggregate signature issued by $S_i$.

The aggregate public key of the certificate can
be calculated from:
\begin{equation}
    \APK_i=\APK_C\APK_{C-S_i}^{-1}\label{eq:aggCertPK}\ ,
\end{equation}
where $\APK_{A}$ shows the aggregate public key of all accounts in $A$.

Alternately we can use $\APK_{i-1}$ to calculate the aggregate public key:
\begin{equation}
    \APK_i=\APK_{i-1}\APK_{S_i-S_{i-1}}\APK_{S_{i-1}-S_i}^{-1}\ .\label{eq:aggPK-2}
\end{equation}

When an Argennon account is created, both its $\PK$ and $\PK^{-1}$ is registered in the ARG smart contract, so the
inverse of any aggregate public key can be easily computed.\footnote{since the group operator of a cyclic
group is commutative, we have $(ab)^{-1}=a^{-1}b^{-1}$.}




    \section{Consensus}\label{sec:consensus}
    %! Author = aybehrouz


\note{not yet written....}

\subsection{Estimating A User's Stake}\label{subsec:estimating-a-user's-stake}

In a proof of stake system the influence of a user in the consensus protocol should be proportional to the amount
of stake the user has in the system. Conventionally in these systems, for estimating a user's stake, we use the
amount of native system tokens the user is holding. Unfortunately, one problem with this approach is that a
strong attacker may be able to obtain a considerable amount of system tokens, for example by borrowing from a
DEFI application, and use this stake to attack the system.

To mitigate this problem, for calculating a user's stake at time step \(t\), instead of using the raw ARG
balance, we use the minimum of a \emph{trust value} the system has calculated for the user and the user's
ARG balance:
\[
    S_{u,t} = \min (B_{u,t}, Trust_{u,t})
\]
Where:
\begin{itemize}
    \item \(S_{u,t}\) is the stake of user \(u\) at time step \(t\).
    \item \(B_{u,t}\) is the ARG balance of user \(u\) at time step \(t\).
    \item \(Trust_{u,t}\) is an estimated trust value for user \(u\) at time step \(t\).
\end{itemize}

The agreement protocol, at time step \(t\), will use \(\sum_{u}S_{u,t}\) to determine the required
number of votes for the confirmation of a block, and we let \(Trust_{u,t} = M_{u,t}\), where \(M_{u,t}\) is the
exponential moving average of the ARG balance of user \(u\) at time step \(t\).

In our system a user who held ARGs and participated in the consensus for a long time is more trusted
than a user with a higher balance whose balance has increased recently. An attacker who has obtained a large
amount of ARGs, also needs to hold them for a long period of time before being able to attack the system.

For calculating the exponential moving average of a user's balance at time step \(t\), we can use the following
recursive formula:
\[
    M_{u,t} = (1 - \alpha) M_{u,t-1} + \alpha B_{u,t} = M_{u,t-1} + \alpha (B_{u,t} - M_{u,t-1})
\]
Where the coefficient \(\alpha\) is a constant smoothing factor between \(0\) and \(1\) which represents the
degree of weighting decrease, A higher \(\alpha\) discounts older observations faster.

Usually an account balance will not change in every time step, and we can use older values of EMA for calculating
\(M_{u,t}\): (In the following equations the \(u\) subscript is dropped for simplicity)
\[
    M_{t} = (1 - \alpha)^{t-k}M_{k} + [1 - (1 - \alpha)^{t - k}]B
\]
Where:
\[
    B = B_{k+1} = B_{k+2} = \dots = B_{t}
\]
We know that when \(|nx| \ll 1\) we can use the binomial approximation \({(1 + x)^n \approx 1 + nx}\). So, we can
further simplify this formula:
\[
    M_{t} = M_{k} + (t - k) \alpha (B - M_{k})
\]

For choosing the value of \(\alpha\) we can consider the number of time steps that the trust value of a user needs
for reaching a specified fraction of his account balance. We know that for large \(n\) and \(|x| < 1\) we have
\((1 + x)^n \approx e^{nx}\), so by letting \(M_{u,k} = 0\) and \(n = t - k\) we can write:
\[
    \alpha =- \frac{\ln\left(1 - \frac{M_{n+k}}{B}\right)}{n}
\]
The value of \(\alpha\) for a desired configuration can be calculated by this equation. For instance, we could
calculate the \(\alpha\) for a relatively good configuration in which \(M_{n+k} = 0.8B\) and \(n\) equals to the
number of time steps of 10 years.

In our system a newly created account will not have voting power for some time, no matter how high its
balance is. While this is a desirable property, in case a large proportion of total system tokens are
transferred to newly created accounts, it can result in too much voting power for older accounts. This may decrease
the degree of decentralization in our system.

However, this situation is easily detectable by comparing the total stake of the system with the total balance of
users. If after confirming a block the total stake of the system goes too low and we have:
\[
    \sum_{u}S_{u,t} < \gamma \sum_{u}B_{u,t}
\]
The protocol will perform a \emph{time shift} in the system: the time step of the system
will be incremented for \(m\) steps while no blocks will be confirmed. This will increase the value of \(M_{u,t}\)
for new accounts with a non-zero balance, giving them more influence in the agreement protocol.

For calculating the value of \(m\) which determines the amount of time shift in the system, we should note that when
\(B_{u,t} = B_{u, t-1} = B_u\), we can derive a simple recursive rule for the stake of a user:
\[
    S_{u,t} = (1 - \alpha) S_{u,t-1} + \alpha B_u
\]
Therefore, we have:
\[
    \sum_{u}S_{u,t} = (1 - \alpha) \sum_{u}S_{u,t - 1} + \alpha \sum_{u}B_u
\]
This equation shows that when the balance of users is not changing over time the total stake of the system is the
exponential average of the total ARGs of the system. Consequently, when we shift the time for \(m\) steps, we can
calculate the new total stake of the system from the following equation:

\[
    \sum_{u}S_{u,t+m} = (1 - \alpha)^{m}\sum_{u}S_{u,t} + [1 - (1 - \alpha)^{m}]\sum_{u}B_u
\]
Hence, if we want to increase the total stake of the system from \(\gamma \sum_{u}B_u\) to \(\lambda \sum_{u}B_u\),
we can obtain \(m\) from the following formula, assuming \(\alpha\) is small enough:
\[
    m = \frac{1}{\alpha} \ln \left(\frac{1 - \gamma}{1 - \lambda}\right)
\]

    %! Author = aybehrouz


\section{Applications}\label{sec:applications}

An Argennon application or smart contract is an HTTP server which is represented by an Argennon Standard
Representation (ASR) and whose state is stored in the Argennon blockchain. Each Argennon application is identified by
a unique application identifier.

An application identifier, \texttt{applicationID}, is
a unique prefix code generated by the \emph{applications} prefix tree. (See Section~\ref{sec:identifiers}.)
An application identifier can be considered as the address of an application and has the following standard symbolic
representation:
\begin{verbatim}
<application-id> ::= <decimal-prefix-code>
<decimal-prefix-code> ::= <dec-num>"."<decimal-prefix-code> | <dec-num>
\end{verbatim}
where \texttt{<dec-num>} is a normal decimal number between $0$ and $255$. For example \texttt{21.255.37},
\texttt{0}, \texttt{11.6} and \texttt{2.0.0.0.0}, are valid application addresses.

Argennon has two special smart contracts: the \emph{root smart contract}, also called the \emph{root application}, and
the \emph{ARG smart contract}, which is also called the \emph{Argennon smart contract} or the \emph{ARG application}.

Argennon application use HTTP as the application protocol and they are advised to have a RESTful API design.

\subsection{The Root Application}\label{subsec:the-root-app}

The root application or the root smart contract, with \texttt{applicationID = 0}, is a privileged smart contract
responsible for installation/uninstallation of other smart contracts. The Argennon's root smart contract
performs three main operations:

\begin{itemize}
    \item Installation of new Argennon applications and determining the update policy of a smart
    contract: if the contract is updatable or not, which accounts or smart contracts can update or uninstall
    the contract, and so on.
    \item Removing an Argennon application (if allowed).
    \item Updating an Argennon application (if allowed).
\end{itemize}

The root smart contract is a mutable smart contract and can be updated by the Argennon governance system.
(See Section~\ref{sec:adags})

\subsection{The ARG Application}\label{subsec:the-arg-app}

The ARG application or the ARG smart contract,
with \texttt{applicationID = 1}, controls the ARG token, the main
currency of the Argennon blockchain. This smart contract also manages a database of public keys and
handles signature verification.

The ARG smart contract is a mutable smart contract and can be updated by the Argennon governance system.


\section{Accounts}\label{sec:accounts}

Argennon accounts are entities defined inside the ARG application.
Every Argennon account is uniquely identified by a prefix code generated using \emph{accounts} prefix
tree. (See Section~\ref{sec:identifiers}) An account
identifier can be considered as the address of an account and has the following standard symbolic representation:
\begin{verbatim}
<account-id> ::= "0x"<hex-num>
\end{verbatim}
where \texttt{<hex-num>} is a hexadecimal number, using lower case
letters \texttt{[a-f]} for showing digits greater than $9$.

For example \texttt{0x24ffda}, \texttt{0x0} and \texttt{0x03a0000}, are valid standard symbolic
representations of account addresses.

A new account can be created by sending a proper HTTP request to the ARG smart contract. For creating
a new account two public keys need to be provided by the caller and registered in the Argennon smart contract.
One public key will be used for issuing digital signatures, and the other one will be used for voting. The
provided public keys need to meet certain cryptographic requirements,\footnote{Argennon uses Prove
Knowledge of the Secret Key (KOSK) scheme.} and can not be already registered in the system.

If the owner of the new account is an application, the \texttt{applicationID} of the owner will be registered in the
ARG smart contract and no public keys are needed. An application can own an arbitrary number of accounts.

\note{Explicit key registration enables Argennon to decouple cryptography from the blockchain design. In this way,
    if the cryptographic algorithms used become insecure for some reason, for example because
    of the introduction of quantum computers, they could be easily upgraded.}


\section{Transactions}\label{sec:transactions}

An Argennon transaction consist of an HTTP request made by a user, to an Argennon application and a resource
declaration object. Transactions can only be
issued by users and applications can not create transactions. An Argennon transaction is also called
an \emph{external request}.

\subsection{Resource Declaration}\label{subsec:resource-declaration}

Every Argennon transaction needs to specify a maximum for
The Argennon protocol requires every  an execution cost for
every AVM instruction, reflecting the amount of resources its emulation needs. Every
transaction is required to specify two maximum execution costs: \texttt{maxInternalCost}
and \texttt{maxExternalCost}. The \emph{external} execution cost of a transaction is the \textbf{overall} cost of its
\texttt{invoke\_dispatcher} and \texttt{invoke\_later} instructions,\footnote{By overall cost, we mean the execution
cost needed for reaching the next instruction.}. The remaining execution cost will be considered as
the \emph{internal} cost. If a transaction reaches one of its maximum execution costs, executing any instruction
which has that type of cost, will throw an AVM exception.

\note{When a transaction reaches its \texttt{maxExternalCost}, it can still execute
its own code, while it can not call other smart contracts.
This way the execution cost of a smart contract is completely decoupled from the
smart contract it calls, and a malicious contract can not make its invoker certainly fail
by using infinite loops.}

Also, Argennon transactions are required to specify what heap or code area addresses they will access. This will
enable validators to parallelize transaction validation as we will see in Section~\ref{sec:concurrency}. An instruction
that tries to access a memory location that is not in the access list of the transaction, will throw an exception.
Users could use off-chain \emph{execution oracles} to predict the list of memory locations their transactions need.

An execution oracle is a full AVM emulator that keeps a full local copy of the AVM storage and can emulate AVM
execution without accessing a ZK-EDB server. Execution oracles can be used for reporting useful information about
Argennon transactions such as accessed AVM heap or code area locations, exact amount of execution cost,
and so on.

Every Argennon transaction is required to provide the following information as an upper bound for the
resources it needs:

\begin{itemize}
    \item Maximum internal execution cost
    \item Maximum external execution cost
    \item A list of heap/code-area locations for reading
    \item A list of heap locations for writing
    \item A list of heap chunks it will deallocate (if any)
    \item A list of methods it will delete (if any)
    \item Number and size of heap chunks it will allocate (if any)
    \item Number and size of method bytecodes it will allocate (if any)
\end{itemize}

If a transaction tries to violate any of these predefined limitations, it will be considered failed, and the network
can receive the proposed fee of that transaction.

\begin{lstlisting}[language=python, frame=TB, float, title=An Argennon transaction in YAML format,
    label={lst:txn-example}]
---
fee: |
    PUT /balances/0x73.0xa2?to=fee&amount=0.26&sig=5b73CbmwQNRC7fWUY15 HTTP/1.1
call: |
    POST bttp://dapp.argennon.net/54.189.21/proposals HTTP/1.1
    Content-Type: application/json; charset=utf-8
    Content-Length: 77

    {
        "name": "Grant",
        "recipient": "0x24.0x8f.0x29.0xa1",
        "amount": 25000,
        "sig"= "2a36Gtrw249wQCD70nWY49d"
    }
caps:
    internal: 2500000 # maximum number of AVM execution clocks
    external: 1000000
    read: [(2654,3),(15642,0),(15642,1),(15642,3)]
    write: [(15642,0),(20154,0),(20154,1)]
\end{lstlisting}

\subsection{Authorization}\label{subsec:txn-auth}

Argennon transactions do not have a sender. The authorization of the requested operation is always done by checking the
digital signatures that are provided as a part of the HTTP request to the \texttt{dispatcher} method.

While every block of the Argennon blockchain stores the commitment of the transaction list, Argennon does not enforce
storage of the transaction history. To be able to detect replay attacks, we require
every signature that a user creates to have a nonce. This nonce consists of the issuance round of the signature
and a sequence number: \texttt{(issuance,\ sequence)}. When a user creates more than one signature in a round, he
must sequence his signatures starting from 0 (i.e.~the sequence number restarts from 0 in every round). We define
a maximum lifetime for signatures, so a signature is invalid if \texttt{currentRound - issuance > maxLifeTime} or
if a signature of the same user with a bigger or equal nonce is already used
(i.e.~is recorded in the blockchain). A nonce is bigger than another nonce if it has an older issuance. If two
nonces have an equal issuance, the nonce with the bigger sequence number will be considered bigger.

To be able to detect invalid signatures, we keep the maximum nonce of used digital signatures per user. This information
is stored in the ARG smart contract and when the difference between \texttt{issuance} component of the nonce and
the current round becomes bigger than the maximum allowed lifetime of a signature, it can be safely
deleted.\footnote{in some conventional
blockchains, the nonce data can never be deleted, even if the account has zero balance and is no longer used.}

\subsection{Transaction Fee}\label{subsec:fee}

Every Argennon transaction is required to pay two types of fees: execution fee, which is paid for executing the
transaction, and storage fee, which is paid for the amount of storage the transaction allocates.

A transaction pays
its fees by providing digital signatures of one or more accounts, authorizing the transfer of the amount of fee in
ARGs from one or more accounts to the fee sink accounts. This fee is transferred by the first
\texttt{i\_invoke\_dispatcher} instruction of the transaction.

\note{An Argennon transaction always pays all of its proposed fee, no matter how much of its predefined resources
were not used during the final emulation. This will incentivize users to report the resource usage of their
transactions more accurately.}




    \section{Incentive mechanism}\label{sec:incentive-mechanism}
    %! Author = aybehrouz

\note{TODO: update this section!}

\subsection{Transaction Fee}\label{subsec:transaction-fee}

Every transaction in the Argennon blockchain starts with an \texttt{invoke\_external} instruction which calls a
special method from the root smart contract. This method will transfer the proposed fee of the transaction in ARGs
from a sender account to the fee sink accounts. Argennon has two fee sink accounts: \texttt{execFeeSink} collects
execution fees and \texttt{dbFeeSink} collects fees for ZK-EDB servers. The Protocol decides how to distribute the
transaction fee between these two fee sink accounts.

When a block is added to the blockchain, the proposer of that block will receive a share of the block fees.
Consequently, a block proposer is always incentivized to include more transactions in his block. However, if he
puts too many transactions in his block and the validation of the block becomes too difficult, some validators
may not be able to validate all transactions on time. If a validator can not validate a block in the required
time, he will consider the block invalid. So, when a proposed block contains too many transactions, the network
may reach consensus on another block, and the proposer of that block will not receive any fees. As a result, a
proposer is incentivized to use network transaction capacity optimally.

On the other hand, we believe that the proposer does not have enough incentives for optimizing the storage size
of the transaction set. Therefore, we require that \textbf{the size of the transaction set of every block in
bytes be lower than a certain threshold.}

Validators need to spend resources for validating transactions. When a validator starts the emulation of the AVM
to validate a transaction, solely from the code he can't predict the time the execution will finish. This will
give an adversary an opportunity to attack the network by broadcasting transactions that never ends. Since,
validators can not finish the execution of these transactions, the network will not be able to charge the
attacker any fees, and he would be able to waste validators resources for free.

To mitigate this problem, we require that every transaction specify a cap for all the resources it needs. This
will include memory, network and processor related resources. Also, the protocol defines an execution cost for
every AVM instruction reflecting the amount of resources its emulation needs. This will define a standard way for
measuring the execution cost of any \texttt{avmCall} transaction. Every \texttt{avmCall} transaction is required
to specify a maximum execution cost. If during emulation it reaches this maximum cost, the transaction will be
considered failed and the network can receive the proposed fee of that transaction.


\subsection{Incentives for ZK-EDB Servers}\label{subsec:zk-edb-servers}

The incentive mechanism for ZK-EDB servers should have the following properties:

\begin{itemize}
    \item It incentivizes storing all memory blocks, whether a heap page or a code area block, and not only those
    which are used more frequently.
    \item It incentivizes ZK-EDB servers to actively provide the required memory blocks for validators.
    \item Making more accounts will not provide any advantages for a ZK-EDB server.
\end{itemize}

For our incentive mechanism, we require that every time a validator receives a memory block from a ZK-EDB, after
validating the data, he give a receipt to the ZK-EDB. In this receipt the validator signs the following information:

\begin{itemize}
    \item \texttt{ownerAddr}: the ARG address of the ZK-EDB\@.
    \item \texttt{receivedBlockID}: the ID of the received memory block.
    \item \texttt{round}: the current round number.
\end{itemize}

\note{In a round, an honest validator never gives a receipt for an identical memory block to two different ZK-EDBs.}

To incentivize ZK-EDB servers, a lottery will be held every round and a predefined amount of ARGs from
\texttt{dbFeeSink} account will be distributed between winners as a prize. This prize will be divided equally
between all \emph{winning tickets} of the lottery.

\note{One ZK-EDB server could own multiple winning tickets in a round.}

To run this lottery, every round, based on the current block seed, a collection of \emph{valid} receipts will be
selected randomly as the \emph{winning receipts} of the round. A receipt is \emph{valid} in round $r$ if:

\begin{itemize}
    \item The signer was a validator in the round $r - 1$ and voted for the agreed-upon block.
    \item The data block in the receipt was needed for validating the \textbf{previous} block.
    \item The receipt round number is $r - 1$.
    \item The signer did not sign a receipt for the same data block for two different ZK-EDBs in the previous round.
\end{itemize}
For selecting the winning receipts we could use a random generator:
\begin{verbatim}
IF random(seed|validatorPK|receivedBlockID) < winProbability THEN
    the receipt issued by validatorPK for receivedBlockID is a winner
\end{verbatim}
\begin{itemize}
    \item \texttt{random()} produces uniform random numbers between 0 and 1, using its input argument as a seed.
    \item \texttt{validatorPK} is the public key of the signer of the receipt.
    \item \texttt{receivedBlockID} is the ID of the memory block that the receipt was issued for.
    \item \texttt{winProbability} is the probability of winning in every round.
    \item \texttt{seed} is the current block seed.
    \item \texttt{|} is a concatenation operator.
\end{itemize}

\note{The winners of the lottery were validators one round before the lottery round.}

Also, based on the current block seed, a random memory block, whether a heap page or a code area block, is
selected as the challenge of the round. A ZK-EDB that owns a winning receipt needs to broadcast a \emph{winning
ticket} to claim his prize. The winning ticket consists of a winning receipt and a \emph{solution} to the round
challenge. Solving a round challenge requires the content of the memory block which was selected as the round
challenge. This will encourage ZK-EDBs to store all memory blocks.

A possible choice for the challenge solution could be the cryptographic hash of the content of the challenge
memory block combined with the ZK-EDB ARG address: \texttt{hash(challenge.content|ownerAddr)}

The winning tickets of the lottery of round $r$ need to be included in the block of the round $r$,
otherwise they will be considered expired. Validation and prize distribution for the winning tickets of round
$r$ will be done in the round $r + 1$. This way, \textbf{the content of the challenge memory block could be
kept secret during the lottery round.} Every winning ticket will get an equal share of the lottery prize.

\subsection{Memory Allocation and De-allocation Fee}\label{subsec:memory-allocation-and-de-allocation}

Every $k$ round the protocol chooses a price per byte for AVM memory. When a smart contract executes a heap
allocation instruction, the protocol will automatically deduce the cost of the allocated memory from the ARG
address of the smart contract.

To determine the price of AVM memory, Every $k$ round, the protocol calculates \texttt{dbFee} and
\texttt{memTraffic} values. \texttt{dbFee} is the aggregate amount of collected database fees, and
\texttt{memTraffic} is the total memory traffic of the system. For calculating the memory traffic of the system
the protocol considers the total size of all the memory pages that were accessed for either reading or writing
during a time period. These two values will be calculated for the last $k$ rounds and the price per byte of
AVM memory will be a linear function of \texttt{dbFee/memTraffic}

When a smart contract executes a heap de-allocation instruction, the protocol will refund the cost of
de-allocated memory to the smart contract. Here, the current price of AVM memory does not matter and the protocol
calculates the refunded amount based on the average price the smart contract had paid for that allocated memory.
This will prevent smart contracts from profit taking by trading memory with the protocol.


    \chapter{Governance}\label{ch:governance}


    \section{ADAGs}\label{sec:adags}
    The Argennon Decentralized Autonomous Governance system (ADAGs)
    \note{not yet written...}


    \chapter{The Argon Language}\label{ch:argon-lang}
    %! Author = aybehrouz

\lstset{
    language=Java,
    morekeywords={main, initialize, constructor, external, virtual, Map, Account, signature, overrides, is, string},
    basicstyle=\small\sffamily,
    columns=fullflexible,
%   keywordstyle=\bfseries,
    breaklines=true,
    showstringspaces=false,
    mathescape
}

\section{Introduction}\label{sec:introduction2}

The Argon programming language is a class-based, object-oriented language designed for writing Argennon's smart
contracts. The Argon programming language is inspired by Solidity and is similar to Java, with a number of aspects
of them omitted and a few ideas from other languages included. Argon is designed to be fully compatible with
the Argennon Virtual Machine and be able to use all advanced features of the Argennon blockchain.

Argon applications (i.e.\ smart contracts) are organized as sets of packages. Each package has its own set of names
for types, which helps to prevent name conflicts. Every package can contain an arbitrary number of classes.
Every Argon application is required
to have exactly one \texttt{main} method and one \texttt{initialize} method. The \texttt{main} method is the
only method of an Argon application which can be called by other smart contracts.

The \texttt{main} method is required to have a single parameter named \texttt{request}. The type of this parameter
should be \texttt{RestRequest} or \texttt{HttpRequest}. The return value of the \texttt{main} function needs to be a
\texttt{RestResponse} or \texttt{HttpResponse}.


\section{Features Overview}\label{sec:features-overview}

\subsection{Access Level Modifiers}\label{subsec:access-level-modifiers}

Access level modifiers determine whether other classes can use a particular field or invoke a particular method or
if a method can be invoked externally by other smart contracts.

\begin{center}
    \begin{tabular}{lllll}
        \hline
        & Class & Package & Subclass & Program \\
        \hline
        private   & yes   & no      & no       & no  \\
        protected & yes   & no      & yes      & no  \\
        package   & yes   & yes     & yes      & no  \\
        public    & yes   & yes     & yes      & yes \\
        \hline
    \end{tabular}\label{tab:table}
\end{center}


\begin{lstlisting}[frame=TB, float, title=A simple Argon application,label={lst:code1}]
public class MirrorToken {
    private static SimpleToken token;
    private static SimpleToken reflection;

    // `initialize` is a special static method that is called by the AVM after the code of a contract
    // is stored in the AVM code area.
    public static void initialize(double supply1, double supply2) {
        // `new` does not create a new smart contract. It just makes an ordinary object.
        token = new SimpleToken(supply1);
        reflection = new SimpleToken(supply2);
    }
    // `main` is the only method of the application (i.e. smart contract) that can be called
    // by other applications. Every application should have exactly one main method defined
    // in some class. Alternatively, the keyword `dispatcher` could be used instead of `main`.
    public static RestResponse main(RestRequest request) {
        RestResponse response = new RestResponse();
        if (request.pathMatches("/balances/{user}")) {
            Account sender = request.getParameter<Account>("user");
            if (request.operationIsPUT()) {
                sender.authorize(request.toMessage(), request.getParameter<byte[]>("sig"));
                Account recipient = request.getParameter<Account>("to");
                double amount = request.getParameter<double>("amount");
                token.transfer(sender, recipient, amount);
                reflection.transfer(recipient, sender, Math.sqrt(amount));
                return response.setStatus(Http.Status.OK);
            } else if (request.operationIsGET()) {
                response.append<double>("balance", token.balanceOf(sender));
                response.append<double>("reflection", reflection.balanceOf(user));
                return response.setStatus(Http.Status.OK);
            } else {
                return response.setStatus(Http.Status.MethodNotAllowed);
            }
        }
    }
}

package class SimpleToken {
    private Map(Account -> double) balances;

    // The visibility of a member without an access modifier will be the package level.
    constructor(double initialSupply) {
        // initializes the object
    }

    void transfer(Account sender, Account recipient, double amount) {
        if (balances[sender] < amount) throw("Not enough balance.");
        // implements the required logic...
    }
    // implements other methods...
}
\end{lstlisting}

\subsection{Shadowing}\label{subsec:shadowing}

If a declaration of a type (such as a member variable or a parameter name) in a particular scope (such as an
inner block or a method definition) has the same name as another declaration in the enclosing scope, it will
result in a compiler error. In other words, the Argon programming language does not allow shadowing.

\end{document}


