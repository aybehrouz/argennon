%! suppress = TooLargeSection
%! Author = aybehrouz
%! Date = 2/28/21


\documentclass[11pt, a4paper]{report}

% Packages
\usepackage{tipa}
\usepackage{graphicx}
\usepackage{hyperref}
\usepackage{amsfonts}
\usepackage[ruled,boxed]{algorithm2e}
\usepackage{algpseudocode}
\usepackage{listings}
\usepackage{amssymb}
\usepackage{tcolorbox}
\usepackage{amsmath}
\usepackage{mathtools}
\usepackage[a4paper]{geometry}
\usepackage{tikz}

%\binoppenalty=99999
%\relpenalty=99999
\newcommand{\note}[1] {
    \begin{tcolorbox}[colframe=white,colback=white]
        \emph{#1}
        %\emph{#1}\medskip
    \end{tcolorbox}
}
\mathchardef\mhyphen="2D
\lstset{
    language=Java,
    morekeywords={main, initialize, constructor, external, virtual, Map, Account, signature, overrides, is, string},
    basicstyle=\small\sffamily,
    columns=fullflexible,
%   keywordstyle=\bfseries,
    breaklines=true,
    showstringspaces=false,
    mathescape
}

\title{Argennon: A Scalable Cloud Based Smart Contract Platform Using Argument of Knowledge Systems}
\author{aybehrouz}
\date{January 2021}


\begin{document}
    \maketitle
    \begin{abstract}
        Argennon is a next generation cloud based blockchain and smart
        contract platform. The Argennon blockchain uses
        a hybrid proof of stake (HPoS) consensus protocol, which is capable of combining the benefits of
        a centralized and a decentralized system. In Argennon, ledger storage and transaction processing are
        outsourced to the cloud and normal personal computers or smartphones, with limited hardware
        capabilities, are able to validate transactions and actively
        participate in the Argennon consensus protocol. This property makes Argennon a truly decentralized and
        democratic blockchain and one of the most secure existing platforms.

        The Argennon cloud is trustless and publicly verifiable. Computational Integrity (CI) is
        achieved by using succinct argument of knowledge systems (STARK/SNARK)
        and data integrity is guaranteed by cryptographic accumulators.

        The Argennon protocol strongly incentivizes the formation of a permission-less network of Publicly Verifiable
        Cloud (PVC) servers. A PVC server in Argennon, is a conventional data server which uses its computational and
        storage resources to help the Argennon network process transactions.
    \end{abstract}
    \tableofcontents


    \chapter{Introduction}\label{ch:intro}
    The most common use for blockchains is in financial applications. This gives a crucial importance to the
security of the consensus protocol used in a blockchain. Unfortunately, many currently used blockchains are
vulnerable to a certain type of consensus attack, known as the Bribery Attack. In a Bribery Attack, an adversary
tries to corrupt participants of a protocol by offering them money and seducing them to violate the protocol.

At the time of writing of this document,
the total mining reward for a Bitcoin block is around \$150,000. If we assume, in decision theoretic terminology,
that the mining reward accurately defines the utility function of a Bitcoin miner, one could hire all hashing power
of the Bitcoin network for one hour by
spending only \$750,000. The situation is not much different for PoS blockchains, as long as the total
stake of the validator set is a relatively small value. By
stake, we mean a real number measuring the total interest of a user in the system, and we are not referring, in
particular, to some locked amount of a user's money that is known as stake in some PoS protocols.
This problem is more severe in blockchains that use randomly selected small sets of validators. These small sets
usually have low total stake and could be easily bribed and corrupted. Selecting these random sets
by hidden random procedures would not help, since the validator himself knows he has been selected, before casting
his vote.

It appears that the only solution to this important vulnerability is to effectively participate all the stakeholders
of the system in the consensus protocol. This large participation makes the protocol resilient against bribery or
collusion because the adversary would need to spend unrealistic amounts of money to bribe enough users.

However, for effective participation in the consensus protocol, a validator needs to be able to detect
illegal transactions. Detecting illegal transactions can be done by accessing the ledger
state and executing transactions according to the protocol rules. The
ledger state, even for small blockchains, could be several
hundreds of gigabytes, and executing transactions could easily become costly when a blockchain is acting as a
smart contract platform. This computational and storage overhead, in practice, could prevent most of
ordinary users from any type of participation in the process of securing a blockchain.

Although a fully decentralized blockchain based on the participation of every user looks appealing, it is not as perfect
as it might seem. The consensus protocol of a blockchain relies on a network of computers, not humans. Ordinary users
use simple and similar computer systems. That means, they all have similar vulnerabilities and weaknesses which could be
used by an adversary to catastrophically attack the consensus protocol. For instance, if a malware, probably using a
common zero-day
vulnerability, has the ability to infect a large portion of normal personal computers, an adversary can use it to
take control of the majority of participants in the consensus protocol and compromise the security of the
blockchain.

Securing a computer system against cyberattacks needs planning ahead and access to engineering resources.
Special software and hardware, like custom-built operating systems and isolated specialized hardware are required.
This is not something a normal user can afford. But powerful centralized entities, having large financial and
technical resources, could actually build
systems that are resilient against sophisticated cyberattacks. In this regard, we have to admit, a centralized system is
arguably superior to a decentralized\footnote{Note that decentralized and distributed are two different concepts.}
system.

To overcome these difficulties, Argennon\footnote{The classical pronunciation should be used:\textipa{/Ar"gen.non/}}
uses a Hybrid Proof of Stake consensus protocol, which is
capable of combining the benefits of a centralized and a decentralized system. A small committee of
delegates is democratically elected by users via the Argennon Decentralized Autonomous Governance system
(ADAGs). This committee usually\footnote{The election term and the number of committee members can be changed by the
ADAGs.} is elected for a one-year term, has five members, and is responsible for minting new
blocks of the Argennon blockchain.
Each minted block gets certified by all members of the delegates committee and after that it must get approved
by its corresponding validator assembly.

Validator assemblies are very
large sets of validators including at least three percent of the maximum possible stake of the Argennon network.
Every validator assembly has an index between $0$ and $m - 1$, and is responsible for validating block number $n$,
if $n$ modulo $m$ equals the index of the assembly. A block is approved if it takes approval votes from at
least $2/3$ of the total stake of its validator assembly.

In case the main committee fails to generate new blocks or behaves maliciously, a special Recovery Protocol is
initiated by validators.
This recovery protocol can recover the functionality
of the Argennon blockchain as long as more than $2/3$ of the total stake of the system is controlled by honest users
and any network partition resolves after a finite amount of time. The Recovery Protocol uses two main emergency
procedures to recover the functionality of the Argennon blockchain: \emph{Emergency Forking} and \emph{Emergency
Agreement} protocol.

In Argennon, a block is considered final after its \textbf{next} block gets certified by \textbf{both} the
delegates committee and its validator assembly.
The Argennon protocol ensures that as long as more than half of the total stake of validators is
controlled by honest users, the probability of discarding a final block is near zero even if all the delegates are
malicious.

Each block of the Argennon blockchain contains a set of Computational Integrity (CI) statements and a commitment to
the final ledger state of the block (i.e.\ the ledger state after executing all transactions of the block). The CI
set defines how the final state of the block can be reached from
the state of its previous block via a set of intermediate state transitions. Each individual CI statement defines
an ordered list of external requests\footnote{External requests in Argennon are
similar to transactions in older blockchains.} and determines the state before and after executing those requests.

More formally, each CI statement has the form $\tau_{\mathbb{S}, \hat{\mathbb{S}}, \mathbb{R}}$, and states that
$\hat{\mathbb{S}}$ is the commitment of the next state after executing an ordered list of external requests with
commitment $\mathbb{R}$ on a state which has commitment $\mathbb{S}$.
Commitments to states are produced by a cryptographic accumulator.
Validators can\footnote{Using the Argennon cloud is optional for a validator.} receive succinct cryptographic proofs
of these CI statements (STARK/SNARK) from the Argennon cloud and validate blocks without storing the ledger state
or performing costly computation.

Verifying a succinct proof can be exponentially faster than replaying the computation. Moreover, verifications of
different CI statements are independent of each other and can be done in parallel. As a result, a validator can use
multiple cores for verifying CI statements of a \textbf{single block} and different validator assemblies can
simultaneously and
independently verify \textbf{different blocks}.
In addition, proof generation of different CI statements similarly can be done in parallel. However, for parallel
proof generation, the state transition needs to be known in advance. That's why in Argennon, delegates do not try
to generate proofs and focus all their computational power on executing external requests and
generating the state transition as fast as possible.

Argennon applications (i.e smart contracts) are stored in a high level text based language on the blockchain. This
language is intended for preserving the
high level information of the application logic to facilitate platform specific compiler optimization at a host
machine. This enables delegates to compile and optimize Argennon applications for their specific hardware platforms
and execute applications efficiently, ensuring the state transition can be found as fast as possible. Moreover,
validators by using a special compiler, generate arithmetization of this language and use it for verifying CI proofs.

Independence of CI statements is useful, but is not enough for having a truly scalable blockchain. To increase
parallelism, the Argennon protocol enforces all external requests to pre-declare their memory access locations. That
would enable a block proposer\footnote{In Argennon, delegates are the only block proposers.} to use Data Dependency
Analysis\footnote{See Section~\ref{sec:concurrency}} (DDA) to indicate independent sets of external requests (i.e.\
transactions) and use those sets for parallel processing. More importantly, these sets can be used for generating CI
statements that are defined on the \textbf{same initial state} and their proof can be generated independently without
the need to calculate the state transition in advance.

Centralized block generation brings some interesting features to the Argennon platform, such as flexible and lower
fees, off chain fee payment, optimistic instant transaction confirmation, and front running protection.
However, it also increases the possibility of transaction censorship. In Argennon, this problem is addressed by a
special High Priority Request (HPR) protocol.

Using Succinct Argument of Knowledge systems makes the main functionalities of an Argennon validator light enough to
be implemented as a smart contract. By deploying a validator contract on another platform, Argennon
could use more established blockchains as an extra layer of security, specially
during the bootstrapping phase, when ARG is not well distributed yet. In addition, this contract will facilitate
trustless bridging of assets from that platform to the Argennon blockchain. To reduce the execution fee, only
roll-ups of the state transition can be validated by the contract.

The Argennon protocol strongly incentivizes the formation of a \textbf{permission-less} network of Publicly Verifiable
Cloud (PVC) servers. To do so, the Argennon protocol conducts repetitive automatic lotteries between PVC servers.
A PVC server can increase its chance of winning by (i) generating proofs for more CI statements, (ii) storing all
parts of the ledger state and providing \emph{proofs of storage}.

A PVC server in Argennon, is a conventional data server which uses its computational and
storage resources to help the Argennon network process transactions. This encourages the development
of conventional networking, storage and compute hardware, which can benefit all areas of information technology.
This contrasts with the approach of some older blockchains that incentivized the development of a totally
useless technology of hash calculation.


    \chapter{The Argennon Smart Contract Execution Environment}\label{ch:AscEE}
    %! Author = aybehrouz


\section{Introduction}\label{sec:introduction}

The Argennon Smart Contract Execution Environment (AscEE) is an abstract high level execution environment for executing
Argennon smarts contracts (a.k.a\ Argennon applications) in an efficient and isolated environment. An Argennon
application essentially is an HTTP server whose state is
kept in the Argennon blockchain and its logic is described using an Argennon Standard Application Representation (ASAR).

An Argennon Standard Application Representation is a programming language for describing Argennon
applications, optimized
for the architecture and properties of the Argennon platform. This text based representation is low level
enough to enable easy compilation from any high level programming language and is high level enough to preserve
the high level information of the application logic and facilitate platform specific compiler optimization at a host
machine. In this regard, the ASAR can be considered as an Intermediate Representation (IR).

The state of an Argennon application is stored in byte addressable finite arrays of memory called
\emph{heap chunks}. An application may have several heap chunks with different sizes, and can remove or
resize its heap chunks or allocate new chunks. Every chunk belongs to exactly one application and can only be modified
by its owner. In addition to heap chunks, every application has a limited amount of non-persistent local memory for
storing temporary data.

The AscEE executes the requests contained in each block of the Argennon
blockchain in a three-step procedure. The first step is the \emph{preprocessing step}. In this step, the required
data for executing requests are retrieved and verified and the
helper data structures for next steps are constructed. This step is
designed in a way that can be done fully in parallel for each request without any risk of data races. The second
step is the \emph{Data Dependency Analysis (DDA) step}.
In this step by analyzing data dependency between requests, the AscEE determines requests that can be run in parallel
and requests that need to be run sequentially. This information is represented using an \emph{execution DAG} data
structure and in the final step, the requests are executed using this data structure.


\section{Execution Sessions}\label{sec:sessions}

The Argennon Smart Contract Execution Environment can be seen as a machine for executing Argennon applications to
fulfill \emph{external} HTTP requests\footnote{External requests are requests that are not made by other Argennon
applications.}, produce their HTTP responses and update related heap chunks. The execution of requests can be
considered sequential\footnote{Actually requests may be executed in parallel but by performing data dependency analysis
the result is guaranteed to be identical with the sequential execution of requests.} and each request has a separate
\emph{execution session}. In other words, an execution session is a separate session of executing application's code in
order to fulfill an external HTTP request. We call this external request the \emph{initiator} of the execution session.

The state of an execution session will be
destroyed at the end of the session and only the state of heap chunks is preserved. If a session fails and does not
complete normally, it will not have any effects on any heap chunks.

During an execution session an application can make \emph{internal} HTTP requests to other applications. These
requests will not start a new execution session and will be executed within the current session. In AscEE making an
internal HTTP request to an application is similar to a function invocation, and for that reason, we also refer to
them as \emph{application calls}.


\section{Memory}\label{mem}

Every Argennon application has two types of memory: local memory and heap. Local memory is not persistent and is
destroyed when the application call ends. Local memory is used for storing local variables and is not directly
addressable. Heap, on the other hand, is persistent and can be used for persisting data between application calls.
Heap is addressable and provides low level direct access. Both local memory and heap are limited, but the
AscEE does not specify any particular limits for them. When an application tries to use too much memory, that may
cause the execution session to end abruptly. In that case, the execution session will not have any effects on the
state of heap.

\subsection{Heap Chunks}\label{subsec:heap}

The AscEE heap is split into chunks. Each heap chunk is a continuous finite array of bytes, has a unique identifier, and
is byte addressable. An application may have several heap chunks with different sizes and can remove or
resize its chunks or allocate new ones. Every chunk belongs to exactly one application. Only the owner application can
modify a chunk but there is no restrictions for reading a chunk\footnote{The reason behind this type of access
control design is the fact that smart contract
code is usually immutable. That means if a smart contract does not implement a
getter mechanism for some parts of its internal data, this functionality can never
be added later, and despite the internal data is publicly available, there will be no
way for other smart contracts to use this data on-chain.}.

When an application allocates a new heap chunk, the identifier of the new chunk is not generated by
the AscEE. Instead, the application can choose an identifier itself, provided the new id has a correct format. This
is an important feature of the AscEE heap which allows applications to use the AscEE heap as a map
data structure\footnote{also called a dictionary.}.
Since the \texttt{chunkID} is a prefix code, any application has its own identifier space, and an application
can easily find unique identifiers for its chunks.

During an execution session every heap chunk has an access-type which may disallow certain operations. The
access-type of a chunk is declared by the initiator request of the execution session:

\begin{itemize}
    \item \texttt{check\_only}: only allows check operations. These operations query the persistence
    status of a memory location.
    \item \texttt{read\_only}: only allows read and check operations.
    \item \texttt{writable}: allows reading and writing.
    \item \texttt{additive}: only allows additive operations. By additive we mean an addition-like operator without
    overflow checking which is associative and commutative. Note that the content of these chunks cannot be read.
\end{itemize}

\subsubsection{Chunk Resizing}\label{subsubsec:ch-resize}

At the start of executing requests of a block, a validator can consider two values for every
heap chunk, the size: \texttt{chunkSize} and a size upper bound: \texttt{sizeUpperBound}. The value of
\texttt{chunkSize} can be determined uniquely at the start of
every execution session, and it may be updated during the session by the owner application. On the other hand,
the value of \texttt{sizeUpperBound} is proposed by the block proposer for each block and is calculated based on values
declared by external requests (i.e.\ transactions) that want to perform chunk resizing. This value needs to be
an upper bound of all the declared resizing values and indicates the upper bound of \texttt{chunkSize} during the
execution of a block.

The address space of a chunk starts from zero and only offsets lower than \texttt{sizeUpperBound} are valid. Trying to
access any offset higher than \texttt{sizeUpperBound} will always cause the execution session to end abruptly. There
is no way for an application to query \texttt{sizeUpperBound}. As a result, in the view of an application,
accessing offsets higher than \texttt{chunkSize} will result in undefined behaviour, while the behaviour is
well-defined for validators.
This enables validators to determine the validity of an offset at the start of the block validation in a parallelized
preprocessing phase without actually executing requests.

The value of \texttt{chunkSize}, can be modified during an execution session. However, the new values of size can
only be increasing or decreasing. More precisely, if a request declares that it wants to expand (shrink) a chunk, it
can only increase (decrease) the value of \texttt{chunkSize} and any specified value during the execution
session, needs to be greater (smaller) than the previous value of the \texttt{chunkSize}. Any request that wants to
expand (shrink) a chunk needs to specify a max size (min size). The value of \texttt{chunkSize} can not be set higher
(lower) than this value.

The value of \texttt{chunkSize} at the end of an execution session will determine if a memory location at an
offset is persistent or not: Offsets lower than the chunk size are persistent, and higher offsets are not.
Non-persistent locations will be re-initialized with zero at the start of every execution session.

Usually an application should not have any assumptions about the content of memory locations that are outside the chunk.
While these locations are zero initialized at the start of every execution session, multiple
invocations of an application may occur in a single execution session, and if one of them modifies a location outside
the chunk, the changes can be seen by next invocations.

While an application can use \texttt{chunkSize} to determine if an offset is persistent or not, that is not
considered a good practice. Reading \texttt{chunkSize} decreases transaction parallelization, and should be avoided.
Instead, applications should use a built-in AscEE function for checking the persistence status of memory addresses.

An application can load any chunk with a valid prefix identifier even if that chunk does not exist. For a non-existent
chunk the value of \texttt{chunkSize} is always zero.


\section{Identifiers}\label{sec:identifiers}

In Argennon a unique identifier is assigned to every application, heap chunk and account. Consequently, three distinct
identifier types exist: \texttt{appID}, \texttt{accountID}, and \texttt{chunkID}.
All these identifiers are \emph{prefix codes}, and hence can be represented by
\emph{prefix trees}\footnote{Also called tries.}.

Argennon has four primitive prefix trees: \emph{applications, accounts, local} and \emph{varUint}. All these trees
are in base 256, with the maximum height of 8.

An Argennon identifier may be simple or compound. A simple identifier is generated using a single tree, while a
compound identifier is generated by concatenating prefix codes generated by two or more trees:

\begin{itemize}
    \item \texttt{appID} is a prefix code built by \emph{applications} prefix tree. An \texttt{appID} cannot
    be \texttt{0x0}.

    \item \texttt{accountID} is a prefix code built by \emph{accounts} prefix tree. An \texttt{accountID} cannot
    be \texttt{0x0} or \texttt{0x1}.

    \item \texttt{chunkID} is a composite prefix code built by concatenating an \texttt{applicationID} to
    an \texttt{accountID} to a prefix code made by \emph{local} prefix tree:
    \subitem \texttt{chunkID = (applicationID|accountID|<local-prefix-code>)} .
\end{itemize}

All Argennon prefix trees have an equal branching
factor \(\beta\)\footnote{A typical choice for \(\beta\) is \(2^8\).}, and we can represent an Argennon
prefix tree as a sequence of fractional numbers in base \(\beta\):
\[
    (A^{(1)},A^{(2)},A^{(3)},\dots)\ ,
\]
where \(A^{(i)}=(0.a_{1}a_{2}\dots a_{i})_\beta\), and we have \(A^{(i)} \leq A^{(i+1)}\). \footnote{It's possible to
have \(a_i=0\). For example, \(A^{(4)}=(0.2000)_{10}\) is correct.}

One important property of prefix identifiers is that while they have variable and unlimited length, they are
uniquely extractable from any sequence. Assume that we have a string of digits in base $\beta$, we
know that the sequence starts with an Argennon identifier, but we do not know the length of that identifier.
Algorithm~\ref{alg:prefix_id} can be used to extract the prefixed identifier uniquely. Also, we can apply this algorithm
multiple times to extract a composite identifier, for example \texttt{chunkID}, from a sequence.

%##\includegraphics[width=17cm]{../img/Alg1s.png}
\begin{algorithm}[t]
    \DontPrintSemicolon
    \SetKwInOut{Input}{input}\SetKwInOut{Output}{output}
    \Input{A sequence of $n$ digits in base $\beta$: $d_{1}d_{2}\dots d_{n}$ \newline
    A prefix tree: $<A^{(1)},A^{(2)},A^{(3)},\dots>$}
    \BlankLine
    \Output{Valid identifier prefix of the sequence.}
    \BlankLine
    \For{$i = 1$ \KwTo $n$}
    {
        \If{$(0.d_{1}d_{2}\dots d_{i})_\beta < A^{(i)}$}
        {
            \KwRet{$d_{1}d_{2}\dots d_{i}$}\;
        }
    }
    \KwRet{NIL}\;
    \caption{Finding a prefixed identifier}\label{alg:prefix_id}
\end{algorithm}

When we have a prefixed identifier, and we want to know if a sequence of digits is marked by that identifier,
we use Algorithm~\ref{alg:match_id} to match the prefixed identifier with the start of the sequence. The matching
can be done with only three comparisons, and an invalid prefixed identifier can be detected and will not match
any sequence.

In Argennon the shorter prefix codes are assigned to more active accounts and applications which tend to own more
data objects in the system. The prefix trees are designed by analyzing empirical data to make sure the number
of leaves in each level is chosen appropriately.

\begin{algorithm}[h]
    \DontPrintSemicolon
    \SetKwData{Id}{$id$}
    \SetKwInOut{Input}{input}\SetKwInOut{Output}{output}
    \Input{A prefixed identifier in base $\beta$ with $n$ digits: $\Id=a_{1}a_{2}\dots a_{n}$ \newline
    A sequence of digits in base $\beta$: $d_{1}d_{2}d_{3}\dots $ \newline
    A prefix tree: $<0,A^{(1)},A^{(2)},A^{(3)},\dots>$
    }
    \BlankLine
    \Output{$TURE$ if and only if the identifier is valid and the sequence starts with the identifier.}
    \BlankLine
    \If{$(0.a_{1}\dots a_{n})_\beta = (0.d_{1}\dots d_{n})_\beta$}
    {
        \If{$A^{(n-1)} \leq (0.a_{1}a_{2}\dots a_{n})_\beta < A^{(n)}$}
        {
            \KwRet{TRUE}\;
        }
    }
    \KwRet{FALSE}\;
    \caption{Matching a prefixed identifier}\label{alg:match_id}
\end{algorithm}

\section{Request Attachments}\label{sec:attachments}

The attachment of a request is a list of request identifiers of the current block that are \emph{attached} to the
request. That means, for validating the request a validator first needs to \emph{inject} the digest of attachments
into the current HTTP request text. By doing so, the called application will have access to the digest of
attachments in a secure and authenticated way.

The main usage of this feature is for fee payment. A request that wants to pay the fees for a number of requests,
declares those requests as its attachments. For paying fees the payer signs the digest of requests for which he
wants to pay fees. After injecting the digest of those request by validators, that signature can be validated
correctly and securely by the application that handles fee payment.


\section{Authorization}\label{sec:auth}

In blockchain applications, we usually need to authorize certain operations. For example, for sending an asset
from a user to another user, we need to make sure that the sender has authorized that operation.

The AscEE uses \emph{Authenticated Message Passing} for authorizing operations. In this method, every execution
session has a set of authenticated messages, and those messages are \textbf{explicitly} passed in requests to
applications for authorizing operations. These messages act exactly like digital signatures and applications can
ensure that they are issued by their claimed issuer accounts. The only difference is that the process of
message authentication is performed by the AscEE internally and an application does not explicitly verify cryptographic
signatures.

Each execution session has a list of authenticated messages. Each authenticated message has an index which will be
used for passing the message to an application as a request parameter. The AscEE uses cryptographic signatures to
authenticate messages for user accounts. The signatures are validated during the
preprocessing step in parallel, and any type of cryptographic signature scheme can be used.

Also, applications can use built-in functions of the AscEE to generate authenticated messages in run-time.
This enables an application to authorize operations for another application even if it is not calling that
application directly.

In addition to authenticated messages, the AscEE provides a set of
cryptographic functions for validating signatures and calculating cryptographic entities. By using these functions and
passing cryptographic signatures as parameters to methods, a programmer, having users' public keys, can implement
the required logic for authorizing operations.

Authorizing operations by Authenticated Message Passing and explicit signatures eliminates the need for approval
mechanisms or call back patterns in Argennon.\footnote{The AscEE has no instructions for issuing cryptographic
signatures.}


\section{Reentrancy Protection}\label{sec:reentrancy}

The AscEE provides optional low level reentrancy protection by providing low
level \emph{entrance locks}. When an application acquires an entrance lock it cannot acquire that lock again and trying
to do so will result in a revert. The entrance lock of an application will be released when the application explicitly
releases its lock or when the call that has acquired that lock completes.

The AscEE reentrancy protection mechanism is optional. An application can allow reentrancy, it can protect only certain
areas of its code, or can completely disallow reentrancy.


\section{Deferred Calls}\label{sec:deferred-calls}

\ldots



\section{The ArgC Language}\label{sec:the-argc-language}
\note{This section is outdated...}
\subsection{The ArgC Standard Library}\label{sec:asl}

In Argennon, some applications (smart contracts) are updatable. The ArgC Standard Library is an updatable smart
contract which can be updated by the Argennon governance
system. This means that bugs or security vulnerabilities in the ArgC Standard Library could be quickly patched and
applications could benefit from bugfixes and improvements of the ArgC Standard Library even if they are
non-updatable. Many important and useful functionalities,
such as fungible and non-fungible assets, access control mechanisms,
and general purpose DAOs are implemented in the ArgC Standard Library.

All Argennon standards, for instance ARC standard series, which defines standards regarding transferable assets,
are defined based on how a contract should use the ArgC standard library. As a result, Argennon standards are
different from conventional blockchain standards. Argennon standards define some type of standard logic and
behaviour for a smart contract, not only a set of method signatures. This enables users to expect certain type
of behaviour from a contract which complies with an Argennon standard.


\section{Data Dependency Analysis}\label{sec:concurrency}
%! Author = aybehrouz


\subsection{Memory Dependency Graph}\label{subsec:memory-dependency-graph}

Every block of the Argennon blockchain contains a list of transactions. This list is an ordered list and the
effect of its contained transactions must be applied to the AscEE state sequentially as they appear in the ordered
list.\footnote{This ordering is solely chosen by the block proposer, and users should not have any assumptions about
the ordering of transactions in a block.}

The fact that block transactions constitute a sequential list, does not imply they can not be executed and applied
to the AscEE state concurrently. Many transactions are actually independent and the order of their execution does not
matter. These transactions can be safely validated in parallel by validators.

A transaction can change the AscEE state by modifying either the code area or the AscEE heap. In Argennon, all
transactions declare the list of memory locations they want to read or write. This will enable us to determine the
independent sets of transactions which can be executed in parallel. To do so, we define the \emph{memory dependency
graph} \(G_d\) as follows:

\begin{itemize}
    \item \(G_d\) is an undirected graph.
    \item Every vertex in \(G_d\) corresponds to a transaction and vice versa.
    \item Vertices \(u\) and \(v\) are adjacent in \(G_d\) if and only if \(u\) has a memory location \(L\) in its
    writing list and \(v\) has \(L\) in either its writing list or its reading list.
\end{itemize}

If we consider a proper vertex coloring of \(G_d\), every color class will give us an independent set of
transactions which can be executed concurrently. To achieve the highest parallelization, we need to color \(G_d\)
with minimum number of colors. Thus, the \emph{chromatic number} of the memory dependency graph shows how good a
transaction set could be run concurrently.

Graph coloring is computationally NP-hard. However, in our use case we don't need to necessarily find an optimal
solution. An approximate greedy algorithm will perform well enough in most circumstances.

After constructing the memory dependency graph, we can use it to construct the
\emph{execution DAG} of transactions. The execution DAG of transaction set \(T\) is a directed acyclic
graph \(G_e = (V_e,E_e)\) which has the \emph{execution invariance} property:
\begin{itemize}
    \item Every vertex in \(V_e\) corresponds to a transaction in \(T\) and vice versa.
    \item Executing the transactions of \(T\) in any order that \emph{respects} \(G_e\) will result in
    the same AscEE state.
    \begin{itemize}
        \item An ordering of transactions of \(T\) respects \(G_e\) if for every directed edge \((u,v) \in E_e\)
        the transaction \(u\) comes before the transaction \(v\) in the ordering.
    \end{itemize}
\end{itemize}

Having the execution DAG of a set of transactions, using Algorithm~\ref{alg:exec_dag}, we can apply the transaction
set to the AscEE state concurrently, using multiple processor, while we can be sure that the resulted AscEE state will
always be the same no matter how many processor we have used.

%##\includegraphics[width=17cm]{../img/Alg1s.png}
\begin{algorithm}
    \DontPrintSemicolon
    \SetKwData{Ready}{$R_e$}\SetKwData{V}{$v_f$}\SetKwData{Graph}{$G_e$}\SetKwData{Vertices}{$V$}\SetKwData
    {Txns}{$T$}
    \KwData{The execution dag $\Graph = (\Vertices,E)$ of transaction set \Txns}
    \KwResult{The state after applying \Txns with any ordering respecting \Graph}
    \BlankLine
    \Ready $\gets$ the set of all vertices of \Vertices with in degree 0\;
    \While{$\Vertices \neq \varnothing$}
    {
        wait until a new free processor is available\;
        \If{the execution of a transaction was finished}
        {
            remove the vertex of the finished transaction \V from \Graph\;
            \For{each vertex $u \in Adj[\V]$}
            {
                \If{$u$ has zero in degree}
                {
                    $\Ready \gets \Ready \cup u$\;
                }
            }
        }
        \If{$\Ready \neq \varnothing$}
        {
            remove a vertex from \Ready and assign it to a processor\;
        }
    }
    \caption{Executing DAG transactions}\label{alg:exec_dag}
\end{algorithm}

By replacing every undirected edge of a memory dependency graph with a directed edge in such a way that the
resulted graph has no cycles, we will obtain a valid execution DAG. Thus, from a memory dependency graph different
execution DAGs can be constructed with different levels of parallelization ability.

If we assume that we have unlimited number of processors and all transactions take equal time for executing, it
can be shown that by providing a minimal graph coloring to Algorithm~\ref{alg:gen_dag} as input, the resulted
DAG will be optimal, in the sense that it results in the minimum overall execution time.

%##\includegraphics[width=17cm]{../img/Alg2s.png}
\begin{algorithm}
    \DontPrintSemicolon
    \SetKwData{Txns}{$T$}\SetKwData{Gd}{$G_d=(V_d,E_d)$}
    \SetKwInOut{Input}{input}\SetKwInOut{Output}{output}
    \Input{The memory dependency graph \Gd of transaction set \Txns\\A proper coloring of $G_d$}
    \Output{An execution dag $G_e=(V_e,E_e)$ for the transaction set \Txns}
    \BlankLine
    $V_e \gets V_d$\;
    $E_e \gets \varnothing$\;
    define a total order on colors of $G_d$\;
    \For{each edge $\{u,v\} \in E_d$}
    {
        \eIf{$color[u] < color[v]$}
        {
            $E_e \gets E_e \cup (u,v)$\;
        }{
            $E_e \gets E_e \cup (v,u)$\;
        }
    }
    \caption{Constructing an execution DAG}\label{alg:gen_dag}
\end{algorithm}

The block proposer is responsible for proposing an efficient execution DAG alongside his proposed block. This
execution DAG will determine the ordering of block transactions and help validators to validate transactions in
parallel. Since with better parallelization a block can contain more transactions, a proposer is incentivized enough
to find a good execution DAG for transactions.

\subsection{Memory Spooling}\label{subsec:spooling}

When two transactions are dependant and they are connected with an edge \((u,v)\) in the execution DAG,
the transaction \(u\) needs to be run before the transaction \(v\). However, if \(v\) does not read any
memory locations that \(u\) modifies, we can run \(u\) and \(v\) in parallel. We just need to make sure
\(u\) does not see any changes \(v\) is making in AscEE memory. This can be done by appropriate versioning
of the memory locations which is shared between \(u\) and \(v\). We call this method \emph{memory spooling}.
After enabling memory spooling between two transactions the edge connecting them can be safely removed from the
execution DAG\@.

\subsection{Concurrent Counters}\label{subsec:concurrent-counters}

We know that in Argennon every transaction needs to transfer its proposed fee to the \texttt{feeSink} accounts
first. This essentially makes every transaction a reader and a writer of the memory locations which store the
balance record of the \texttt{feeSink} accounts. As a result, all transactions in Argennon will be dependant and
parallelism will be completely impossible. Actually, any account that is highly active, for example the account
of an exchange or a payment processor, could become a concurrency bottleneck in our system which makes all
transactions interacting with them dependant.

This problem can be easily solved by using a concurrent counter for storing the balance record of this type of
accounts. A concurrent counter is a data structure which improves concurrency by using multiple memory locations for
storing a single counter. The value of the concurrent counter is equal to the sum of its sub counters and it can
be incremented or decremented by incrementing/decrementing any of the sub counters. This way, a concurrent
counter trades concurrency with memory usage.

Algorithm~\ref{alg:CC} implements a concurrent counter which returns an error when the value of the counter
becomes negative.

%##\includegraphics[width=17cm]{../img/Alg3s.png}
\begin{algorithm}
    \DontPrintSemicolon
    \SetKwData{CC}{Counter}
    \SetKwFunction{Inc}{Increment}\SetKwFunction{Dec}{Decrement}\SetKwFunction{AtomInc}{AtomicIncrement}
    \SetKwFunction{AtomDec}{AtomicDecrement}\SetKwFunction{AtomSet}{AtomicSet}\SetKwFunction{Get}{GetValue}
    \SetKwFunction{Acquire}{Lock.Acquire}\SetKwFunction{Release}{Lock.Release}
    \SetKwProg{Fn}{Function}{}{}
    \Fn{\Get{\CC}}
    {
        $s \gets 0$\;
    \Acquire{}\;
    \For{$i \gets 0$ \KwTo $\CC.size - 1$}
    {
        $s \gets s + \CC.cell[i]$\;
    }
    \Release{}\;
    \KwRet{s}\;
    }
    \BlankLine
    \Fn{\Inc{\CC, value, seed}}
    {
        $i \gets seed \bmod \CC.size$\;
    \AtomInc{$\CC.cell[i]$, value}\;
    }
    \BlankLine
    \Fn{\Dec{\CC, value, seed, attempt}}
    {
        \If {attempt = \CC.size}
        {
            restore \CC by adding back the subtracted value\;
            \KwRet{Error}\;
        }
        $i \gets seed \bmod \CC.size$\;
        $i \gets (i + attempt) \bmod \CC.size$\;
    \eIf {$\CC.cell[i] \geq value$}
    {
        \AtomDec{$\CC.cell[i]$, value}\;
    }{
        $r \gets value - \CC.cell[i]$\;
        \AtomSet{$\CC.cell[i]$, $0$}\;
        \Dec{\CC, r, seed, $attempt + 1$}\;
    }
    }
    \caption{Concurrent counter}\label{alg:CC}
\end{algorithm}

It should be noted that in a blockchain application we don't have concurrent threads and therefore we don't need
atomic functions. For usage in a smart contract, the atomic functions of this pseudocode can be implemented like
normal functions.

Concurrent counter data structure is a part of the ArgC standard library, and any smart contract can use this data
structure for storing the balance record of highly active accounts.


    \chapter{The Argennon Prover Machine}\label{ch:APM}
    %! Author = aybehrouz
%! Date = 9/2/22

The Argennon Prover Machine (APM) is a virtual machine tailored for efficient verification of AscEE computations by
argument systems. The APM has a minimal RISC architecture with a very compact instruction set. This ensures that
its transition function has an efficient circuit complexity. Here by circuit complexity we mean the size
of the smallest circuit that, given two adjacent states in the trace, verifies that the transition between the two
states indeed respects the APM specification. The APM is a stack machine and has a random access key-value memory.
The word size of the APM is 64-bit.

The Argennon Prover Machine gets as its primary input a vector $(\mathbf{C}_H,\mathbf{C}_P,\mathbf{C}_R)$ where
$\mathbf{C}_H$ is a commitment to the AscEE heap, $\mathbf{C}_P$ is a commitment to the AscEE program area and
$\mathbf{C}_R$ is a commitment to an ordered list of requests. The APM then outputs the updated
commitments to the AscEE heap and program area and a final $\mathbf{accept}$ flag which indicates if the execution has
ended successfully or not: $(\mathbf{C}_{H'},\mathbf{C}_{P'},\mathbf{accept})$.

Producing the required outputs from these inputs is not computationally feasible so the APM receives an auxiliary
input vector $(H,\pi_H,P,\pi_P,R,\pi_R,h)$, where $\pi_X$ is a proof that proves $X$ can be opened w.r.t
$\mathbf{C}_X$ and $h$ is a hint that helps the APM make nondeterministic choices correctly.

The APM consists of four main modules:
\begin{itemize}
    \item \textbf{Preprocessor:} This module prepares the input data for other modules. It verifies that $H,
    P,R$ are valid w.r.t the provided commitments. It also processes the input data and ensures it has the
    correct format and is valid. For example it verifies the signatures of authenticated messages or
    checks that the proposed chunk bounds are valid.
    \item \textbf{Normal Execution Unit (NEU):} This module executes the requests whose execution completes
    normally. It outputs the updated heap chunks and an $accept_N$ flag. More formally it evaluates
    $H',accept_N=P(H,R_{good})$, where $H'$ is the updated heap chunks and $accept_N$ flag will be set to false, If an
    application terminates abruptly.
    \item \textbf{Failure Repeater Unit (FRU):} This module evaluates $accept_F=P(H',R_{bad})$, where $R_{bad}$ is
    those requests whose execution terminates abnormally, $H'$ is the output of the NEU and $accept_F$ flag is set
    to false if a request execution completes normally.

    The FRU has a higher circuit complexity than the NEU\@. Unlike the NEU, the FRU is able to restore the initial state
    of the heap when an application fails. It also calculates the execution cost of every instruction and if the
    application's execution cost exceeds its predefined cap, the FRU will terminate the application.
    \item \textbf{Postprocessor:} This module is responsible for calculating the updated commitments
    $(\mathbf{C}_{H'}, \mathbf{C}_{P'})$ and the final $\mathbf{accept}$ flag. Installing new applications or
    updating the code of existing application is performed by this module.
\end{itemize}


The configuration of the APM can be considered as a tuple
\[
(\mathcal{S},\mathcal{L}_{\text{NEU}},\mathcal{L}_{\text{FRU}},\mathcal{T}_{\text{NEU}},
\mathcal{T}_{\text{FRU}})\ ,
\]
where $\mathcal{S}$ is the size of the internal stack, $\mathcal{L}_{\text{NEU}},\mathcal{L}_{\text{FRU}}$ are the
amount of local memory of the normal execution and failure repeater units respectively, and
$\mathcal{T}_{\text{NEU}},\mathcal{T}_{\text{FRU}}$ are the number of cycles that the NEU and FRU runs for. It should
be noted that the APM does not use a different local memory for each application call.

%
%\begin{tikzpicture}
%    \draw (0,0) -- (4,0) -- (4,4) -- (0,4) -- (0,0);
%\end{tikzpicture}


    \chapter{Persistence Layer}\label{ch:persistance}
    %! Author = aybehrouz

The Argennon Smart Contract Execution Environment has two persistent memory areas: \emph{program area}, and \emph{heap}.
Program area stores the ASAR and the APM code of
applications\footnote{also it stores applications' constants.}, and heap stores heap chunks.
Both of these data elements can be considered as continuous arrays of bytes.
Throughout this chapter, we shall call these data elements \emph{chunks}.


\section{Storage Pages}\label{sec:storage-pages}

In the AscEE persistence layer, similar objects are clustered together and constitute a bigger data element which we
call a \emph{page}.\footnote{we avoid calling them clusters, because usually a cluster refers to a \emph{set}. AscEE
chunk clusters are not sets. They are ordered lists, like a page containing an ordered list of words or sentences.}
A page is an ordered list of an arbitrary number of chunks. Every page has a \emph{native} chunk that has the same
identifier as the page. In addition to the native chunk, a page can host any number of \emph{migrant} chunks.
A page of the AscEE storage should consists of chunks that have very similar access patterns. Ideally, when a page is
needed for validating a block, almost all of its chunks should be needed for either reading or writing. We prefer
that the chunks are needed for the same access type. In other words, the chunks of a page should be chosen in a way that
for validating a block, we need to either read all of them or modify\footnote{and probably read.} all of them.

When a page contains migrants, its native chunk can not be migrated. If the page does not
have any migrants, its native chunk can be migrated and after that the page will be converted into a special
\texttt{<moved>} page. When a non-native chunk is migrated to another page, it will be simply removed from the page.

\section{Publicly Verifiable Database Servers}\label{sec:zk-edb}

Pages of the AscEE storage are persisted using \emph{dynamic universal accumulators}. Argennon
has two dynamic accumulators: \emph{program} database, which stores the AscEE program area, and \emph{heap} database,
which stores the AscEE heap. The commitment of these accumulators are included in every block of the Argennon
blockchain. In the Argennon cloud, nodes that store these accumulators are called Publicly Verifiable Database
(PV-DB) servers.

We consider the following properties for a PV-DB:
\begin{itemize}
    \item The PV-DB contains a mapping from a set of keys to a set of values.
    \item Every state of the database has a commitment \(C\).
    \item The PV-DB has a method \((D, \pi) = \text{get}(x)\), where \(x\) is a key and \(D\) is the associated data
    with \(x\), and \(\pi\) is a proof.
    \item A user having \(C\) and \(\pi\) can verify that \(D\) is really associated with \(x\), and \(D\) is not
    altered. Consequently, a user who can obtain \(C\) from a trusted source does not need to trust the PV-DB\@.
    \item Having \(\pi\) and \(C\) a user can compute the commitment \(C'\) for the database in which \(D'\) is
    associated with \(x\) instead of \(D\).
\end{itemize}

The commitments of the AscEE cryptographic accumulators are affected by the way data chunks are clustered in pages.
Therefore, the Argennon clustering algorithm has to be a part of the consensus protocol.

Every block of the Argennon blockchain contains a set of \emph{clustering directives}. These directives
can only modify pages that were used for validating the block, and can
include directives for moving a chunk from one page to another or directives specifying which pages will contain
newly created chunks. These directives are applied at the end of block validation, after executing requests.

A block proposer is allowed to obtain clustering directives from any third party source\footnote{we can say the AscEE
clustering algorithm is essentially off-chain.}. This will not
affect Argennon security, since the integrity of a database can not be altered by clustering directives.
Those directives can only affect the performance of the Argennon network, and directives of a single block can
not affect the performance considerably.

\section{Object Clustering Algorithm}\label{sec:clustering}

\note{not yet written...}





    \chapter{Networking Layer}\label{ch:networking}


    \section{Normal Mode}\label{sec:normal-mode}
    Unlike conventional blockchains, Argennon does not use a P2P network architecture. Instead, it uses a
    client-server topology, based on a permission-less network of PVC servers. PVC servers are a
    crucial part of the Argennon ecosystem, and they form the backbone of the Argennon networking layer.
    \note{not yet written...}


    \section{Censorship Resilient Mode}\label{sec:cens-res-mode}
    \note{not yet written...}


    \chapter{The Argennon Blockchain}\label{ch:argennon-blockchain}

    \section{Blocks}\label{sec:blocks}
    %! Author = aybehrouz

The Argennon blockchain is a sequence of blocks. Every block represents an ordered list of external requests, intended
to be executed by the Argennon Smart Contract Execution Environment (AscEE). The first block of the blockchain, the
\emph{genesis} block, is a spacial block that fully describes the initial state of the AscEE. Every block of the
Argennon blockchain thus corresponds to a unique AscEE state which can be calculated deterministically from the genesis
block.

A block of the Argennon blockchain contains the following information:

\begin{center}
    \begin{tabular}{||c||}
        \hline
        \textbf{Block} \\ [0.6ex]
        \hline\hline
        height: $h$                          \\ [1.2ex]
        commitment to the program database: $\mathbf{C}_{P}$             \\ [1.2ex]
        commitment to the heap database: $\mathbf{C}_{H}$               \\ [1.2ex]
        commitment to the ordered list of requests: $\mathbf{C}_{R}$        \\ [1.2ex]
        clustering directives: $dir$                        \\ [1.2ex]
        certificate of the validators assembly for \\
        the block with height $h - k$: $v \mhyphen cert_{h-k}$         \\ [1.2ex]
        certificate of the delegates for\\
        the previous block: $d \mhyphen cert_{h-1}$        \\ [1.2ex]
        previous block hash                           \\ [1.2ex]
        \hline
    \end{tabular}
\end{center}

\subsection{Block Validation}\label{subsec:block-validation}

To validate a block three main conditions need to be verified: (i) commitments to the program and heap
database result from applying the request list and clustering directives of the block to the previous AscEE
state, (ii) $v \mhyphen cert_{h-k}$ is valid, (iii) previous block is valid and has height $h - 1$.

We can denote condition (i) by a Computational Integrity (CI) statement:
\begin{equation}
    \label{eq:ci}
    \mathbf{C}_{P_h},\mathbf{C}_{H_h} \coloneqq \tau(\mathbf{C}_{P_{h-1}},\mathbf{C}_{H_{h-1}},\mathbf{C}_{R_{h}},
    dir_h)\ ,
\end{equation}
where $\tau$ is a transition function that encodes the AscEE computation logic and the necessary preprocessing and
postprocessing steps\footnote{including opening and updating the state commitments}.

Verifying statement~\ref{eq:ci} can be done by either replaying the AscEE computation and performing the required
preprocessing and postprocessing steps, or alternatively by
verifying a computation proof. To use computation proofs the verifier and prover need to share an arithmetized
representation of $\tau$ and use it for both proof generation and verification.

The Argennon Prover Machine provides a convenient and universal arithmetization for any Argennon
application. Moreover, since a compiled version of every Argennon application to the APM code is stored in the AscEE
program area, validators do not need to locally store the APM code of applications.

The Argennon protocol does not enforce the usage of the APM. Validators and PVC servers can use different methods
for arithmetization. This naturally would require that the ASAR of an application is used
for generating the appropriate arithmetization.

The certificate of validators, $v \mhyphen cert_{h-k}$, is an aggregate signature; validating it requires accessing the
AscEE heap state at block $h-k-1$ to obtain public keys and stake values. Again, in addition to direct verification,
a validator can use computation proofs, received from the Argennon cloud, to perform cheaper verification.

For validating the previous block, instead of directly validating the content of the block, a validator only
verifies the block certificates. Verifying $d \mhyphen cert_{h-1}$ is straightforward and verifying $v \mhyphen
cert_{h-1}$ can be cheaply done by using computation proofs, obtained from the Argennon cloud. This type of block
validation only validates the transition from block $h-1$ to block $h$, and the block is valid only if its previous
block is valid too. We
call this type of block verification \emph{conditional} block validation, since the validity of the current block is
conditioned on the validity of the previous block.

In summary, a validator validates a block by verifying three computation proofs: $\pi_{\tau}$ which proves the
transition is correct, $\pi_{h-k}$ which proves the validity of the included certificate of block $h-k$ and $\pi_{h-1}$
which proves the previous block has a correct certificate of validators. It should be noted that these proofs are not
a part of the block contents.

Interestingly, conditional block validation of multiple blocks can be done in parallel. Moreover, the required proofs
can be generated independently and by different PVC servers. As we will see in
Section~\ref{subsec:validators-committee}, this property plays an important role in the Argennon consensus protocol.

\subsection{Block Certificate}\label{subsec:block-certificate}

An Argennon block certificate is an aggregate signature of some predefined subset of accounts. This predefined subset
is called the certificate assembly or committee and their signature ensures that the certified block is conditionally
valid given the validity of some previous block.

Because it is not usually possible to collect the signatures of all members of a certificate committee, an Argennon
block certificate essentially is an Accountable-Subgroup Multi-signature (ASM).

Argennon uses a parallel algorithm to produce block certificates and therefore the signature scheme needs to satisfy
certain properties:
\begin{itemize}
    \item \textbf{Associative aggregation:} the signature aggregation operator is associative.
    \item \textbf{Efficient cancellation:} if $S$ is the predefined set of users that we need their aggregate
    signature, verifying an aggregate signature of $S-U$ can be done in time $O(T+|U|)$, if the aggregate
    signature of $S$ can be verified in $O(T)$.
\end{itemize}

An example for a signature scheme that supports all these properties is the BLS signature scheme.

\subsubsection{BLS Signatures}

The BLS signature scheme operates in a prime order group and supports simple threshold signature generation,
threshold key generation, and signature aggregation. To review, the scheme uses the following ingredients:

\newcommand{\G}{\mathbb{G}}
\newcommand{\Z}{\mathbb{Z}}
\newcommand{\adv}{{\cal A}}
\newcommand{\bdv}{{\cal B}}
\newcommand{\deq}{\mathrel{\mathop:}=}
\newcommand{\SK}{\mathit{sk}}
\newcommand{\PK}{\mathit{pk}}
\newcommand{\C}{\mathit{cert}}
\newcommand{\APK}{\mathit{apk}}
\newcommand{\DPK}{\mathit{\Delta pk}}
\newcommand{\MM}{\mathcal{M}}
\newcommand{\xwedge}{\, \operatorname{\text{$\wedge$}}\, }
\newcommand{\abs}[1]{\lvert #1 \rvert}
\newcommand{\Hm}{H_0}
\newcommand{\Hpk}{H_1}
\newcommand{\qHpk}{Q_{\Hpk}}
\newcommand{\qHm}{Q_{\Hm}}
\newcommand{\qsig}{Q_{\text{sig}}}

\begin{itemize}
    \item An efficiently computable \emph{non-degenerate} pairing $e:\G_0 \times \G_1 \to \G_T$
    in groups $\G_0$, $\G_1$ and $\G_T$ of prime order $q$. We let $g_0$ and $g_1$ be generators
    of $\G_0$ and $\G_1$ respectively.
    \item A hash function $H_0: \mathcal{M} \rightarrow \mathbb{G}_0$, where $\mathcal{M}$ is the message space.
    The hash function will be treated as a random oracle.
\end{itemize}

The BLS signature scheme is defined as follows:

\begin{itemize}
    \item $\textbf{KeyGen}()$: choose a random $\alpha$ from $\Z_q$ and set $h \gets g_1^\alpha \in \G_1$.
    output $\PK \deq (h)$ and $\SK \deq (\alpha)$.
    \item $\textbf{Sign}(\SK, m)$: output $\sigma \gets \Hm(m)^\alpha \in \G_0$.
    The signature $\sigma$ is a \emph{single} group element.
    \item $\textbf{Verify}(\PK,m,\sigma)$: if $e(g_1, \sigma) = e\big(\PK,\ \Hm(m)\big)$  then output "accept",
    otherwise output "reject".
\end{itemize}

Given triples $(\PK_i,\ m_i,\ \sigma_i)$ for $i=1,\ldots,n$,
anyone can aggregate the signatures $\sigma_1,\ldots,\sigma_n \in \G_0$
into a short convincing aggregate signature $\sigma$ by computing
\begin{equation}
    \label{eq:agg}
    \sigma \gets \sigma_1 \cdots \sigma_n \in \G_0\ .
\end{equation}
Verifying an aggregate signature $\sigma \in \G_0$ is done by checking that
\begin{equation}
    \label{eq:aggdiff}
    e(g_1, \sigma) = e\big(\PK_1,\ \Hm(m_1)\big) \cdots e\big(\PK_n,\ \Hm(m_n)\big)\ .
\end{equation}
When all the messages are the same ($m = m_1 = \ldots = m_n$), the verification relation~\eqref{eq:aggdiff} reduces to
a simpler test that requires only two pairings:
\begin{equation}
    \label{eq:aggsame}
    e(g_1, \sigma) = e\Big(\PK_1 \cdots \PK_n,\ \Hm(m)\Big)\ .
\end{equation}
We call $\APK=\PK_1 \cdots \PK_n$ the aggregate public key.

To defend against \emph{rogue public key} attacks, Argennon uses Prove Knowledge of the Secret Key (KOSK) scheme. As we
explained in Section~\ref{sec:accounts}, when an account is created its public keys need to be registered in
the ARG smart contract. Therefore, the KOSK scheme can be easily implemented in Argennon.

Argennon uses a simple ASM scheme based on BLS aggregate signatures.
Argennon block certificates constitute an ordered sequence based on the order of blocks they certify. If we show
the $i$-th certificate\footnote{note that the $i$-th certificate is not
necessarily the certificate of the $i$-th block.} of committee $C$ with $\C_i$, and the set of signers
with $S_i$, then the block certificate $\C_i$ can be considered as a tuple:
\begin{equation}
    \C_i=(\sigma_i,\ C-S_{i})\label{eq:cert}\ ,
\end{equation}
where $\sigma_i$ is the aggregate signature issued by $S_i$.

The aggregate public key of the certificate can
be calculated from:
\begin{equation}
    \APK_i=\APK_C\APK_{C-S_i}^{-1}\label{eq:aggCertPK}\ ,
\end{equation}
where $\APK_{A}$ shows the aggregate public key of all accounts in $A$.

Alternately, we can use $\APK_{i-1}$ to calculate the aggregate public key:
\begin{equation}
    \APK_i=\APK_{i-1}\APK_{S_i-S_{i-1}}\APK_{S_{i-1}-S_i}^{-1}\ .\label{eq:aggPK-2}
\end{equation}

When an Argennon account is created, both its $\PK$ and $\PK^{-1}$ is registered in the ARG smart contract, so the
inverse of any aggregate public key can be easily computed.\footnote{since the group operator of a cyclic
group is commutative, we have $(ab)^{-1}=a^{-1}b^{-1}$.}




    \section{Consensus}\label{sec:consensus}
    %! Author = aybehrouz

The credibility of a block of the Argennon blockchain is determined by the certificates it receives
from different sets of users, known as committees. There are two primary types of certificate committee in
Argennon: the committee of \emph{delegates} and the assembly of \emph{validators}. Argennon has \emph{one} committee
of delegates and $m$ assemblies of validators.

The committee of delegates issues a certificate for every block of the Argennon blockchain, and each
assembly of validators issues a certificate every $m$ blocks. A validator assembly will
certify a block only if it has already been certified by the committee of delegates. Every assembly of validators has
an index between $0$ and $m - 1$, and it issues a certificate for block with height $h$, if $h$ modulo $m$ equals
the assembly index.

Every block of the Argennon blockchain needs a certificate from both the committee of delegates and
the assembly of validators. A block is considered final after its \textbf{next} block receives \textbf{both} of
its certificates. In Argennon as long as more than half of the total stake of validators is controlled by honest users,
the probability of discarding a final block is near zero even if all the delegates are malicious.

In addition to primary committees, Argennon has several community driven committees. Certificates of these
committees are not required for block finality, but they could be used by members of validator assemblies to better
decide about the validity of a block.

When an anomaly is detected in the consensus mechanism, the \emph{recovery} protocol is initiated by validators. The
recovery protocol is designed to be resilient to many types of attacks in order to be able to restore the normal
functionality of the system.

\subsection{The Committee of Delegates}\label{subsec:the-committee-of-delegates}

The committee of delegates is a small committee of trusted delegates, elected by Argennon users through the
Argennon Decentralized Autonomous Governance system (ADAGs\footnote{pronounced \textipa{/eI-dagz/}.}).
At the start of the Argennon main-net, this committee will be elected for one-year terms and will have five members.
Later, this could be changed by the ADAGs in a procedure described in Section~\ref{sec:adags}.

The committee of delegates is responsible for creating new blocks of the Argennon blockchain, and issues a
certificate for every block of the Argennon blockchain. The certificate needs to be signed
by \textbf{all} the committee members in order to be considered valid.

Besides the main committee, a reserve committee of delegates consisting of three members is elected by users
either through the ADAGs or by \emph{emergency agreement} during the recovery protocol. In case the main committee
fails to generate new blocks or behaves maliciously, the task of
block generation will be assigned to the reserve committee until the main committee comes back online or a new
committee is elected through the ADAGs.

A block certified only by the committee of delegates is relatively credible, but it is not considered final
until its next block receives the certificate of its validator assembly. Since a block at height $h$ contains the
validators certificate of the block at height $h-k$, the unvalidated part of the Argennon blockchain can not be longer
than $k$ blocks.

The committee of delegates may use any type of agreement protocol to reach consensus on the
next block. Usually the delegates are large organizations, and they can communicate with each
other efficiently using their reliable networking infrastructure. This mostly eliminates the complications of their
consensus protocol and any protocol could have a good performance in practice. Usually a very simple and fast protocol
can do the job: one of the members is randomly chosen as the proposer, and other members vote ``yes'' or ``no'' on the
proposed block. For better performance, the delegates should run their agreement protocol for reaching consensus about
small batches of transactions in their mem-pools, instead of the whole next block.

If one of the delegates loses its network connectivity, no new blocks can be generated. For this reason,
the delegates should invest on different types of communication infrastructure, to make sure they will never lose
connectivity to each other and to the Argennon network.

\subsection{The Assemblies of Validators}\label{subsec:validators-committee}

The Argennon protocol calculates a stake value for every account, which is an estimate of a user's stake in the
system, and is measured in ARGs. Any account whose stake value is higher than
\texttt{minValidatorStake} threshold is considered a \emph{validator}.
The \texttt{minValidatorStake}
threshold is determined by the ADAGs, but it can never be higher than $1000$ ARGs.

Every \texttt{AssemblyLifeTime} number of blocks, randomly $m$ assemblies are selected from
validators, in a way that the total stakes of different assemblies are almost equal, and every
account is a member of \textbf{at least} one assembly.\footnote{An account can be a member of multiple assemblies.}

The value of $m$ is determined by the ADAGs, but it can never be higher than $32$. This way, it is guaranteed
that on average, any block of the Argennon blockchain is validated by at least $2\%$ of the total ARG supply.

\subsubsection{Activity Status}

Every validator has a status which can be either \texttt{online} or \texttt{offline}.
This status is stored in the ARG smart contract and is part of the staking database. A validator can change
his status to \texttt{offline} through an external request (transaction) to the ARG application. In this request he
exactly specifies for how long he wants to be offline and after this period his status will be automatically considered
\texttt{online} again. When a validator sets his status to \texttt{offline} for some period of time, he
will receive a small portion of the maximum possible reward that a validator can receive in that period of time by
actively participating in the consensus protocol. This ensures that a validator has an incentive for changing his
status to offline rather than simply becoming inactive.

The staking database of a validator assembly can be updated only by the assembly itself. That means, an external
request which changes the status of a validator can be included only in a block that is validated by the assembly of
that validator.

There is no transaction type for changing the status of a validator to \texttt{online}. A malicious delegates committee
would be able to censor this type of transactions and prevent honest validators from coming back online. For this reason
the status of a validator is considered online automatically when the specified period of time for being offline ends.

A block certificate issued by some members of a validators assembly is considered valid, if according to
the staking database of the previous block \textbf{certified by the same assembly}, we have:\footnote{If we calculate
the stake values based on the previous block, a malicious assembly can select the validators of the next block by
manipulating the staking database.}
\begin{itemize}
    \item The total stake of \texttt{online} members of the assembly is higher than \texttt{minOnlineStake} fraction
    of the total stake of the assembly. This threshold can be changed by the ADAGs, but it can never be lower
    than $2/3$.
    \item All signers of the certificate have \texttt{online} status.
    \item The sum of stake values of the certificate signers is higher than $3/4$ of the total stake
    of the assembly members that have \texttt{online} status.
\end{itemize}

If according to the staking database of block $h$, the total online stake of the assembly with index $h$ modulo $m$ is
lower than \texttt{minOnlineStake} threshold, the block $h + m$ can never be certified by validators. To prevent the
blockchain from halting in such situations, the validator assembly with index $h$ modulo $m$ will get merged into the
assembly that has the most online stake at block $h$. This will decrease the number of assemblies to $m-1$, and the
indices of assemblies will be updated appropriately.

The merging will continue recursively until the online stake of all remaining assemblies is higher than
\texttt{minOnlineStake} fraction.
If eventually all assemblies get merged together and only one assembly remains, the condition
for validity of block certificates changes: A certificate of validators will be considered valid if the sum of stakes
of the certificate signers is higher than $2/3$ of the total stake of validators and \texttt{online/offline}
status of validators becomes ineffective.

\subsubsection{Signing the Block Certificate}

The delegates can generate blocks very fast. Consequently, the Argennon blockchain always has an
unvalidated part which contains the blocks that have a certificate from the committee of delegates but have not
yet received a certificate from the validators.

As we mentioned before, the block with height $h$ needs a certificate from the assembly of
validators with index $h$ modulo $m$. To decide about signing the certificate of a block which already has
a certificate from the delegates, a validator checks the conditional
validity of the block (See Section~\ref{subsec:block-validation}), and if the block is valid, he issues
an ``accept'' signature. If the block is invalid, he initiates the recovery protocol. The validator will broadcast the
certificate \textbf{only after} he sees the certificate of the validator assembly of the previous block.
Some validators may also require seeing a certificate from
some community based committees. An honest validator never signs a certificate for two different blocks with the
same height.

Consequently, in Argennon the block validation by assemblies is performed in parallel, and validators
do not wait for seeing the validators certificate of the previous block to start block validation. On the
other hand, the block certificates are published and broadcast sequentially. A validator does not publish his
signature, if the certificate of the previous block validator assembly has not been published yet. This ensures that
an invalid fork made by malicious delegates will not receive any certificates from any validator assemblies.

\subsubsection{Analysis}

We analyze the minimum amount of stake that is required for conducting different types of attacks against the Argennon
blockchain. In these attack scenarios, we assume that a single validator assembly is corrupted, all the delegates
are malicious and the adversary is able to fully control message transmission and partition the network arbitrarily.

We denote the total stake of the corrupted validator assembly with $s$ and the total stake of malicious users of the
assembly with $m$. We use $d$ to denote the stake of users of the assembly who have \texttt{offline} status and $h$
to denote the stake of users of the assembly who have \texttt{online} status and do not participate in the
protocol.\footnote{$d$ stands for \emph{deactivated} and $h$ stands for \emph{hidden}.} We assume
that a certificate is accepted if it is signed by more than $r$ fraction of the total online stake of the assembly.
We obtain the minimum required malicious stake for three types of attacks:
\begin{itemize}
    \item Confirming an invalid block:
    \[ m > r(s-d) \]
    \item Forking the blockchain by double voting and network partitioning:
    \[m > (2r-1)(s-d)+h \]
    \item Halting the blockchain by refusing to vote:
    \[m > (1-r)(s-d)-h \]
\end{itemize}

In Argennon we have $r=\frac{3}{4}$ and $d < \frac{1}{3}s$.
Consequently, in Argennon to confirm an invalid block, the adversary needs at least $\frac{1}{2}$ of the total stake
of an assembly. For forking the blockchain, interestingly $m$ is
minimized when $h=0$, and $\frac{1}{3}s$ is the minimum required stake. For halting the blockchain,
an adversary requires a stake higher than $\frac{1}{6}s-h$.

In particular, we are interested in comparing the Argennon protocol with a simple protocol that accepts a certificate
if it is signed by more than $\frac{2}{3}$ of the total stake of the assembly and there is no \texttt{online/offline}
status for users.

To compare the minimum stake required for halting the blockchain in two protocols, we can solve the following
inequality:
\[
    \frac{1}{4}(s-d)-h > \frac{1}{3}s-(d+h)\ \cdot
\]
We observe that as long as $d>\frac{1}{9}s$, the minimum required stake for halting the blockchain is higher in the
Argennon protocol and the value of $h$ does not matter.

\subsubsection{Signature Aggregation}\label{subsubsec:sig-agg}

The validators certificate of a block is an aggregate signature of members of the corresponding assembly of
validators. Validator assemblies could include millions of users and calculating their aggregate
signature requires an efficient distributed algorithm.

In Argennon, signature aggregation is mostly performed by PVC servers. To distribute the aggregation workload
between different servers, every validator assembly is divided into pre-determined groups, and each PVC
server is responsible for signature aggregation of one group. To make sure that there is enough redundancy, the
total number of groups should be less than the number of PVC servers and each group should be assigned to
multiple PVC servers.

Any member of a group knows all the servers that are responsible for signature aggregation of his group. When a member
signs a block certificate, he sends his signature to all the severs that aggregate the signatures of his group.
These PVC servers aggregate the signatures they receive and then send the aggregated signature to the delegates.
Furthermore, the delegates aggregate these signatures to produce the final block certificate
and then broadcast it to the PVC servers network.

The role of the delegates in the signature aggregation algorithm is limited. The important part of the work is done by
PVC servers and slightly modified versions of this algorithm can perform signature aggregation
even if all the delegates are malicious, as long as there are enough honest PVC servers.

\subsection{The Recovery Protocol}\label{subsec:recovery}

The recovery protocol is a resilient protocol designed for recovering the Argennon blockchain from critical situations.
In the terminology of the CAP theorem, the recovery protocol is designed to choose consistency over availability,
and is not a protocol supposed to be executed occasionally. Ideally this protocol should never be used
during the lifetime of the Argennon blockchain.

We assume that an adversary is able to fully control message transmission between users and is able to partition the
network arbitrarily for finite periods of time. Under these circumstances, the recovery protocol can recover the
functionality of the Argennon blockchain as long as more than $2/3$ of the total stake of every validator assembly is
controlled by honest users. The recovery protocol uses two main emergency procedures to
recover the functionality of the Argennon blockchain: \emph{emergency forking} and \emph{emergency
agreement} protocol.

\subsubsection{Emergency Forking}

The reserve committee of delegates is able to fork the Argennon blockchain, if it receives a valid fork request
from the validators.
This fork needs to be confirmed by validators and can not discard more than one block that has been certified by
validators.
A valid fork request is an unexpired request signed by more than half of the total stake of validators.

For forking at block $h$, the reserve committee of delegates
makes a special \emph{fork block} which only contains a valid fork request, and its parent is the block $h$.
The height of the fork block therefore is $h + 1$, and the fork block needs a valid certificate from the assembly of
validators with index $h+1$ modulo $m$. When a
fork block gets certified by validators, its parent is also confirmed and will become a part of the blockchain, even if
it does not have a validators certificate.

For signing a fork block at height $h+1$, a validator ensures that the following conditions hold:
\begin{itemize}
    \item the fork block is signed by the reserve committee.
    \item the fork block contains a valid fork request.
    \item the parent block of the fork block is issued by the previous committee.
    \item the parent block of the fork block is certified by validators, or the parent block is conditionally
    valid and there is a fork block with height $h$ which is certified by validators, or the parent block is
    conditionally valid and the parent block of the parent has a validators certificate.
    \item the validator has not already signed a certificate for a fork or normal block at height $h+1$.
\end{itemize}

The parent of the fork block does not necessarily need a validators certificate. This enables the
reserve committee to recover the liveliness of the blockchain in a situation where a malicious committee has
generated multiple blocks at the same height. Notice that the block before the parent always needs a validators
certificate.

A validator always chooses a valid fork block over a block of the main chain and may sign different fork
blocks with different heights. However, as we mentioned before, an honest validator
\textbf{never signs a certificate for two different blocks with the same height}. Consequently, a validator never
signs two fork blocks at the same height, and if he has already signed the fork block at height $h+1$, he will not
sign the block $h+1$ of the main chain and vice versa.

The reserve committee of delegates is allowed to generate multiple fork blocks with different heights, as long as the
parent block is generated by the previous committee. When the reserve committee generates multiple fork blocks
at different heights, the next normal block must be always added after the fork block with the highest height.

The reserve committee of delegates should try to perform the emergency forking in such a way that
valid blocks do not get discarded, including blocks that have not been certified by the validators yet.

For forking the blockchain, the reserve committee uses a straightforward algorithm: let $h_v$ be the height of the last
block with a validator certificate and $h_v+k$ be the height of the last valid block that the reserve committee has
seen. For forking the main chain, the reserve committee generates all fork blocks with heights $h_v+1,h_v+2,\dots,
h_v+k+1$. The parent of the fork block with height $h_v+i$ will be the block $h_v+i-1$ of the main chain. The
reserve committee will wait until the fork block with height $h_v+k+1$ receives a certificate from validators and
then will continue the normal chain after that fork block. Hence, the fork block with height $h_v+k+1$
will be the parent of the first normal block generated by the reserve committee.

\paragraph{Analysis}

When the reserve committee gets activated, the main committee might have been malicious, so any number of blocks could
exist at each height. However, at each height at max one block can have a validators certificate. Moreover, if at some
height there are not any blocks with a validators certificate, then no blocks at higher heights can have a
validators certificate either, because validators do not sign the certificate of a normal block if its parent does
not have a certificate.

If $h_{\max}$ denotes the height of the highest block with a validators certificate, as long as more than $2/3$ stake
of every assembly of validators are honest, for a fork block with height $h_f$ we have:\footnote{This fork block
forks the blockchain at height $h_f-1$.}
\begin{itemize}
    \item if $h_f \leq h_{\max}$, the fork block can not receive a certificate from validators.
    \item if $h_f = h_{\max} + 1$, the fork block may receive a certificate from validators or not. It is possible that
    the validators of the assembly with index $h_f$ modulo $m$ get divided between the fork block and a block at
    height $h_f$ of the main chain.
    \item if $h_f = h_{\max} + 2$, the fork block can always receive a certificate from validators, if network
    partitions do not last forever.
    \item if $h_f \geq h_{\max} + 3$, the fork block can always receive a certificate from validators \textbf{only if} a
    fork block at height $h_f-1$ gets certified by validators.
\end{itemize}

Consequently, if two fork blocks at heights $h_0$ and $h_0+k$ are generated by the reserve committee, if both blocks
receive a certified from validators, we must have $h_0 > h_{\max}$, and there must exist fork blocks with heights
$h_0+1,\dots,h_0+k-1$ which are certified by validators. That means if a malicious reserve committee generates a normal
block after any fork block with height less than $h_0+k$, that normal block can not receive a certificate from
validators.

During the emergency forking, at max one block with a validators certificate may be discarded. This happens when
the malicious main committee of delegates has forked the main chain by producing blocks $b_1$ and $b_2$ at height
$h$; the block $b_1$ has received a validators certificate and validators have not certified any blocks at height $h+1$.
The reserve committee adds a fork block whose parent is block $b_2$ and that will essentially discard $b_1$. Notice
that if a block at height $h+1$ had been certified by validators the fork block could not have received a certificate
from validators.

\subsubsection{Emergency Agreement}

The emergency agreement protocol is a resilient protocol for deciding between a set of proposals when no committee of
delegates can be trusted. For initiating the protocol, a validator signs a message containing the subject
of the agreement and a start time.

A validator enters the agreement protocol if he
receives a request that is signed by more than half of the total stake
of the validators and its start time has not passed. The validator calculates
the stake values based on the staking database of the last \textbf{final} block in his blockchain without
considering the \texttt{online}/\texttt{offline} status of validators.

The emergency agreement protocol is essentially an election procedure and involves human interaction. Users need to
determine who they want to vote for by interacting with the software. As long as users can not agree upon electing a
candidate, the voting process has to continue.

The voting process is done in rounds and each round usually lasts for approximately $\lambda$ units of time.
$\lambda$ is selected by th ADAGs and could be several hours. All votes and messages \textbf{are tagged} in a way
that a vote cast in a round can not be used in another round. Votes are weighted based on users' stakes and
\texttt{online/offline} status of users is not considered. When we say 2/3 votes, we mean the sum of the stake of
voters is 2/3 of the total stake.

A user executes the following procedure in each agreement session:
\paragraph{Voting Phase in Round $r$:}
\begin{itemize}
    \item if the user has locked his vote on a proposal $p$, he votes $p$, otherwise he votes a single desired
    proposal.
    \item when $clock = \lambda$, if the users has seen more than $2/3$ votes for a proposal $p$, he votes $p$-win,
    otherwise he votes $draw$. A user votes either $p$-win or $draw$, not both.
    \item when $clock = k \lambda$ for $k=2,3,\dots$, if the user has seen more than $2/3$ votes for a proposal
    $p$ he votes $p$-lock.
    \item if the users sees more than 2/3 $draw$ votes, he goes to the round $r+1$ and sets $clock=0$.
    \item if the user sees more than 2/3 $p$-lock votes, he goes to the round $r+1$, sets $clock=0$ and locks his
    vote on $p$.
\end{itemize}

\paragraph{Termination:}
\begin{itemize}
    \item as soon as the user sees more than 2/3 $p$-win votes for $p$, he selects $p$ and ends
    the agreement protocol. p-win votes can be for any round but all must belong to the same round.
\end{itemize}


We assume that users have clocks with the same speed, and $\lambda \gg \epsilon$, where $\epsilon$ is the maximum
clock difference between users. We also assume that more than 2/3 of the total stake of the system is controlled
by honest users, and network partitions are resolved after a finite amount of time. With these assumptions it can be
shown that the emergency recovery protocol has the following important properties:
\begin{itemize}
    \item no two users will end the agreement protocol with two different proposals as the result of the agreement.
    \item if honest users can agree upon some proposal value, the agreement protocol will converge to that value
    after a finite number of sessions.
\end{itemize}

A honest user during a round only votes a single proposal and either votes $p$-win or $draw$. This ensures that
as long as more than $2/3$ of the total stake of the system is controlled by honest users, no two different proposals
can get more than $2/3$ votes. In addition, only one of $p$-win or $draw$ can get more than $2/3$ votes. When a
user sees more than $2/3$ $p$-win votes, he can be sure that more than $1/3$ of the \textbf{honest}
stake has voted $p$-win, so $draw$ has less than $2/3$ votes. Therefore, for going to the next round we
will need 2/3 $p$-lock votes and all honest users will lock their vote on $p$. As a result only $p$ can be confirmed by
the agreement protocol.

If honest users can agree upon some proposal value, the agreement protocol will converge to that value. When a
single hones user terminates the protocol, he can convince all other honest users to terminate their protocol by
sending those 2/3 $p$-win votes that he has seen. When we go to the next round all honest users will lock their vote on
the same proposal and an agreement could be reached in next rounds. We will never get stuck in a round. If at some
round $draw$ gets less than 2/3 votes, that means at least $\epsilon$ honest stake has voted
$p$-win.\footnote{It's possible that the $\epsilon$ stake has not voted yet. However based on the finite time
partitioning assumption, at some point that honest stake should get connected to the network and vote.}
Consequently, there must be more than 2/3 votes for some proposal $p$ which convinced the $\epsilon$ honest stake to
vote $p$-win. Therefore, after waiting long enough, all the honest stake will see those votes and will eventually
vote for $p$-lock, so $p$-lock will get more than 2/3 votes.

\subsubsection{Initiating the Recovery Protocol}

When the validator software does not receive any blocks for \texttt{blockTimeOut} amount of time, or when it observes an
evidence which proves the delegates are malicious, after prompting the user and after his confirmation, it will
initiate the recovery protocol.

To do so, first the validator software activates the censorship resilient mode of the networking module, then it checks
the validity of the blocks that do not have a validators certificate and determines the last valid block of its
version of the blockchain.

In the next step, it will sign and broadcast an \textbf{emergency fork request} message, alongside some useful metadata
such as the last valid block of its blockchain and the evidence of delegates' misbehaviour.\footnote{this metadata is
not a part of the fork request.} Before starting the recovery protocol, validators try to synchronize their blockchains
as much as possible.

If the reserve committee of delegates is already active, or if the validator software sees a valid fork request signed
by more than half of the total online stake of the validators, but does not receive the fork
block after a certain amount of time, after user confirmation, it will sign and broadcast a request for
\textbf{emergency agreement} on a new reserve committee. The agreement on new delegates usually needs user
interaction and is not a fully automatic process.

The evidence which proves a committee of delegates is malicious is an invalid block that is signed by at least
one delegate:
\begin{itemize}
    \item a block that is not conditionally valid
    \item two different blocks with the same parent
\end{itemize}

\subsection{Estimating Stake Values}\label{subsec:user's-stake}

In a proof of stake system the influence of a user in the consensus protocol should be proportional to the amount
of stake the user has in the system. Conventionally in these systems, a user's stake is considered to be equal with the
amount of native system tokens, he has ``staked'' in the system. A user stakes his tokens by locking them in
his account or a separate staking account for some period of time. During this time, he will not be able to transfer
his tokens.

Unfortunately, there is a subtle problem with this approach. It is not clear in a real world economic system
how much of the main currency of the system can be locked and kept out of the circulation indefinitely. It seems that
this amount for currencies like US dollar, is quite low comparing to the total market cap of the currency.
This means that for a real world currency this type of staking mechanism will result in putting the
fate of the system in the hands of the owners of a small fraction of the total supply.

To mitigate this problem, Argennon uses a hybrid approach for estimating the stake of a user.
Every \texttt{stakingDuration} blocks, which is called a \emph{staking period}, Argennon calculates
a \emph{trust value} for each user.

The user's stake
at time step \(t\), is estimated based on the user's trust value and his ARG balance:
\begin{equation}
    S_{u,t} = \min (B_{u,t}, Trust_{u,k})\ ,\label{eq:stake}
\end{equation}
where:
\begin{itemize}
    \item \(S_{u,t}\) is the stake of user \(u\) at time step \(t\).
    \item \(B_{u,t}\) is the ARG balance of user \(u\) at time step \(t\).
    \item \(Trust_{u,k}\) is an estimated trust value for user \(u\) at staking period \(k\).
\end{itemize}

Argennon users can lock their ARG tokens in their account for any period of time. During this time a user
will not be able to transfer his tokens and there is no way for cancelling a lock.
The trust value of a user is calculated based on the amount of his locked tokens and the
Exponential Moving Average (EMA) of his ARG balance:
\begin{equation}
    Trust_{u,k} = L_{u,k} + M_{u,t_k}\ ,\label{eq:trust}
\end{equation}
where
\begin{itemize}
    \item $L_{u,k}$ is the amount of locked tokens of user $u$, whose release time is \textbf{after the end} of
    the staking period $k+1$.
    \item $M_{u,t_k}$ is the Exponential Moving Average (EMA) of the ARG balance of user \(u\) at time step \(t_k\).
    $t_k$ is the start time of the staking period $k$.
\end{itemize}

In Argennon a user who held ARGs and participated in the consensus for a long time is more trusted
than a user with a higher balance whose balance has increased recently. An attacker who has obtained a large
amount of ARGs, also needs to hold them for a long period of time before being able to attack the system.

For calculating the EMA of a user's balance at time step \(t\), we can use the following
recursive formula:
\[
    M_{u,t} = (1 - \alpha) M_{u,t-1} + \alpha B_{u,t} = M_{u,t-1} + \alpha (B_{u,t} - M_{u,t-1})\ ,
\]
where the coefficient \(\alpha\) is a constant smoothing factor between \(0\) and \(1\), which represents the
degree of weighting decrease. A higher \(\alpha\) discounts older observations faster.

Usually an account balance will not change in every time step, and we can use older values of EMA for calculating
\(M_{u,t}\): (In the following equations the \(u\) subscript is dropped for simplicity)
\[
    M_{t} = (1 - \alpha)^{t-k}M_{k} + [1 - (1 - \alpha)^{t - k}]B\ ,
\]
where:
\[
    B = B_{k+1} = B_{k+2} = \dots = B_{t}\ \cdot
\]
We know that when \(|nx| \ll 1\) we can use the binomial approximation \({(1 + x)^n \approx 1 + nx}\). So, we can
further simplify this formula:
\[
    M_{t} = M_{k} + (t - k) \alpha (B - M_{k})\ \cdot
\]

For choosing the value of \(\alpha\) we can consider the number of time steps that the trust value of a user needs
for reaching a specified fraction of his account balance. We know that for large \(n\) and \(|x| < 1\) we have
\((1 + x)^n \approx e^{nx}\), so by letting \(M_{u,k} = 0\) and \(n = t - k\) we can write:
\begin{equation}
    \alpha =- \frac{\ln\left(1 - \frac{M_{n+k}}{B}\right)}{n}\ .\label{eq:alpha}
\end{equation}
The value of \(\alpha\) for a desired configuration can be calculated by this equation. For instance, we could
calculate the \(\alpha\) for a relatively good configuration in which \(M_{n+k} = 0.8B\) and \(n\) equals to the
number of time steps of 10 years.

\subsection{Analysis}\label{subsec:consensus-math}
\note{not yet written...}


    %! Author = aybehrouz


\section{Applications}\label{sec:applications}

An Argennon application or smart contract is an HTTP server which is represented by an Argennon Standard
Representation (ASR) and whose state is stored in the Argennon blockchain. Each Argennon application is identified by
a unique application identifier.

An application identifier, \texttt{applicationID}, is
a unique prefix code generated by the \emph{applications} prefix tree. (See Section~\ref{sec:identifiers}.)
An application identifier can be considered as the address of an application and has the following standard symbolic
representation:
\begin{verbatim}
<application-id> ::= <decimal-prefix-code>
<decimal-prefix-code> ::= <dec-num>"."<decimal-prefix-code> | <dec-num>
\end{verbatim}
where \texttt{<dec-num>} is a normal decimal number between $0$ and $255$. For example \texttt{21.255.37},
\texttt{0}, \texttt{11.6} and \texttt{2.0.0.0.0}, are valid application addresses.

Argennon has two special smart contracts: the \emph{root smart contract}, also called the \emph{root application}, and
the \emph{ARG smart contract}, which is also called the \emph{Argennon smart contract} or the \emph{ARG application}.

Argennon application use HTTP as the application protocol and they are advised to have a RESTful API design.

\subsection{The Root Application}\label{subsec:the-root-app}

The root application or the root smart contract, with \texttt{applicationID = 0}, is a privileged smart contract
responsible for installation/uninstallation of other smart contracts. The Argennon's root smart contract
performs three main operations:

\begin{itemize}
    \item Installation of new Argennon applications and determining the update policy of a smart
    contract: if the contract is updatable or not, which accounts or smart contracts can update or uninstall
    the contract, and so on.
    \item Removing an Argennon application (if allowed).
    \item Updating an Argennon application (if allowed).
\end{itemize}

The root smart contract is a mutable smart contract and can be updated by the Argennon governance system.
(See Section~\ref{sec:adags})

\subsection{The ARG Application}\label{subsec:the-arg-app}

The ARG application or the ARG smart contract,
with \texttt{applicationID = 1}, controls the ARG token, the main
currency of the Argennon blockchain. This smart contract also manages a database of public keys and
handles signature verification.

The ARG smart contract is a mutable smart contract and can be updated by the Argennon governance system.


\section{Accounts}\label{sec:accounts}

Argennon accounts are entities defined inside the ARG application.
Every Argennon account is uniquely identified by a prefix code generated using \emph{accounts} prefix
tree. (See Section~\ref{sec:identifiers}) An account
identifier can be considered as the address of an account and has the following standard symbolic representation:
\begin{verbatim}
<account-id> ::= "0x"<hex-num>
\end{verbatim}
where \texttt{<hex-num>} is a hexadecimal number, using lower case
letters \texttt{[a-f]} for showing digits greater than $9$.

For example \texttt{0x24ffda}, \texttt{0x0} and \texttt{0x03a0000}, are valid standard symbolic
representations of account addresses.

A new account can be created by sending a proper HTTP request to the ARG smart contract. For creating
a new account two public keys need to be provided by the caller and registered in the Argennon smart contract.
One public key will be used for issuing digital signatures, and the other one will be used for voting. The
provided public keys need to meet certain cryptographic requirements,\footnote{Argennon uses Prove
Knowledge of the Secret Key (KOSK) scheme.} and can not be already registered in the system.

If the owner of the new account is an application, the \texttt{applicationID} of the owner will be registered in the
ARG smart contract and no public keys are needed. An application can own an arbitrary number of accounts.

\note{Explicit key registration enables Argennon to decouple cryptography from the blockchain design. In this way,
    if the cryptographic algorithms used become insecure for some reason, for example because
    of the introduction of quantum computers, they could be easily upgraded.}


\section{Transactions}\label{sec:transactions}

An Argennon transaction consist of an HTTP request made by a user to an Argennon application, a resource
declaration object and a list of signed messages. Transactions can only be
issued by users and applications can not create transactions. An Argennon transaction is also called
an \emph{external request}.

\subsection{Resource Declaration}\label{subsec:resource-declaration}

Every Argennon transaction is required to provide the following information as an upper bound for the
resources it needs:

\begin{itemize}
    \item Maximum AscEE clocks
    \item The list of applications the request will call
    \item The list of access blocks the request needs
    \item \texttt{maxSize} for chunks it wants to expand
    \item \texttt{minSize} for chunks it wants to shrink
    \item A list of applications it will update (if any)
\end{itemize}

If a transaction tries to violate any of these predefined limitations, it will be considered failed, and the network
can receive the proposed fee of that transaction.

\begin{lstlisting}[language=python, frame=TB, float, title=An Argennon transaction in YAML format,
    label={lst:txn-example}]
---
request: |
    PATCH /balances/0x95ab HTTP/1.1
    Content-Type: application/json; charset=utf-8
    Content-Length: 46

    {"to":0xaabc,"amount":1399,"sig":0}

messages:
   - issuer: 0x95ab000000000000
     msg: {"to":0xaabc000000000000,"amount":1399,"forApp":0x100000000000000,"nonce":11}
     sig: LNUC49Lhyz702uszzNcfaU3BhPIbdaSgzqDUKzbJzLPTlFS2J9GzHl-cDKb

caps:
    maxClocks: 150 # maximum number of AVM execution clocks
    apps: [1,124.16]
    read: [(2654,3),(15642,0),(15642,1),(15642,3)]
    write: [(15642,0),(20154,0),(20154,1)]
\end{lstlisting}


\section{Resource Management}\label{sec:res-man}

Completing an execution session requires computational resources. The amount of resources used by an execution session
should be monitored and managed, otherwise a malicious user would be able to easily spam and exhaust resources of the
execution environment.

In most consensus protocols, we can assume that the block proposer has enough incentive to filter out transactions
that spam run-time resources. Here by a run-time resource, we mean a resource that at run-time, a limited amount
of it is available, but its surplus can not be stored for later use. Execution time and local memory are
examples of such a resource but permanent storage is not.

If a proposed block contains many transactions which need a lot of run-time resources, validators would not
be able to validate all transactions in a timely manner. Consequently, they may decide to reject the block or if they
spend enough resources, the confirmation of that block could take more than usual. Longer block time is not
favoured by block proposers, because it means less throughput of the system which usually means less overall rewards
for them. In the Argennon protocol the management of run-time resources, is left to the block proposer.

Resource usage can be measured per session or per application call. Obviously per session measurement is
easier and more efficient. However, when we measure resources per session if
a session violates its resource caps, determining the point of failure may require precise and error-free resource
measurement. For example, assume that a session containing an application call
violates a 2 milliseconds execution time cap. If in the caller's code, the call happens exactly after 2 milliseconds,
a small fluctuation in the execution time measurement can
change the point of failure between the called and the caller application. Note that for implementing optional
decoupling principle in Argennon, we need to determine the
exact application call which has failed in a call chain, and this fluctuation introduces a nondeterministic
behaviour which could make block validation impossible. That's why we need per application call
resource management sometimes. If we define the resource caps per application call and
perform our measurements for each application call separately, by using caps that are larger than the measurement
error, errors in the measurements would not change the point of failure in the call chain.

The AscEE has two types of execution sessions: \emph{optimistic} and \emph{monitored}. Resource usage of an optimistic
session is always measured per session and \textbf{default} pre-defined resource caps are used. On the other hand, for a
monitored session, resources that can not be measured precisely are measured per application call and their caps are
determined per application call by the external HTTP request (i.e.\ transaction). The block proposer decides the
type of the execution session for each transaction.


Different computational resources are measured and monitored during an AscEE session:
\begin{itemize}
    \item \textbf{execution time}:
    is the amount of cpu time that is required for executing a session or an application call. The execution time is
    measured in \emph{AscEE clocks}. One AscEE clock is defined as 1/1000 of the amount
    of \textbf{cpu time} needed for executing a predefined standard application which is used for benchmarking a host's
    performance. In other words, by definition, the AscEE imaginary standard machine completes the standard benchmark
    in 1000 clocks.

    Optimistic sessions have a predefined \texttt{maxClocks} value which is determined by the Argennon protocol. This
    value defines a bound on the \textbf{total} AscEE clocks of the session, and no per application measurement is done.

    Monitored sessions perform per application call cpu-time measurement, and every application call during a monitored
    session has a separate \texttt{maxClocks} value. This value determines the maximum amount of time that the cpu
    can be used for executing that particular application call. It should be noted that the cpu timer of the
    application call is paused when the application makes a call to another application, and is resumed when the
    control returns. An application call
    needs at least 10 clocks and if the value of its \texttt{maxClock} is lower than this value the call will be
    considered a failed call.

    Each application call has some amount of \texttt{externalClocks}. When an application makes a request to another
    application it has to \emph{forward} a portion of its \textbf{external} clocks to the called application. This
    amount will determine the value of \texttt{maxClocks} for the called application, and is subtracted from the value
    of \texttt{externalClocks} of the caller.

    The amount of external clocks of an application
    call is defined to be \(2/3\) of its \texttt{maxClocks}. As a result, the total number of clocks of a monitored
    session is always less than \(3 \times \texttt{maxClocks}\) of the root application call. The value of
    \texttt{maxClocks} for
    the root application call is determined by the external request (i.e.\ transaction).

    \item \textbf{local memory}:
    any memory usage of an application that is not part of a heap chunk and is not part of another measured resource
    will be considered as local memory usage. Local memory is not persistent and when an application finishes serving a
    request and returns the HTTP response (i.e.\ the application call completes) its local memory is deleted.
    This resource is measured in bytes.

    Optimistic sessions measure local memory usage per session and enforce a protocol-defined cap on the total amount
    of local memory a session can use. Monitored sessions measure local memory usage per
    application call and enforce a protocol-defined cap for each application call separately. An application call
    which tries to use more local memory than the cap, fails.
    \item \textbf{heap access list}:
    every session can only access heap locations that are declared in its access list. In addition,
    resizing heap chunks can only be done in the range of the pre-declared lower bound and upper bound.
    \item \textbf{app access list}:
    a session may only make requests to applications that are declared in its application access list.
    \item \textbf{call depth}:
    during a session the number of nested application calls can not be more than a threshold. This threshold is
    determined by the Argennon protocol. It should be noted that a differed call is considered like a normal call and
    increases the call depth by one level.
    \item \textbf{differed calls}:
    every application call can have a limited number of differed calls which is determined by the protocol.
    To simplify the implementation, this limit is defined per application call. Since in Argennon the call depth is
    limited, a per-application call limit will also define an implicit limit for the total number of active differed
    calls.
    \item \textbf{virtual signatures}:
    \item \textbf{number of entrance locks}:
\end{itemize}

In Argennon, execution time and local memory are considered \emph{nondeterministic} resources. A nondeterministic
resource is a resource that can not be measured precisely and its measurement always contains a random error.
Optimistic sessions are not allowed to fail because of violating nondeterministic resource limits. As a result, the
block proposer must always choose a monitored session for a transaction that is included in the block and violates a
nondeterministic resource cap.

When a transaction fails due to the violation of a limit for a nondeterministic resource, the proposer is
required to exactly specify the application call which violates that limit in the execution session. When validators are
executing a monitored session, for each application call, they enforce considerably larger limits for
nondeterministic resources. Only in case the proposer has declared that an application call violates a limit, the
validators will enforce the actual value of the limit. This simple mechanism ensures that, with a very high
probability, validators agree with the proposer, although their measurements could be different.




    \section{Incentive mechanism}\label{sec:incentive-mechanism}
    %! Author = aybehrouz

\subsection{Fees}\label{subsec:fees}

The Argennon protocol does not explicitly define any fees for normal transactions. Only for high priority
transactions a fixed fee is determined by the governance system (See Section ...). Because the protection of the
Argennon network against spams and DOS attacks is mostly done by the delegates, they are also responsible for
determining and collecting transaction fees. A good fee collection policy could considerably increase the chance of
delegates for being reelected in the next terms, therefore they are incentivized to use creative and effective
methods\footnote{For example, they may
allow a limited number of free transactions per month for every account.}.

In Argennon fee payment can be done off-chain or on-chain. Off-chain fee payment is more efficient and flexible but
requires some level of trust in the delegates. For trust-less fee payment, the Argennon protocol provides the
concept of request attachments (See Section~\ref{sec:attachments}).
When a user does not want to use off-chain fee payment methods, he can simply define his transaction as the attachment
of the fee payment transaction. That way, the fee payment transaction will be performed only if the attached
transaction is also included in the same block.

While transaction fee is not enforced by the Argennon protocol, there are other types of fee that are mandatory: the
\emph{database fee} and the \emph{block fee}. Both of these fees are required to be paid for every block of the Argennon
blockchain and are paid by the delegates. The block fee is a constant fee that is paid for each new block of the
blockchain and its amount is
determined by the ADAGs. The database fee depends on the data access and storage overhead that a new block
is imposing on the Argennon storage cloud. The amount of this fee is determined by the ADAGs, and is collected in
a special account: the \texttt{dbFeeSink}.

\subsection{Certificate Rewards}\label{subsec:rewards}

The validators who sign the certificate of a block will receive the block fee paid for that block. Every validator
will be rewarded
proportional to his stake (i.e voting power). As we mentioned before the block fee is a constant fee which the
delegates pay for each block.

Rewards will not be distributed instantly, instead they will be distributed at the end of the staking period.
This will facilitate efficient implementations which avoid frequent updates in the Argennon storage.

As long as ARG is allowed to be minted and its cap is not reached, the delegates will receive a reward at the
\textbf{end} of their election term. This reward will consist of newly minted ARGs, and its amount will be
determined by the ADAGs. In addition, for each block certificate that is added to the Argennon blockchain some amount
of ARGs will be minted and added to the \texttt{dbFeeSink} account.

\subsection{Penalties}\label{subsec:penalties}

If an account behaves maliciously, and that behaviour could not have happened due to a mistake, by providing a proof
in a block, the account will be disabled forever in the ARG smart contract. Disabling an account in the
ARG smart contract will prevent that account from signing any valid signatures in the future.

Punishable behaviours include:
\begin{itemize}
    \item Signing a certificate for a block that is not conditionally valid.
    \item Signing a certificate for two different blocks at the same height if none of them
    is a fork block or a seal block.\footnote{Signing
    a fork block and a normal block at the same height usually is a malicious behaviour. However, it will not be
    penalized because there are circumstances that an honest user could mistakenly do that.}
\end{itemize}

\subsection{Incentives for PVC Servers}\label{subsec:PVC-servers}

The incentive mechanism for PVC servers should have the following properties:

\begin{itemize}
    \item It incentivizes storing all storage pages and not only those pages that are used more frequently.
    \item It incentivizes PVC servers to actively provide the required storage pages for validators.
    \item Making more accounts will not provide any advantage for a PVC server.
\end{itemize}

For our incentive mechanism, we require that every time a validator receives a storage page from a PVC, after
validating the data, he gives a receipt to the PVC server. In this receipt the validator signs the
following information:

\begin{itemize}
    \item \texttt{ownerAddr}: the account address of the PVC server.
    \item \texttt{receivedPageID}: the ID of the received page.
    \item \texttt{round}: the current block number.
\end{itemize}

\note{In a round, an honest validator never gives a receipt for an identical page to two different PVC servers.}

To incentivize PVC servers, a lottery will be held every round,\footnote{A round is the time interval between
two consecutive blocks.} and a predefined amount of ARGs from
\texttt{dbFeeSink} account will be distributed between the winners as a prize. This prize will be divided equally
between all \emph{winning tickets} of the lottery.

\note{One PVC server could own multiple winning tickets in a round.}

To run this lottery, every round, based on the current block seed, a collection of \emph{valid} receipts will be
selected randomly as the \emph{winning receipts} of the round. A receipt is \emph{valid} in round $r$ if:

\begin{itemize}
    \item The signer was a member of the validators' committee of the block $r - 1$ and signed the block certificate.
    \item The page in the receipt was needed for validating the \textbf{previous} block.
    \item The receipt round number is $r - 1$.
    \item The signer did not sign a receipt for the same storage page for two different PVC servers in
    the previous round.
\end{itemize}
For selecting the winning receipts we could use a random generator:
\begin{verbatim}
IF random(seed|validatorPK|receivedPageID) < winProbability THEN
    the receipt issued by validatorPK for receivedPageID is a winner
\end{verbatim}
\begin{itemize}
    \item \texttt{random()} produces uniform random numbers between 0 and 1, using its input argument as a seed.
    \item \texttt{validatorPK} is the public key of the signer of the receipt.
    \item \texttt{receivedPageID} is the ID of the storage page that the receipt was issued for.
    \item \texttt{winProbability} is the probability of winning in every round.
    \item \texttt{seed} is the current block seed.
    \item \texttt{|} is the concatenation operator.
\end{itemize}

Also, based on the current block seed, a random storage page is
selected as the challenge of the round. A PVC server that owns a winning receipt needs to broadcast a \emph{winning
ticket} to claim his prize. The winning ticket consists of a winning receipt and a \emph{solution} to the round
challenge. Solving a round challenge requires the content of the storage page which was selected as the round
challenge. This will encourage PVC servers to store all storage pages.

A possible choice for the challenge solution could be the cryptographic hash of the content of the challenge
page combined with the server account address:

\texttt{hash(challenge.content|ownerAddr)}

The winning tickets of the lottery of round $r$ need to be included in the block of the round $r$,
otherwise they will be considered expired. However, finalizing and prize distribution for the winning tickets
should be done in a later round. This way, \textbf{the content of the challenge page could be
kept secret during the lottery round.} Every winning ticket will get an equal share of the lottery prize.



    \chapter{Governance}\label{ch:governance}


    \section{ADAGs}\label{sec:adags}
    The Argennon Decentralized Autonomous Governance system (ADAGs)
    \note{not yet written...}


    \chapter{The Argon Language}\label{ch:argon-lang}
    \input{sections/argon}
\end{document}


